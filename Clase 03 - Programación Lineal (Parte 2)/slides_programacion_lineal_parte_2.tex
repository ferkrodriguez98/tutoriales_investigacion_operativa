\documentclass{beamer}
\usetheme{metropolis}

\usepackage[spanish]{babel}
\usepackage[utf8]{inputenc}
\usepackage{graphicx}
\usepackage{tikz}
\usepackage{xcolor}
\usepackage{amsmath}
\usepackage{listings}
\usepackage{pgfplots}
\pgfplotsset{compat=1.18}
\usepackage{booktabs}
\usepackage{fontawesome5}

% Definición de colores personalizados
\definecolor{primary}{RGB}{46, 204, 113}
\definecolor{secondary}{RGB}{52, 152, 219}
\definecolor{accent}{RGB}{231, 76, 60}
\definecolor{background}{RGB}{236, 240, 241}
\definecolor{gradient1}{RGB}{255, 107, 107}
\definecolor{gradient2}{RGB}{255, 159, 67}

% Configuración del tema
\setbeamercolor{normal text}{fg=black,bg=background}
\setbeamercolor{structure}{fg=primary}
\setbeamercolor{alerted text}{fg=accent}

\definecolor{lightgray}{rgb}{0.95,0.95,0.95}
\definecolor{darkgreen}{rgb}{0,0.5,0}
\definecolor{darkblue}{rgb}{0,0,0.5}

\lstset{
  backgroundcolor=\color{lightgray},
  basicstyle=\tiny\ttfamily,
  keywordstyle=\color{darkblue}\bfseries,
  commentstyle=\color{darkgreen},
  stringstyle=\color{red},
  numbers=left,
  numberstyle=\tiny\color{gray},
  stepnumber=1,
  numbersep=5pt,
  showspaces=false,
  showstringspaces=false,
  showtabs=false,
  frame=single,
  tabsize=2,
  language=Python,
  breaklines=true,
  breakatwhitespace=true
}

\usetikzlibrary{arrows.meta}
\tikzset{
    vertex/.style={circle, draw, minimum size=20pt, inner sep=0pt},
    edge/.style={-{Latex[length=3mm]}, thick},
    cost/.style={midway, fill=white}
}

\title{\Huge\textbf{Programación Lineal - Pt 2}}
\subtitle{Problemas de Redes}
\author{Investigación Operativa}
\date{}

\begin{document}

\begin{frame}
    \titlepage
    \begin{tikzpicture}[remember picture,overlay]
        \node[anchor=south west,inner sep=30pt] at (current page.south west) {
            \includegraphics[height=1cm]{../misc/UdeSA.png}
        };
    \end{tikzpicture}
\end{frame}

\begin{frame}{¿Qué son los Problemas de Redes?}
    \textbf{Definición:}
    \begin{itemize}
        \item Tipo especial de programación lineal
        \item Variables representan flujos entre nodos
        \item Nodos conectados por arcos con capacidades
        \item Visualización gráfica facilita el análisis
    \end{itemize}
    
    \vspace{1em}
    \textbf{Ventajas:}
    \begin{itemize}
        \item Estructura predefinida
        \item Fácil interpretación
        \item Algoritmos especializados más eficientes
    \end{itemize}
\end{frame}

\begin{frame}{Tipos de Problemas de Redes}
    \textbf{Principales categorías:}
    \begin{itemize}
        \item[{\makebox[1em][c]{\textcolor{black}{\faTruck}}}] \textbf{Transporte:} Envío desde orígenes a destinos
        \item[{\makebox[1em][c]{\textcolor{black}{\faTrain}}}] \textbf{Transshipment:} Permite nodos intermedios
        \item[{\makebox[1em][c]{\textcolor{black}{\faTint}}}] \textbf{Flujo Máximo:} Maximizar flujo en la red
        \item[{\makebox[1em][c]{\textcolor{black}{\faRoute}}}] \textbf{Asignación:} Matching uno a uno
    \end{itemize}
\end{frame}

\begin{frame}{Problema de Transporte}
    \textbf{Características:}
    \begin{itemize}
        \item Múltiples orígenes con capacidades
        \item Múltiples destinos con demandas
        \item Costo por unidad transportada
        \item Objetivo: minimizar costo total
    \end{itemize}
    
    \vspace{1em}
    \textbf{Aplicaciones:}
    \begin{itemize}
        \item Distribución de productos
        \item Logística de almacenes
        \item Asignación de recursos
    \end{itemize}
\end{frame}

\begin{frame}{Ejemplo: Fábricas y Centros de Distribución}
    \textbf{Situación:}
    \begin{itemize}
        \item 3 fábricas con capacidades: 100, 150, 200 unidades
        \item 4 centros con demandas: 120, 80, 150, 100 unidades
        \item Costos de transporte variables
    \end{itemize}
    
    \vspace{1em}
    \begin{table}[H]
        \centering
        \tiny
        \begin{tabular}{c|cccc}
            \toprule
            & CD1 & CD2 & CD3 & CD4 \\
            \midrule
            F1 & 10 & 8 & 6 & 5 \\
            F2 & 7 & 9 & 8 & 6 \\
            F3 & 8 & 7 & 5 & 9 \\
            \bottomrule
        \end{tabular}
    \end{table}
\end{frame}

\begin{frame}{Representación Gráfica}
    \begin{center}
        \begin{tikzpicture}[node distance=4cm]
            % Nodos de origen (fábricas)
            \node[vertex] (F1) at (0,6) {F1};
            \node[vertex] (F2) at (0,3) {F2};
            \node[vertex] (F3) at (0,0) {F3};
            
            % Capacidades de las fábricas
            \node[left] at (F1.west) {100};
            \node[left] at (F2.west) {150};
            \node[left] at (F3.west) {200};
            
            % Nodos de destino (centros de distribucion)
            \node[vertex] (CD1) at (8,6) {CD1};
            \node[vertex] (CD2) at (8,4) {CD2};
            \node[vertex] (CD3) at (8,2) {CD3};
            \node[vertex] (CD4) at (8,0) {CD4};
            
            % Demandas de los centros
            \node[right] at (CD1.east) {120};
            \node[right] at (CD2.east) {80};
            \node[right] at (CD3.east) {150};
            \node[right] at (CD4.east) {100};
        \end{tikzpicture}
    \end{center}
\end{frame}

\begin{frame}{Representación Gráfica}
    \begin{center}
        \begin{tikzpicture}[node distance=4cm]
            % Nodos de origen (fábricas)
            \node[vertex] (F1) at (0,6) {F1};
            \node[vertex] (F2) at (0,3) {F2};
            \node[vertex] (F3) at (0,0) {F3};
            
            % Capacidades de las fábricas
            \node[left] at (F1.west) {100};
            \node[left] at (F2.west) {150};
            \node[left] at (F3.west) {200};
            
            % Nodos de destino (centros de distribucion)
            \node[vertex] (CD1) at (8,6) {CD1};
            \node[vertex] (CD2) at (8,4) {CD2};
            \node[vertex] (CD3) at (8,2) {CD3};
            \node[vertex] (CD4) at (8,0) {CD4};
            
            % Demandas de los centros
            \node[right] at (CD1.east) {120};
            \node[right] at (CD2.east) {80};
            \node[right] at (CD3.east) {150};
            \node[right] at (CD4.east) {100};
            
            % Primer grupo de caminos (F1) en azul
            \draw[edge, color=blue] (F1) -- (CD1) node[pos=0.25,above] {10};
            \draw[edge, color=blue] (F1) -- (CD2) node[pos=0.25,above] {8};
            \draw[edge, color=blue] (F1) -- (CD3) node[pos=0.25,above] {6};
            \draw[edge, color=blue] (F1) -- (CD4) node[pos=0.25,above] {5};
        \end{tikzpicture}
    \end{center}
\end{frame}

\begin{frame}{Representación Gráfica}
    \begin{center}
        \begin{tikzpicture}[node distance=4cm]
            % Nodos de origen (fábricas)
            \node[vertex] (F1) at (0,6) {F1};
            \node[vertex] (F2) at (0,3) {F2};
            \node[vertex] (F3) at (0,0) {F3};
            
            % Capacidades de las fábricas
            \node[left] at (F1.west) {100};
            \node[left] at (F2.west) {150};
            \node[left] at (F3.west) {200};
            
            % Nodos de destino (centros de distribucion)
            \node[vertex] (CD1) at (8,6) {CD1};
            \node[vertex] (CD2) at (8,4) {CD2};
            \node[vertex] (CD3) at (8,2) {CD3};
            \node[vertex] (CD4) at (8,0) {CD4};
            
            % Demandas de los centros
            \node[right] at (CD1.east) {120};
            \node[right] at (CD2.east) {80};
            \node[right] at (CD3.east) {150};
            \node[right] at (CD4.east) {100};
            
            % Todos los caminos
            \draw[edge] (F1) -- (CD1) node[pos=0.25,above] {10};
            \draw[edge] (F1) -- (CD2) node[pos=0.25,above] {8};
            \draw[edge] (F1) -- (CD3) node[pos=0.25,above] {6};
            \draw[edge] (F1) -- (CD4) node[pos=0.25,above] {5};
            \draw[edge, color=blue] (F2) -- (CD1) node[pos=0.25,above] {7};
            \draw[edge, color=blue] (F2) -- (CD2) node[pos=0.25,above] {9};
            \draw[edge, color=blue] (F2) -- (CD3) node[pos=0.25,above] {8};
            \draw[edge, color=blue] (F2) -- (CD4) node[pos=0.25,above] {6};
        \end{tikzpicture}
    \end{center}
\end{frame}

\begin{frame}{Representación Gráfica}
    \begin{center}
        \begin{tikzpicture}[node distance=4cm]
            % Nodos de origen (fábricas)
            \node[vertex] (F1) at (0,6) {F1};
            \node[vertex] (F2) at (0,3) {F2};
            \node[vertex] (F3) at (0,0) {F3};
            
            % Capacidades de las fábricas
            \node[left] at (F1.west) {100};
            \node[left] at (F2.west) {150};
            \node[left] at (F3.west) {200};
            
            % Nodos de destino (centros de distribucion)
            \node[vertex] (CD1) at (8,6) {CD1};
            \node[vertex] (CD2) at (8,4) {CD2};
            \node[vertex] (CD3) at (8,2) {CD3};
            \node[vertex] (CD4) at (8,0) {CD4};
            
            % Demandas de los centros
            \node[right] at (CD1.east) {120};
            \node[right] at (CD2.east) {80};
            \node[right] at (CD3.east) {150};
            \node[right] at (CD4.east) {100};
            
            % Todos los caminos
            \draw[edge] (F1) -- (CD1) node[pos=0.25,above] {10};
            \draw[edge] (F1) -- (CD2) node[pos=0.25,above] {8};
            \draw[edge] (F1) -- (CD3) node[pos=0.25,above] {6};
            \draw[edge] (F1) -- (CD4) node[pos=0.25,above] {5};
            \draw[edge] (F2) -- (CD1) node[pos=0.25,above] {7};
            \draw[edge] (F2) -- (CD2) node[pos=0.25,above] {9};
            \draw[edge] (F2) -- (CD3) node[pos=0.25,above] {8};
            \draw[edge] (F2) -- (CD4) node[pos=0.25,above] {6};
            \draw[edge, color=blue] (F3) -- (CD1) node[pos=0.25,above] {8};
            \draw[edge, color=blue] (F3) -- (CD2) node[pos=0.25,above] {7};
            \draw[edge, color=blue] (F3) -- (CD3) node[pos=0.25,above] {5};
            \draw[edge, color=blue] (F3) -- (CD4) node[pos=0.25,above] {9};
        \end{tikzpicture}
    \end{center}
\end{frame}

\begin{frame}{Formulación del Problema de Transporte}
    \onslide<1->{
    \textbf{Variables de decisión:}
    \begin{itemize}
        \item $x_{ij}$: cantidad transportada desde fábrica $i$ a centro $j$
    \end{itemize}
    }
    
    \vspace{1em}
    \onslide<2->{
    \textbf{Función objetivo:}
    \begin{align*}
        \text{Minimizar: } Z = \sum_{i=1}^3 \sum_{j=1}^4 c_{ij}x_{ij}
    \end{align*}
    }
    
    \onslide<3->{
    \textbf{Restricciones:}
    \begin{align*}
        \sum_{j=1}^4 x_{ij} = \text{capacidad}_i \quad &\forall i
        \hspace{2cm}
        \sum_{i=1}^3 x_{ij} = \text{demanda}_j \quad \forall j
    \end{align*}
    \begin{center}
        $x_{ij} \geq 0 \quad \forall i,j$
    \end{center}
    }
\end{frame}

\begin{frame}[fragile]{Implementación en PICOS}
    \begin{lstlisting}[language=Python]
import picos
import numpy as np

# Crear el problema
P = picos.Problem()

# Definir variables
x = picos.RealVariable('x', 12)

# Definir costos
c = np.array([10, 8, 6, 5, 7, 9, 8, 6, 8, 7, 5, 9])

# Funcion objetivo
P.set_objective('min', picos.sum([c[i] * x[i] for i in range(12)]))

# Restricciones de capacidad
P.add_constraint(sum(x[0:4]) == 100)  # F1
P.add_constraint(sum(x[4:8]) == 150)  # F2
P.add_constraint(sum(x[8:12]) == 200) # F3

# Restricciones de demanda
P.add_constraint(x[0] + x[4] + x[8] == 120)    # CD1
P.add_constraint(x[1] + x[5] + x[9] == 80)     # CD2
P.add_constraint(x[2] + x[6] + x[10] == 150)   # CD3
P.add_constraint(x[3] + x[7] + x[11] == 100)   # CD4

P.solve(solver='glpk')
    \end{lstlisting}
\end{frame}

\begin{frame}{Transporte Desbalanceado}
    \textbf{¿Qué pasa cuando oferta $\neq$ demanda?}
    \begin{itemize}
        \item<1-> \textbf{Exceso de oferta:} Introducir destino ficticio
        \item<2-> \textbf{Exceso de demanda:} Introducir origen ficticio
        \item<3-> \textbf{Costo de penalización:} Valor $M$ muy grande
    \end{itemize}
    
    \vspace{1em}
    \onslide<4->{
    \textbf{Origen/Destino ficticio:}
    \begin{itemize}
        \item Capacidad/demanda = $\left| \text{oferta} - \text{demanda} \right|$
        \item Costo = $M$ (penalización)
        \item Representa demanda insatisfecha o capacidad ociosa
    \end{itemize}
    }
\end{frame}

\begin{frame}{Ejemplo: Demanda Insatisfecha}
    \textbf{Situación desbalanceada:}
    \begin{itemize}
        \item Oferta total: 100 + 150 + 200 = 450 unidades
        \item Demanda total: 120 + 80 + 150 + 150 = 500 unidades
        \item Déficit: 50 unidades
    \end{itemize}
\end{frame}

\begin{frame}{Gráfico Demanda Insatisfecha}
    \begin{center}
    \scalebox{0.9}{
    \begin{tikzpicture}[node distance=4cm]
        % Nodos de origen (fábricas)
        \node[vertex] (F1) at (0,6) {F1};
        \node[vertex] (F2) at (0,3) {F2};
        \node[vertex] (F3) at (0,0) {F3};
        \node[vertex,dashed] (FF) at (3,8) {FF};
        
        % Nodos de destino
        \node[vertex] (CD1) at (8,6) {CD1};
        \node[vertex] (CD2) at (8,4) {CD2};
        \node[vertex] (CD3) at (8,2) {CD3};
        \node[vertex] (CD4) at (8,0) {CD4};
        
        % Capacidades y demandas
        \node[left] at (F1.west) {100};
        \node[left] at (F2.west) {150};
        \node[left] at (F3.west) {200};
        \node[left] at (FF.west) {$\infty$};
        
        \node[right] at (CD1.east) {120};
        \node[right] at (CD2.east) {80};
        \node[right] at (CD3.east) {150};
        \node[right] at (CD4.east) {150};
        
        % Arcos con costos
        \draw[edge] (F1) -- (CD1) node[pos=0.25,above] {10};
        \draw[edge] (F1) -- (CD2) node[pos=0.25,above] {8};
        \draw[edge] (F1) -- (CD3) node[pos=0.25,above] {6};
        \draw[edge] (F1) -- (CD4) node[pos=0.25,above] {5};
        \draw[edge] (F2) -- (CD1) node[pos=0.25,above] {7};
        \draw[edge] (F2) -- (CD2) node[pos=0.25,above] {9};
        \draw[edge] (F2) -- (CD3) node[pos=0.25,above] {8};
        \draw[edge] (F2) -- (CD4) node[pos=0.25,above] {6};
        \draw[edge] (F3) -- (CD1) node[pos=0.25,above] {8};
        \draw[edge] (F3) -- (CD2) node[pos=0.25,above] {7};
        \draw[edge] (F3) -- (CD3) node[pos=0.25,above] {5};
        \draw[edge] (F3) -- (CD4) node[pos=0.25,above] {9};
        \draw[edge,dashed] (FF) -- (CD1) node[pos=0.45,above] {$M$};
        \draw[edge,dashed] (FF) -- (CD2) node[pos=0.45,above] {$M$};
        \draw[edge,dashed] (FF) -- (CD3) node[pos=0.45,above] {$M$};
        \draw[edge,dashed] (FF) -- (CD4) node[pos=0.45,above] {$M$};
    \end{tikzpicture}
    }
    \end{center}
\end{frame}

\begin{frame}{Función Objetivo con Penalización}
    \textbf{Función objetivo original:}
    \begin{align*}
        \text{Minimizar: } Z = \sum_{i=1}^3 \sum_{j=1}^4 c_{ij}x_{ij}
    \end{align*}
    
    \vspace{1em}
    \textbf{Función objetivo con fábrica ficticia:}
    \begin{align*}
        \text{Minimizar: } Z = \sum_{i=1}^3 \sum_{j=1}^4 c_{ij}x_{ij} + M \sum_{j=1}^4 x_{FF,j}
    \end{align*}
    
    \vspace{1em}
    \textbf{Donde:}
    \begin{itemize}
        \item $M$ es un número muy grande (ej: 1000)
        \item $x_{FF,j}$ es la demanda insatisfecha del centro $j$
        \item La penalización hace que se evite la demanda insatisfecha
    \end{itemize}
\end{frame}

\begin{frame}{Problema de Transshipment}
    \textbf{Extensión del problema de transporte:}
    \begin{itemize}
        \item Permite nodos intermedios
        \item Los nodos pueden recibir y enviar
        \item Más flexible que transporte puro
    \end{itemize}
    
    \vspace{1em}
    \textbf{Características adicionales:}
    \begin{itemize}
        \item[{\makebox[1em][c]{\textcolor{black}{\faCube}}}] \textbf{Nodos de tránsito:} Hubs, puertos, almacenes
        \item[{\makebox[1em][c]{\textcolor{black}{\faBalanceScale}}}] \textbf{Restricciones de balance:} Entrada = Salida
        \item[{\makebox[1em][c]{\textcolor{black}{\faRoad}}}] \textbf{Múltiples rutas:} Caminos alternativos
    \end{itemize}
\end{frame}

\begin{frame}{Ejemplo: Sunco Oil Co.}
    \textbf{Situación:}
    \begin{itemize}
        \item 2 pozos (W1: 150k, W2: 200k barriles/día)
        \item 2 ciudades (NY: 140k, LA: 160k barriles/día)
        \item 2 puertos intermedios (Mobile, Galveston)
        \item Costos variables por ruta
    \end{itemize}
    
    \vspace{1em}
    \textbf{Objetivo:}
    \begin{itemize}
        \item Minimizar costo total de transporte
        \item Satisfacer demanda de ambas ciudades
        \item Usar puertos como nodos de tránsito
    \end{itemize}
\end{frame}

\begin{frame}{Red de Transshipment: Nodos}
    \begin{center}
    \begin{tikzpicture}[node distance=3cm]
        % Nodos de pozos
        \node[vertex] (W1) at (0,5) {W1};
        \node[vertex] (W2) at (0,0) {W2};
        
        % Nodos de puertos
        \node[vertex] (M) at (4,4) {M};
        \node[vertex] (G) at (4,1) {G};
        
        % Nodos de ciudades
        \node[vertex] (NY) at (8,5) {NY};
        \node[vertex] (LA) at (8,0) {LA};

        \node[left] at (W1.west) {150};
        \node[left] at (W2.west) {200};
        
        % Demandas de las ciudades
        \node[right] at (NY.east) {140};
        \node[right] at (LA.east) {160};
    \end{tikzpicture}
    \end{center}
\end{frame}

\begin{frame}{Red de Transshipment}
    \begin{center}
    \begin{tikzpicture}[node distance=3cm]
        % Nodos de pozos
        \node[vertex] (W1) at (0,5) {W1};
        \node[vertex] (W2) at (0,0) {W2};
        
        % Nodos de puertos
        \node[vertex] (M) at (4,4) {M};
        \node[vertex] (G) at (4,1) {G};
        
        % Nodos de ciudades
        \node[vertex] (NY) at (8,5) {NY};
        \node[vertex] (LA) at (8,0) {LA};

        \node[left] at (W1.west) {150};
        \node[left] at (W2.west) {200};
        
        % Demandas de las ciudades
        \node[right] at (NY.east) {140};
        \node[right] at (LA.east) {160};
        
        % Arcos desde pozos a puertos
        \draw[edge, color=blue] (W1) -- (M) node[pos=0.5, above] {10};
        \draw[edge, color=blue] (W1) -- (G) node[pos=0.3, below] {12};
        \draw[edge, color=blue] (W2) -- (M) node[pos=0.2, above] {5};
        \draw[edge, color=blue] (W2) -- (G) node[pos=0.5, above] {12};
    \end{tikzpicture}
    \end{center}
\end{frame}

\begin{frame}{Red de Transshipment}
    \begin{center}
    \begin{tikzpicture}[node distance=3cm]
        % Nodos de pozos
        \node[vertex] (W1) at (0,5) {W1};
        \node[vertex] (W2) at (0,0) {W2};
        
        % Nodos de puertos
        \node[vertex] (M) at (4,4) {M};
        \node[vertex] (G) at (4,1) {G};
        
        % Nodos de ciudades
        \node[vertex] (NY) at (8,5) {NY};
        \node[vertex] (LA) at (8,0) {LA};

        \node[left] at (W1.west) {150};
        \node[left] at (W2.west) {200};
        
        % Demandas de las ciudades
        \node[right] at (NY.east) {140};
        \node[right] at (LA.east) {160};
        
        % Arcos desde pozos a puertos
        \draw[edge] (W1) -- (M) node[pos=0.5, above] {10};
        \draw[edge] (W1) -- (G) node[pos=0.3, below] {12};
        \draw[edge] (W2) -- (M) node[pos=0.2, above] {5};
        \draw[edge] (W2) -- (G) node[pos=0.5, above] {12};
        
        % Arcos desde pozos a ciudades
        \draw[edge, color=blue] (W1) -- (NY) node[pos=0.5, above] {45};
        \draw[edge, color=blue] (W1) -- (LA) node[pos=0.3, above] {28};
        \draw[edge, color=blue] (W2) -- (NY) node[pos=0.3, above] {26};
        \draw[edge, color=blue] (W2) --(LA) node[pos=0.5, below] {25};
    \end{tikzpicture}
    \end{center}
\end{frame}

\begin{frame}{Red de Transshipment}
    \begin{center}
    \begin{tikzpicture}[node distance=3cm]
        % Nodos de pozos
        \node[vertex] (W1) at (0,5) {W1};
        \node[vertex] (W2) at (0,0) {W2};
        
        % Nodos de puertos
        \node[vertex] (M) at (4,4) {M};
        \node[vertex] (G) at (4,1) {G};
        
        % Nodos de ciudades
        \node[vertex] (NY) at (8,5) {NY};
        \node[vertex] (LA) at (8,0) {LA};

        \node[left] at (W1.west) {150};
        \node[left] at (W2.west) {200};
        
        % Demandas de las ciudades
        \node[right] at (NY.east) {140};
        \node[right] at (LA.east) {160};
        
        % Arcos desde pozos a puertos
        \draw[edge] (W1) -- (M) node[pos=0.5, above] {10};
        \draw[edge] (W1) -- (G) node[pos=0.3, below] {12};
        \draw[edge] (W2) -- (M) node[pos=0.2, above] {5};
        \draw[edge] (W2) -- (G) node[pos=0.5, above] {12};
        
        % Arcos desde pozos a ciudades
        \draw[edge] (W1) -- (NY) node[pos=0.5, above] {45};
        \draw[edge] (W1) -- (LA) node[pos=0.3, above] {28};
        \draw[edge] (W2) -- (NY) node[pos=0.3, above] {26};
        \draw[edge] (W2) --(LA) node[pos=0.5, below] {25};
        
        % Arcos desde puertos a ciudades
        \draw[edge, color=blue] (M) -- (NY) node[pos=0.5, above] {16};
        \draw[edge, color=blue] (M) -- (LA) node[pos=0.5, right] {17};
        \draw[edge, color=blue] (M) -- (G) node[pos=0.2, right] {6};
        \draw[edge, color=blue] (G) -- (NY) node[pos=0.5, above] {14};
        \draw[edge, color=blue] (G) -- (LA) node[pos=0.5, below] {16};
    \end{tikzpicture}
    \end{center}
\end{frame}

\begin{frame}{Red de Transshipment}
    \begin{center}
    \begin{tikzpicture}[node distance=3cm]
        % Nodos de pozos
        \node[vertex] (W1) at (0,5) {W1};
        \node[vertex] (W2) at (0,0) {W2};
        
        % Nodos de puertos
        \node[vertex] (M) at (4,4) {M};
        \node[vertex] (G) at (4,1) {G};
        
        % Nodos de ciudades
        \node[vertex] (NY) at (8,5) {NY};
        \node[vertex] (LA) at (8,0) {LA};

        \node[left] at (W1.west) {150};
        \node[left] at (W2.west) {200};
        
        % Demandas de las ciudades
        \node[right] at (NY.east) {140};
        \node[right] at (LA.east) {160};
        
        % Arcos desde pozos a puertos
        \draw[edge] (W1) -- (M) node[pos=0.5, above] {10};
        \draw[edge] (W1) -- (G) node[pos=0.3, below] {12};
        \draw[edge] (W2) -- (M) node[pos=0.2, above] {5};
        \draw[edge] (W2) -- (G) node[pos=0.5, above] {12};
        
        % Arcos desde pozos a ciudades
        \draw[edge] (W1) -- (NY) node[pos=0.5, above] {45};
        \draw[edge] (W1) -- (LA) node[pos=0.3, above] {28};
        \draw[edge] (W2) -- (NY) node[pos=0.3, above] {26};
        \draw[edge] (W2) --(LA) node[pos=0.5, below] {25};
        
        % Arcos desde puertos a ciudades
        \draw[edge] (M) -- (NY) node[pos=0.5, above] {16};
        \draw[edge] (M) -- (LA) node[pos=0.5, right] {17};
        \draw[edge] (M) -- (G) node[pos=0.2, right] {6};
        \draw[edge] (G) -- (NY) node[pos=0.5, above] {14};
        \draw[edge] (G) -- (LA) node[pos=0.5, below] {16};
    
        \draw[edge, color=blue] (NY) -- (LA) node[pos=0.5, right] {15};
    \end{tikzpicture}
    \end{center}
\end{frame}

\begin{frame}{Formulación Transshipment}
    \textbf{Variables:} $x_{ij}$ = flujo desde nodo $i$ a nodo $j$
    
    \vspace{1em}
    \textbf{Restricciones de balance:}
    \begin{align*}
        \text{Entrada} - \text{Salida} = \text{Oferta neta}
    \end{align*}
    
    \textbf{Para cada nodo:}
    \begin{itemize}
        \item \textbf{Pozos:} Salida $\leq$ Capacidad
        \item \textbf{Ciudades:} Entrada = Demanda
        \item \textbf{Puertos:} Entrada = Salida (balance)
    \end{itemize}
\end{frame}

\begin{frame}{Problema de Flujo Máximo}
    \textbf{Objetivo:} Maximizar flujo desde fuente hasta sumidero
    
    \vspace{1em}
    \textbf{Características:}
    \begin{itemize}
        \item Un nodo fuente (source)
        \item Un nodo sumidero (sink)
        \item Capacidades en los arcos
        \item Conservación de flujo en nodos intermedios
    \end{itemize}
    
    \vspace{1em}
    \textbf{Aplicaciones:}
    \begin{itemize}
        \item[{\makebox[1em][c]{\textcolor{black}{\faWater}}}] Redes de agua/gas
        \item[{\makebox[1em][c]{\textcolor{black}{\faWifi}}}] Tráfico de red
        \item[{\makebox[1em][c]{\textcolor{black}{\faCar}}}] Flujo vehicular
    \end{itemize}
\end{frame}

\begin{frame}{Ejemplo: Red de Tuberías}
    \begin{center}
    \begin{tikzpicture}[node distance=2.5cm, scale=0.9]
        % Nodos
        \node[vertex, fill=red!30] (so) at (0,0) {so};
        \node[vertex, fill=gray!30] (1) at (2.5,0) {1};
        \node[vertex, fill=gray!30] (2) at (5,0) {2};
        \node[vertex, fill=gray!30] (3) at (3.75,2) {3};
        \node[vertex, fill=blue!30] (si) at (7.5,0) {si};

        % Arcos con capacidades
        \draw[edge] (so) -- (1) node[pos=0.5, above] {2};
        \draw[edge] (1) -- (2) node[pos=0.5, above] {3};
        \draw[edge] (2) -- (si) node[pos=0.5, above] {2};
        \draw[edge] (1) -- (3) node[pos=0.5, above] {4};
        \draw[edge] (3) -- (si) node[pos=0.5, above] {1};
        \draw[edge] (so) .. controls +(1.5,-1.5) and +(-1.5,-1.5) .. (2) node[pos=0.5, below] {3};
        
        % Etiquetas
        \node[below] at (so.south) {Fuente};
        \node[below] at (si.south) {Sumidero};
    \end{tikzpicture}
    \end{center}
\end{frame}

\begin{frame}{Formulación Flujo Máximo}
    \textbf{Variables:} $x_i$ = flujo por arco $i$
    
    \vspace{1em}
    \textbf{Función objetivo:}
    \begin{align*}
        \text{Maximizar: } \text{Flujo total de salida desde fuente}
    \end{align*}
    
    \textbf{Restricciones:}
    \begin{itemize}
        \item \textbf{Capacidad:} $x_i \leq \text{capacidad}_i$ para cada arco
        \item \textbf{Conservación:} Entrada = Salida en nodos intermedios
        \item \textbf{Balance:} Salida fuente = Entrada sumidero
    \end{itemize}
\end{frame}

\begin{frame}[fragile]{Implementación Flujo Máximo}
    \begin{lstlisting}[language=Python]
import picos

# Crear el problema
P = picos.Problem()

# Definir variables
x = picos.RealVariable('x', 6)

# Funcion objetivo
P.set_objective('max', x[0] + x[1])

# Restricciones de conservacion
P.add_constraint(x[0] + x[1] == x[4] + x[5])  # salida = llegada
P.add_constraint(x[0] + x[1] == x[2] + x[3])  # nodo 1
P.add_constraint(x[1] + x[3] == x[5])         # nodo 2
P.add_constraint(x[2] == x[4])                # nodo 3

# Restricciones de capacidad
P.add_constraint(x[0] <= 2)  # so-1
P.add_constraint(x[1] <= 3)  # so-2
P.add_constraint(x[2] <= 3)  # 1-2
P.add_constraint(x[3] <= 4)  # 1-3
P.add_constraint(x[4] <= 1)  # 3-si
P.add_constraint(x[5] <= 2)  # 2-si

P.solve(solver='glpk') # Solucion: 2 unidades
    \end{lstlisting}
\end{frame}

\begin{frame}{Comparación de Problemas de Redes}
    \begin{table}[H]
        \centering
        \begin{tabular}{l|l|l|l}
            \toprule
            \textbf{Problema} & \textbf{Objetivo} & \textbf{Restricciones} & \textbf{Aplicación} \\
            \midrule
            Transporte & Min costo & Oferta/Demanda & Logística \\
            Transshipment & Min costo & Balance + O/D & Distribución \\
            Flujo Máximo & Max flujo & Capacidad & Redes \\
            \bottomrule
        \end{tabular}
    \end{table}
    
    \vspace{1em}
    \textbf{Elementos comunes:}
    \begin{itemize}
        \item Variables de flujo
        \item Restricciones de balance/conservación
        \item Representación gráfica
        \item Algoritmos especializados
    \end{itemize}
\end{frame}

\begin{frame}{Próximos Pasos}
    \textbf{Para practicar:}
    \begin{itemize}
        \item Resolver problemas de transporte balanceado y desbalanceado
        \item Experimentar con diferentes configuraciones de red
        \item Analizar cuellos de botella en flujo máximo
        \item Implementar soluciones en PICOS
    \end{itemize}
\end{frame}

\begin{frame}{Terminamos}
    \begin{center}
        \Large{\textbf{¿Dudas?\\¿Consultas?}}
    \end{center}
    \begin{tikzpicture}[remember picture,overlay]
        \node[anchor=south,inner sep=30pt] at (current page.south) {
            \includegraphics[height=1cm]{../misc/UdeSA.png}
        };
    \end{tikzpicture}
\end{frame}

\end{document}
