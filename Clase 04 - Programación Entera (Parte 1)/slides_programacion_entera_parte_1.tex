% Slides de Programacion Entera - Parte 1
\documentclass{beamer}
\usetheme{metropolis}

\usepackage[spanish]{babel}
\usepackage[utf8]{inputenc}
\usepackage{graphicx}
\usepackage{tikz}
\usepackage{xcolor}
\usepackage{amsmath}
\usepackage{listings}
\usepackage{booktabs}
\usepackage{fontawesome5}

% Definicion de colores personalizados
\definecolor{primary}{RGB}{46, 204, 113}
\definecolor{secondary}{RGB}{52, 152, 219}
\definecolor{accent}{RGB}{231, 76, 60}
\definecolor{background}{RGB}{236, 240, 241}
\definecolor{lightgray}{rgb}{0.95,0.95,0.95}
\definecolor{darkgreen}{rgb}{0,0.5,0}
\definecolor{darkblue}{rgb}{0,0,0.5}

% Configuracion del tema
\setbeamercolor{normal text}{fg=black,bg=background}
\setbeamercolor{structure}{fg=primary}
\setbeamercolor{alerted text}{fg=accent}

\lstset{
  backgroundcolor=\color{lightgray},
  basicstyle=\tiny\ttfamily,
  keywordstyle=\color{darkblue}\bfseries,
  commentstyle=\color{darkgreen},
  stringstyle=\color{red},
  numbers=left,
  numberstyle=\tiny\color{gray},
  stepnumber=1,
  numbersep=5pt,
  showspaces=false,
  showstringspaces=false,
  showtabs=false,
  frame=single,
  tabsize=2,
  language=Python,
  breaklines=true,
  breakatwhitespace=true
}

\title{\Huge\textbf{Programación Entera - Parte 1}}
\author{Investigación Operativa}
\date{}

\begin{document}

% Portada
\begin{frame}
    \titlepage
    \begin{tikzpicture}[remember picture,overlay]
        \node[anchor=south west,inner sep=30pt] at (current page.south west) {
            \includegraphics[height=1cm]{../misc/UdeSA.png}
        };
    \end{tikzpicture}
\end{frame}

% Introduccion
\begin{frame}{¿Qué es la Programación Entera?}
    \textbf{Definición:}
    \begin{itemize}
        \item Buscamos la mejor solucion a problemas donde las variables deben ser enteras
        \item Hay dos familias: \textbf{pura} (todas enteras) y \textbf{mixta} (algunas enteras, otras continuas)
        \item No es convexa: requiere ramificación y acotación, cortes y heurísticas
    \end{itemize}
\end{frame}

\begin{frame}{¿Cuándo aparece la PE?}
    \begin{itemize}
        \item Decisiones si/no (abrir planta, invertir, etc.)
        \item Ítems indivisibles (personas, máquinas, micros)
        \item Restricciones lógicas (mutua exclusión, dependencia, al menos/a lo sumo k)
    \end{itemize}
\end{frame}

% Modelado con binarias
\begin{frame}{Modelado con variables binarias}
    \textbf{Las binarias $y \in \{0,1\}$ permiten:}
    \begin{itemize}
        \item Mutua exclusión: $y_1 + y_2 \leq 1$
        \item Dependencia: $y_B \leq y_A$
        \item Costos fijos y variables: $F y + c x$
        \item Activaciones: $x \leq M y$
    \end{itemize}
\end{frame}

% Ejemplo 1: Inversion
\begin{frame}{Ejemplo: Problema de inversión}
    Se analizan 6 inversiones, cada una con ganancia y capital requerido. Capital total: 100. Algunas son mutuamente excluyentes y otras dependen de otras.
    \begin{table}[H]
        \centering
        \begin{tabular}{c|cccccc}
            \toprule
            Inversión & 1 & 2 & 3 & 4 & 5 & 6 \\
            \midrule
            Ganancia & 15 & 12 & 16 & 18 & 9 & 11 \\
            Capital & 38 & 33 & 39 & 45 & 23 & 27 \\
            \bottomrule
        \end{tabular}
    \end{table}
\end{frame}

% Formulación: función objetivo
\begin{frame}{Modelo de inversión: Función objetivo}
    \textbf{Variables:} $y_j = 1$ si se selecciona la inversión $j$.
    \\[0.5em]
    \textbf{Queremos maximizar la ganancia total:}
    \begin{align*}
        \text{Max } Z = 15y_1 + 12y_2 + 16y_3 + 18y_4 + 9y_5 + 11y_6
    \end{align*}
\end{frame}

% Formulación: restricción de capital
\begin{frame}{Modelo de inversión: Restricción de capital}
    \textbf{No podemos invertir más de lo disponible:}
    \begin{align*}
        38y_1 + 33y_2 + 39y_3 + 45y_4 + 23y_5 + 27y_6 \leq 100
    \end{align*}
    (El capital total disponible es 100)
\end{frame}

% Formulación: exclusión
\begin{frame}{Modelo de inversión: Exclusión}
    \textbf{Algunas inversiones son mutuamente excluyentes:}
    \begin{align*}
        y_1 + y_2 \leq 1 \\
        y_3 + y_4 \leq 1
    \end{align*}
    (No se pueden hacer ambas a la vez)
\end{frame}

% Formulación: dependencia
\begin{frame}{Modelo de inversión: Dependencia}
    \textbf{Dependencia entre inversiones:}
    \begin{align*}
        y_3 \leq y_1 + y_2 \\
        y_4 \leq y_1 + y_2
    \end{align*}
    (Solo se puede hacer la 3 o la 4 si se hizo la 1 o la 2)
\end{frame}

% Formulación: binariedad
\begin{frame}{Modelo de inversión: Variables binarias}
    \textbf{Las variables son binarias:}
    \begin{align*}
        y_j \in \{0,1\} \quad \forall j
    \end{align*}
    (Cada inversión se toma o no)
\end{frame}

% Codigo PICOS - parte 1
\begin{frame}[fragile]{Modelo en PICOS: Definicion y restricciones}
    \begin{lstlisting}[language=Python]
import picos

P = picos.Problem()
y = picos.BinaryVariable('y', 6)

P.set_objective('max', 15*y[0] + 12*y[1] + 16*y[2] + 18*y[3] + 9*y[4] + 11*y[5])
P.add_constraint(38*y[0] + 33*y[1] + 39*y[2] + 45*y[3] + 23*y[4] + 27*y[5] <= 100)
P.add_constraint(y[0] + y[1] <= 1)
P.add_constraint(y[2] + y[3] <= 1)
P.add_constraint(y[2] <= y[0] + y[1])
P.add_constraint(y[3] <= y[0] + y[1])
    \end{lstlisting}
\end{frame}

% Codigo PICOS - parte 2
\begin{frame}[fragile]{Modelo en PICOS: Resolucion y resultados}
    \begin{lstlisting}[language=Python]
P.options.verbosity = 0
P.solve(solver='glpk')
print(y)          # valores optimos de las variables
print(P.value)    # ganancia total optima
    \end{lstlisting}
\end{frame}

% Salida de consola
\begin{frame}[fragile]{Salida}
    \begin{lstlisting}[language=bash,backgroundcolor=\color{black},basicstyle=\color{white}\ttfamily,numbers=none]
[ 1.00e+00]
[ 0.00e+00]
[ 1.00e+00]
[ 0.00e+00]
[ 1.00e+00]
[ 0.00e+00]
40.0
    \end{lstlisting}
\end{frame}

% Ejemplo 2: Plantas y productos
\begin{frame}{Ejemplo: Plantas y productos}
    Cuatro productos, tres plantas, cada una con capacidad y costos distintos. Hay dos variantes: permitir o no permitir división de producción.
    \begin{table}[H]
        \centering
        \begin{tabular}{c|cccc|c}
            \toprule
             & \multicolumn{4}{c|}{Costo unitario (\$/u)} & Capacidad \\
             & 1 & 2 & 3 & 4 & disponible (u/día) \\
            \midrule
            Planta 1 & 41 & 27 & 28 & 24 & 75 \\
            Planta 2 & 40 & 29 & -- & 23 & 75 \\
            Planta 3 & 37 & 30 & 27 & 21 & 45 \\
            \bottomrule
        \end{tabular}
    \end{table}
    Demanda: 20, 30, 30, 40
\end{frame}

% Formulación: función objetivo
\begin{frame}{Modelo: Función objetivo}
    \textbf{Variables:} $x_{ij}$ cantidad de producto $j$ fabricado en planta $i$.
    \\[0.5em]
    \textbf{Queremos minimizar el costo total de producción:}
    \begin{align*}
        \text{Min } Z = \sum_{i=1}^{3} \sum_{j=1}^{4} c_{ij} x_{ij}
    \end{align*}
\end{frame}

% Formulación: restricción de capacidad
\begin{frame}{Modelo: Restricción de capacidad}
    \textbf{Cada planta tiene una capacidad máxima:}
    \begin{align*}
        \sum_{j=1}^{4} x_{ij} \leq cap_i \quad \forall i
    \end{align*}
    (No se puede fabricar más de lo que permite la planta)
\end{frame}

% Formulación: restricción de demanda
\begin{frame}{Modelo: Restricción de demanda}
    \textbf{Se debe cubrir la demanda de cada producto:}
    \begin{align*}
        \sum_{i=1}^{3} x_{ij} = demanda_j \quad \forall j
    \end{align*}
    (La suma de lo producido en todas las plantas debe igualar la demanda de cada producto)
\end{frame}

% Formulación: restricción de división/no división
\begin{frame}{Modelo: División o no división}
    \textbf{¿Se permite fabricar un producto en más de una planta?}
    \begin{itemize}
        \item \textbf{División permitida:} No se agrega ninguna restricción extra (modelo de transporte).
        \item \textbf{Sin división:} Cada producto se fabrica en una sola planta (modelo de asignación binaria):
        \begin{align*}
            \sum_{i=1}^{3} y_{ij} = 1 \quad \forall j
        \end{align*}
        donde $y_{ij} = 1$ si el producto $j$ se fabrica en la planta $i$.
    \end{itemize}
\end{frame}

% Codigo PICOS transporte
\begin{frame}[fragile]{PICOS: modelo de transporte}
    \begin{lstlisting}[language=Python]
import picos
import numpy as np

costos = np.array([
    [41, 27, 28, 24],
    [40, 29, 1e6, 23],
    [37, 30, 27, 21]
])
capacidades = [75, 75, 45]
demanda = [20, 30, 30, 40]

plantas, productos = 3, 4
P = picos.Problem()
x = picos.RealVariable('x', (plantas, productos), lower=0)

P.set_objective('min', picos.sum([
    costos[i, j] * x[i, j]
    for i in range(plantas)
    for j in range(productos)
]))
for i in range(plantas):
    P.add_constraint(picos.sum(x[i, :]) <= capacidades[i])
for j in range(productos):
    P.add_constraint(picos.sum(x[:, j]) == demanda[j])
    \end{lstlisting}
\end{frame}

% Codigo PICOS transporte resultados
\begin{frame}[fragile]{PICOS: resultados transporte}
    \begin{lstlisting}[language=Python]
P.solve(solver='glpk')
print('Asignacion continua (x):')
print(np.round(x.value, 2))
print('Costo total:', round(P.value, 2))
    \end{lstlisting}
\end{frame}

% Codigo PICOS asignacion binaria
\begin{frame}[fragile]{PICOS: modelo de asignacion binaria}
    \begin{lstlisting}[language=Python]
costos_lista = costos.tolist()
demanda_lista = list(demanda)

P2 = picos.Problem()
y = picos.BinaryVariable('y', (plantas, productos))

P2.set_objective('min', picos.sum([
    costos_lista[i][j] * demanda_lista[j] * y[i, j]
    for i in range(plantas)
    for j in range(productos)
]))
for j in range(productos):
    P2.add_constraint(picos.sum([y[i, j] for i in range(plantas)]) == 1)
for i in range(plantas):
    P2.add_constraint(picos.sum([
        y[i, j] * demanda_lista[j]
        for j in range(productos)
    ]) <= capacidades[i])
    \end{lstlisting}
\end{frame}

% Codigo PICOS asignacion binaria resultados
\begin{frame}[fragile]{PICOS: resultados asignacion binaria}
    \begin{lstlisting}[language=Python]
P2.solve(solver='glpk')
print('Asignacion binaria (y):')
asignacion = np.array([[int(round(y[i, j].value)) for j in range(productos)] for i in range(plantas)])
print(asignacion)
print('Costo total:', round(P2.value, 2))
    \end{lstlisting}
\end{frame}

% Salida de consola
\begin{frame}[fragile]{Salida}
    \begin{lstlisting}[language=bash,backgroundcolor=\color{black},basicstyle=\color{white}\ttfamily,numbers=none]
--- Modelo de transporte (division permitida) ---
Asignacion continua (x):
[[ 0. 30. 30.  0.]
[ 0.  0.  0. 15.]
[20.  0.  0. 25.]]
Costo total: 3260.0
--- Modelo de asignacion binaria (sin division) ---
Asignacion binaria (y):
[[0 1 1 0]
[1 0 0 0]
[0 0 0 1]]
Costo total: 3290.0
    \end{lstlisting}
\end{frame}

% Ejemplo 3: Estaciones de bomberos
\begin{frame}{Ejemplo: Estaciones de bomberos}
    Ubicar 2 estaciones en 5 sectores. Cada estación atiende a su propio sector y a los que le sean asignados. Queremos minimizar el tiempo de respuesta ponderado por frecuencia de incendios.
    \begin{table}[H]
        \centering
        \begin{tabular}{c|ccccc}
            \toprule
            & 1 & 2 & 3 & 4 & 5 \\
            \midrule
            1 & 5 & 12 & 30 & 20 & 15 \\
            2 & 20 & 4 & 15 & 10 & 25 \\
            3 & 15 & 20 & 6 & 15 & 12 \\
            4 & 25 & 15 & 25 & 4 & 10 \\
            5 & 10 & 25 & 15 & 12 & 5 \\
            \bottomrule
        \end{tabular}
    \end{table}
    Frecuencia: 2, 1, 3, 1, 3
\end{frame}

% Formulación: función objetivo
\begin{frame}{Modelo de bomberos: Función objetivo}
    \textbf{Variables:}
    \begin{itemize}
        \item $y_i$: 1 si hay estación en el sector $i$
        \item $x_{ij}$: 1 si el sector $j$ es atendido por la estación ubicada en $i$
    \end{itemize}
    \textbf{Queremos minimizar el tiempo de respuesta ponderado:}
    \begin{align*}
        \text{Min } Z = \sum_{i=1}^{5} \sum_{j=1}^{5} t_{ij} x_{ij} f_j
    \end{align*}
    donde $t_{ij}$ es el tiempo de respuesta del sector $j$ atendido por la estación en $i$ y $f_j$ es la frecuencia de incendios en el sector $j$.
\end{frame}

% Formulación: restricción de asignación
\begin{frame}{Modelo de bomberos: Asignación}
    \textbf{Cada sector debe ser atendido por una sola estación:}
    \begin{align*}
        \sum_{i=1}^{5} x_{ij} = 1 \quad \forall j
    \end{align*}
    (Cada sector tiene que estar cubierto por alguna estación)
\end{frame}

% Formulación: restricción de cobertura
\begin{frame}{Modelo de bomberos: Cobertura}
    \textbf{Solo se puede asignar un sector a una estación si esa estación existe:}
    \begin{align*}
        x_{ij} \leq y_i \quad \forall i, j
    \end{align*}
    (No se puede asignar un sector a un lugar donde no hay estación)
\end{frame}

% Formulación: cantidad de estaciones
\begin{frame}{Modelo de bomberos: Cantidad de estaciones}
    \textbf{Queremos instalar exactamente dos estaciones:}
    \begin{align*}
        \sum_{i=1}^{5} y_i = 2
    \end{align*}
\end{frame}

% Formulación: resumen de variables
\begin{frame}{Modelo de bomberos: Variables}
    \textbf{Resumen:}
    \begin{itemize}
        \item $y_i \in \{0,1\}$: 1 si hay estación en el sector $i$
        \item $x_{ij} \in \{0,1\}$: 1 si el sector $j$ es atendido por la estación en $i$
    \end{itemize}
\end{frame}

% Codigo PICOS bomberos
\begin{frame}[fragile]{PICOS: modelo de bomberos}
    \begin{lstlisting}[language=Python]
import picos
import numpy as np

tiempos = np.array([
    [5, 12, 30, 20, 15],
    [20, 4, 15, 10, 25],
    [15, 20, 6, 15, 12],
    [25, 15, 25, 4, 10],
    [10, 25, 15, 12, 5]
])
frecuencias = np.array([2, 1, 3, 1, 3])
sectores = 5

P = picos.Problem()
y = picos.BinaryVariable('y', sectores)
x = picos.BinaryVariable('x', (sectores, sectores))

P.set_objective('min', picos.sum([
    tiempos[i, j] * x[i, j] * frecuencias[j]
    for i in range(sectores)
    for j in range(sectores)
]))
for j in range(sectores):
    P.add_constraint(picos.sum([x[i, j] for i in range(sectores)]) == 1)
for i in range(sectores):
    for j in range(sectores):
        P.add_constraint(x[i, j] <= y[i])
P.add_constraint(picos.sum(y) == 2)
    \end{lstlisting}
\end{frame}

% Codigo PICOS bomberos resultados
\begin{frame}[fragile]{PICOS: resultados bomberos}
    \begin{lstlisting}[language=Python]
P.solve(solver='glpk')
print('Asignacion (x):')
print(np.array([[int(round(x[i, j].value)) for j in range(sectores)] for i in range(sectores)]))
print('Estaciones ubicadas (y):')
print(np.array([int(round(y[i].value)) for i in range(sectores)]))
print('Tiempo total ponderado:', round(P.value, 2))
    \end{lstlisting}
\end{frame}

% Salida de consola
\begin{frame}[fragile]{Salida}
    \begin{lstlisting}[language=bash,backgroundcolor=\color{black},basicstyle=\color{white}\ttfamily,numbers=none]
Asignacion (x):
[[0 0 0 0 0]
 [0 0 0 0 0]
 [0 1 1 0 0]
 [0 0 0 0 0]
 [1 0 0 1 1]]
Estaciones ubicadas (y):
[0 0 1 0 1]
Tiempo total ponderado: 85.0
    \end{lstlisting}
\end{frame}

% Cierre
\begin{frame}{Terminamos}
    \begin{center}
        \Large{\textbf{¿Dudas?\\¿Consultas?}}
    \end{center}
    \begin{tikzpicture}[remember picture,overlay]
        \node[anchor=south,inner sep=30pt] at (current page.south) {
            \includegraphics[height=1cm]{../misc/UdeSA.png}
        };
    \end{tikzpicture}
\end{frame}

\end{document}
