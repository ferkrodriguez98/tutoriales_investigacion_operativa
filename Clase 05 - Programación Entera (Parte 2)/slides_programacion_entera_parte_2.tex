% Slides de Programacion Entera - Parte 2
\documentclass{beamer}
\usetheme{metropolis}

\usepackage[spanish]{babel}
\usepackage[utf8]{inputenc}
\usepackage{graphicx}
\usepackage{tikz}
\usepackage{xcolor}
\usepackage{amsmath}
\usepackage{listings}
\usepackage{booktabs}
\usepackage{fontawesome5}
\usepackage{tikz}
% ------------------------ Paquetes extra de TikZ para posicionamiento y flechas
\usetikzlibrary{positioning,calc,arrows}

% Definicion de colores personalizados
\definecolor{primary}{RGB}{46, 204, 113}
\definecolor{secondary}{RGB}{52, 152, 219}
\definecolor{accent}{RGB}{231, 76, 60}
\definecolor{background}{RGB}{236, 240, 241}
\definecolor{lightgray}{rgb}{0.95,0.95,0.95}
\definecolor{darkgreen}{rgb}{0,0.5,0}
\definecolor{darkblue}{rgb}{0,0,0.5}

% Configuracion del tema
\setbeamercolor{normal text}{fg=black,bg=background}
\setbeamercolor{structure}{fg=primary}
\setbeamercolor{alerted text}{fg=accent}

\lstset{
  backgroundcolor=\color{lightgray},
  basicstyle=\tiny\ttfamily,
  keywordstyle=\color{darkblue}\bfseries,
  commentstyle=\color{darkgreen},
  stringstyle=\color{red},
  numbers=left,
  numberstyle=\tiny\color{gray},
  stepnumber=1,
  numbersep=5pt,
  showspaces=false,
  showstringspaces=false,
  showtabs=false,
  frame=single,
  tabsize=2,
  language=Python,
  breaklines=true,
  breakatwhitespace=true
}

\title{\Huge\textbf{Programación Entera - Parte 2}}
\author{Investigación Operativa}
\date{}

\begin{document}

% Portada
\begin{frame}
    \titlepage
    \begin{tikzpicture}[remember picture,overlay]
        \node[anchor=south west,inner sep=30pt] at (current page.south west) {
            \includegraphics[height=1cm]{../misc/UdeSA.png}
        };
    \end{tikzpicture}
\end{frame}

% Introduccion
\begin{frame}{Programación Mixta}
    \textbf{En la clase anterior:}
    \begin{itemize}
        \item Vimos problemas con variables si/no
        \item Ahora aparecen problemas con variables binarias y continuas juntas
        \item Ejemplo típico: decidir si se opera una planta y cuanto produce
    \end{itemize}
\end{frame}

% Ejemplo 1: Planificacion de produccion
\begin{frame}{Ejemplo: Planificación de producción}
    Una empresa opera con dos turnos: diurno y nocturno. El costo de poner en marcha la planta es:
    \begin{itemize}
        \item Turno diurno: \$8.000
        \item Turno nocturno: \$4.500
    \end{itemize}
    \vspace{0.5em}
    Demanda para los próximos dos días:
    \begin{itemize}
        \item Dia 1: 2000 unidades (diurno)
        \item Noche 1: 3000 unidades (nocturno)
        \item Dia 2: 2000 unidades (diurno)
        \item Noche 2: 3000 unidades (nocturno)
    \end{itemize}
    Costo de almacenaje: \$1 por unidad por día. Se busca minimizar el costo total cumpliendo demanda.
\end{frame}

% Secuencia de armado del grafico de produccion

% Slide 1: Solo D1, I1, P1 y demanda
\begin{frame}{Visualización del problema: Turno D1}
    \begin{center}
    \scalebox{0.7}{
    \begin{tikzpicture}[auto, thick, node distance=1.5cm and 2.2cm]
        % Nodos de demanda
        \node[draw, circle, fill=purple!40, minimum size=1.2cm] (D1) {D1};
        % Nodo de inventario
        \node[draw, circle, fill=red!40, minimum size=1.2cm, right=of D1] (I1) {I1};
        % Nodo de produccion
        \node[draw, circle, fill=teal!60, minimum size=1.2cm, right=of I1] (P1) {P1};
        % Flecha de produccion a inventario
        \draw[-latex] (D1) -- node[above] {$y^d_1$} (I1);
        % Flecha de inventario a produccion
        \draw[-latex] (I1) -- (P1);
        % Demanda
        \node[right=0.7cm of P1, text=teal!80!black, font=\bfseries\large] (dem1) {2000};
    \end{tikzpicture}
    }
    \end{center}
\end{frame}

% Slide 2: Agrego N1, I2, P2 y demanda
\begin{frame}{Visualización del problema: Turno N1}
    \begin{center}
    \scalebox{0.7}{
    \begin{tikzpicture}[auto, thick, node distance=1.5cm and 2.2cm]
        % Nodos de demanda
        \node[draw, circle, fill=purple!40, minimum size=1.2cm] (D1) {D1};
        \node[draw, circle, fill=purple!40, minimum size=1.2cm, below=of D1] (N1) {N1};
        % Nodos de inventario
        \node[draw, circle, fill=red!40, minimum size=1.2cm, right=of D1] (I1) {I1};
        \node[draw, circle, fill=red!40, minimum size=1.2cm, below=of I1] (I2) {I2};
        % Nodos de produccion
        \node[draw, circle, fill=teal!60, minimum size=1.2cm, right=of I1] (P1) {P1};
        \node[draw, circle, fill=teal!60, minimum size=1.2cm, below=of P1] (P2) {P2};
        % Flechas
        \draw[-latex] (D1) -- node[above] {$y^d_1$} (I1);
        \draw[-latex] (I1) -- (P1);
        \draw[-latex] (N1) -- node[above] {$y^n_1$} (I2);
        \draw[-latex] (I2) -- (P2);
        % Almacenaje
        \draw[-latex] (I1) -- node[right] {$x_{I1 \to I2}$} (I2);
        % Demandas
        \node[right=0.7cm of P1, text=teal!80!black, font=\bfseries\large] (dem1) {2000};
        \node[right=0.7cm of P2, text=teal!80!black, font=\bfseries\large] (dem2) {3000};
    \end{tikzpicture}
    }
    \end{center}
\end{frame}

% Slide 3: Agrego D2, I3, P3 y demanda
\begin{frame}{Visualización del problema: Turno D2}
    \begin{center}
    \scalebox{0.7}{
    \begin{tikzpicture}[auto, thick, node distance=1.5cm and 2.2cm]
        % Nodos de demanda
        \node[draw, circle, fill=purple!40, minimum size=1.2cm] (D1) {D1};
        \node[draw, circle, fill=purple!40, minimum size=1.2cm, below=of D1] (N1) {N1};
        \node[draw, circle, fill=purple!40, minimum size=1.2cm, below=of N1] (D2) {D2};
        % Nodos de inventario
        \node[draw, circle, fill=red!40, minimum size=1.2cm, right=of D1] (I1) {I1};
        \node[draw, circle, fill=red!40, minimum size=1.2cm, below=of I1] (I2) {I2};
        \node[draw, circle, fill=red!40, minimum size=1.2cm, below=of I2] (I3) {I3};
        % Nodos de produccion
        \node[draw, circle, fill=teal!60, minimum size=1.2cm, right=of I1] (P1) {P1};
        \node[draw, circle, fill=teal!60, minimum size=1.2cm, below=of P1] (P2) {P2};
        \node[draw, circle, fill=teal!60, minimum size=1.2cm, below=of P2] (P3) {P3};
        % Flechas
        \draw[-latex] (D1) -- node[above] {$y^d_1$} (I1);
        \draw[-latex] (I1) -- (P1);
        \draw[-latex] (N1) -- node[above] {$y^n_1$} (I2);
        \draw[-latex] (I2) -- (P2);
        \draw[-latex] (D2) -- node[above] {$y^d_2$} (I3);
        \draw[-latex] (I3) -- (P3);
        % Almacenaje
        \draw[-latex] (I1) -- node[right] {$x_{I1 \to I2}$} (I2);
        \draw[-latex] (I2) -- node[right] {$x_{I2 \to I3}$} (I3);
        % Demandas
        \node[right=0.7cm of P1, text=teal!80!black, font=\bfseries\large] (dem1) {2000};
        \node[right=0.7cm of P2, text=teal!80!black, font=\bfseries\large] (dem2) {3000};
        \node[right=0.7cm of P3, text=teal!80!black, font=\bfseries\large] (dem3) {2000};
    \end{tikzpicture}
    }
    \end{center}
\end{frame}

% Slide 4: Agrego N2, I4, P4 y demanda (grafico completo)
\begin{frame}{Visualización del problema: Turno N2}
    \begin{center}
    \scalebox{0.7}{
    \begin{tikzpicture}[auto, thick, node distance=1.5cm and 2.2cm]
        % Nodos de demanda
        \node[draw, circle, fill=purple!40, minimum size=1.2cm] (D1) {D1};
        \node[draw, circle, fill=purple!40, minimum size=1.2cm, below=of D1] (N1) {N1};
        \node[draw, circle, fill=purple!40, minimum size=1.2cm, below=of N1] (D2) {D2};
        \node[draw, circle, fill=purple!40, minimum size=1.2cm, below=of D2] (N2) {N2};
        % Nodos de inventario
        \node[draw, circle, fill=red!40, minimum size=1.2cm, right=of D1] (I1) {I1};
        \node[draw, circle, fill=red!40, minimum size=1.2cm, below=of I1] (I2) {I2};
        \node[draw, circle, fill=red!40, minimum size=1.2cm, below=of I2] (I3) {I3};
        \node[draw, circle, fill=red!40, minimum size=1.2cm, below=of I3] (I4) {I4};
        % Nodos de produccion
        \node[draw, circle, fill=teal!60, minimum size=1.2cm, right=of I1] (P1) {P1};
        \node[draw, circle, fill=teal!60, minimum size=1.2cm, below=of P1] (P2) {P2};
        \node[draw, circle, fill=teal!60, minimum size=1.2cm, below=of P2] (P3) {P3};
        \node[draw, circle, fill=teal!60, minimum size=1.2cm, below=of P3] (P4) {P4};
        % Flechas
        \draw[-latex] (D1) -- node[above] {$y^d_1$} (I1);
        \draw[-latex] (I1) -- (P1);
        \draw[-latex] (N1) -- node[above] {$y^n_1$} (I2);
        \draw[-latex] (I2) -- (P2);
        \draw[-latex] (D2) -- node[above] {$y^d_2$} (I3);
        \draw[-latex] (I3) -- (P3);
        \draw[-latex] (N2) -- node[above] {$y^n_2$} (I4);
        \draw[-latex] (I4) -- (P4);
        % Almacenaje
        \draw[-latex] (I1) -- node[right] {$x_{I1 \to I2}$} (I2);
        \draw[-latex] (I2) -- node[right] {$x_{I2 \to I3}$} (I3);
        \draw[-latex] (I3) -- node[right] {$x_{I3 \to I4}$} (I4);
        % Demandas
        \node[right=0.7cm of P1, text=teal!80!black, font=\bfseries\large] (dem1) {2000};
        \node[right=0.7cm of P2, text=teal!80!black, font=\bfseries\large] (dem2) {3000};
        \node[right=0.7cm of P3, text=teal!80!black, font=\bfseries\large] (dem3) {2000};
        \node[right=0.7cm of P4, text=teal!80!black, font=\bfseries\large] (dem4) {3000};
    \end{tikzpicture}
    }
    \end{center}
\end{frame}

\begin{frame}{Variables y función objetivo}
    \textbf{Variables:}
    \begin{itemize}
        \item $y^d_i$: 1 si se produce en el turno diurno $i$, 0 si no
        \item $y^n_i$: 1 si se produce en el turno nocturno $i$, 0 si no
        \item $x_{Ii \to I(i+1)}$: cantidad almacenada entre periodos
    \end{itemize}
    \vspace{0.5em}
    \textbf{Función objetivo:}
    \begin{align*}
        \text{Min } Z = 8000(y^d_1 + y^d_2) + 4500(y^n_1 + y^n_2) + \sum_{i=1}^{3} x_{Ii \to I(i+1)}
    \end{align*}
\end{frame}

\begin{frame}{Restricciones}
    \begin{itemize}
        \item $y^d_i, y^n_i \in \{0,1\}$
        \item $x_{Ii \to I(i+1)} \geq 0$
        \item Conservación de flujo en cada periodo
    \end{itemize}
    \begin{align*}
        y^d_1 - x_{I1 \to I2} &= x_{I1 \to P_1} \\
        y^n_1 + x_{I1 \to I2} - x_{I2 \to I3} &= x_{I2 \to P_2} \\
        y^d_2 + x_{I2 \to I3} - x_{I3 \to I4} &= x_{I3 \to P_3} \\
        y^n_2 + x_{I3 \to I4} &= x_{I4 \to P_4}
    \end{align*}
\end{frame}

% Codigo PICOS - parte 1
\begin{frame}[fragile]{Modelo en PICOS: Definicion de variables y datos}
    \begin{lstlisting}[language=Python]
import picos
import numpy as np

# Datos
costo_diurno = 8000
costo_nocturno = 4500
demanda = [2000, 3000, 2000, 3000]  # D1, N1, D2, N2
costo_almacenaje = 1

# Variables
P = picos.Problem()

y_d = picos.BinaryVariable('y_d', 2)  # diurno (dia 1 y dia 2)
y_n = picos.BinaryVariable('y_n', 2)  # nocturno (noche 1 y noche 2)
x_d = picos.RealVariable('x_d', 2, lower=0)
x_n = picos.RealVariable('x_n', 2, lower=0)
s = picos.RealVariable('s', 3, lower=0)
    \end{lstlisting}
\end{frame}

% Codigo PICOS - parte 2
\begin{frame}[fragile]{Modelo en PICOS: Objetivo y restricciones}
    \begin{lstlisting}[language=Python]
# Objetivo: costos fijos de produccion + almacenaje
P.set_objective('min',
    costo_diurno * picos.sum(y_d) +
    costo_nocturno * picos.sum(y_n) +
    costo_almacenaje * picos.sum(s)
)

tipo_M = 10000
for i in range(2):
    P.add_constraint(x_d[i] <= tipo_M * y_d[i])
    P.add_constraint(x_n[i] <= tipo_M * y_n[i])

# Balance de inventario
P.add_constraint(x_d[0] - demanda[0] == s[0])
P.add_constraint(x_n[0] + s[0] - demanda[1] == s[1])
P.add_constraint(x_d[1] + s[1] - demanda[2] == s[2])
P.add_constraint(x_n[1] + s[2] - demanda[3] == 0)
    \end{lstlisting}
\end{frame}

% Codigo PICOS - parte 3
\begin{frame}[fragile]{Modelo en PICOS: Resolucion y resultados}
    \begin{lstlisting}[language=Python]
P.solve(solver='glpk')
print('Produccion diurna:', np.round(x_d.value, 2))
print('Produccion nocturna:', np.round(x_n.value, 2))
print('Turnos diurnos activados:', [int(round(y_d[i].value)) for i in range(2)])
print('Turnos nocturnos activados:', [int(round(y_n[i].value)) for i in range(2)])
print('Inventario final en cada periodo:', np.round(s.value, 2))
print('Costo total:', round(P.value, 2))
    \end{lstlisting}
\end{frame}

% Salida de consola
\begin{frame}[fragile]{Salida}
    \begin{lstlisting}[language=bash,backgroundcolor=\color{black},basicstyle=\color{white}\ttfamily,numbers=none]
Produccion diurna:
[[2000.]
[   0.]]
Produccion nocturna:
[[5000.]
[3000.]]
Turnos diurnos activados:
[1, 0]
Turnos nocturnos activados:
[1, 1]
Inventario final en cada periodo:
[[   0.]
[2000.]
[   0.]]
Costo total: 19000.0
    \end{lstlisting}
\end{frame}

% Ejemplo 2: Planificacion electrica
\begin{frame}{Ejemplo: Planificación eléctrica}
    Una empresa de energía debe satisfacer una demanda creciente durante 5 años. Hay 4 tipos de plantas con distintas capacidades y costos de construcción y operación. Se debe decidir en que año construir cada planta y cuando operarla para minimizar el costo total.
\end{frame}

\begin{frame}{Datos del problema}
    \begin{table}[H]
        \scalebox{0.7}{
        \centering
        \begin{tabular}{cccc}
            \toprule
            Planta & Capacidad (millones kWh) & Costo construccion (M\$) & Costo operativo anual (M\$) \\
            \midrule
            1 & 70 & 20 & 1.5 \\
            2 & 50 & 16 & 0.8 \\
            3 & 60 & 18 & 1.3 \\
            4 & 40 & 14 & 0.6 \\
            \bottomrule
        \end{tabular}
        }
    \end{table}
    Demanda: 80, 100, 120, 140, 160 millones kWh por año.
\end{frame}

\begin{frame}{Variables y modelo}
    \textbf{Variables:}
    \begin{itemize}
        \item $x_{ij}$: energía generada por planta $i$ en año $j$
        \item $\alpha_{ij}$: 1 si la planta $i$ esta operativa en año $j$, 0 si no
        \item $y_i$: 1 si la planta $i$ se construye
    \end{itemize}
    \vspace{0.5em}
    \textbf{Función objetivo:}
    \begin{align*}
        \text{Min } Z = \sum_{i=1}^{4} c_i y_i + \sum_{i=1}^{4}\sum_{j=1}^{5} o_i \alpha_{ij}
    \end{align*}
\end{frame}

\begin{frame}{Restricciones}
    \begin{enumerate}
        \item Demanda anual: $\sum_{i} x_{ij} \geq d_j$ para todo $j$
        \item Capacidad de planta: $x_{ij} \leq cap_i \alpha_{ij}$
        \item Operar solo si se construyo: $\alpha_{ij} \leq y_i$
        \item Permanencia: una vez operativa, no se puede apagar: $\alpha_{i,j+1} \geq \alpha_{ij}$
    \end{enumerate}
\end{frame}

% Codigo PICOS - parte 1
\begin{frame}[fragile]{Modelo en PICOS: Definicion de datos y variables}
    \begin{lstlisting}[language=Python]
import picos

demand = picos.Constant('demanda',  [80, 100, 120, 140, 160])
limits  = picos.Constant('cap',     [70, 50, 60, 40])
c       = picos.Constant('c',       [20, 16, 18, 14])
o       = picos.Constant('o',       [1.5, 0.8, 1.3, 0.6])

T, P = 5, 4  # anios, plantas
x = picos.RealVariable('x',  (P, T), lower=0)
alfa = picos.BinaryVariable('alfa', (P, T))
y = picos.BinaryVariable('y', P)

p = picos.Problem()
    \end{lstlisting}
\end{frame}

% Codigo PICOS - parte 2
\begin{frame}[fragile]{Modelo en PICOS: Restricciones}
    \begin{lstlisting}[language=Python]
# Demanda anual
for j in range(T):
    p.add_constraint(picos.sum(x[:, j]) >= demand[j])

# Capacidad de planta
for i in range(P):
    p.add_constraint(picos.sum(x[i, :]) <= 1e4*y[i])
    for j in range(T):
        p.add_constraint(x[i, j] <= 1e4*alfa[i, j])
        p.add_constraint(x[i, j] - limits[i] <= 1e4*(1-alfa[i, j]))

# Permanencia
for i in range(P):
    for j in range(T-1):
        p.add_constraint(alfa[i, j+1] >= alfa[i, j])

# Operacion implica construccion
for i in range(P):
    for j in range(T):
        p.add_constraint(alfa[i, j] <= y[i])
    \end{lstlisting}
\end{frame}

% Codigo PICOS - parte 3
\begin{frame}[fragile]{Modelo en PICOS: Objetivo y resolucion}
    \begin{lstlisting}[language=Python]
# Objetivo
p.set_objective('min', c[0]*y[0] + c[1]*y[1] + c[2]*y[2] + c[3]*y[3] +
                     o[0]*picos.sum(alfa[0, :]) + o[1]*picos.sum(alfa[1, :]) +
                     o[2]*picos.sum(alfa[2, :]) + o[3]*picos.sum(alfa[3, :]))

# Resolver
p.options.verbosity = 1
p.solve(solver='glpk')

print(x)           # energia generada
print(p.value)     # costo total minimo
    \end{lstlisting}
\end{frame}

% Salida de consola
\begin{frame}[fragile]{Salida esperada}
    \begin{lstlisting}[language=bash,backgroundcolor=\color{black},basicstyle=\color{white}\ttfamily,numbers=none]
[ 7.00e+01  5.00e+01  7.00e+01  7.00e+01  7.00e+01]
[ 1.00e+01  5.00e+01  5.00e+01  5.00e+01  5.00e+01]
[ 0.00e+00  0.00e+00  0.00e+00  0.00e+00  0.00e+00]
[ 0.00e+00  0.00e+00  0.00e+00  2.00e+01  4.00e+01]
62.7
    \end{lstlisting}
\end{frame}

% Cierre
\begin{frame}{Terminamos}
    \begin{center}
        \Large{\textbf{¿Dudas?\\¿Consultas?}}
    \end{center}
    \begin{tikzpicture}[remember picture,overlay]
        \node[anchor=south,inner sep=30pt] at (current page.south) {
            \includegraphics[height=1cm]{../misc/UdeSA.png}
        };
    \end{tikzpicture}
\end{frame}

\end{document}
