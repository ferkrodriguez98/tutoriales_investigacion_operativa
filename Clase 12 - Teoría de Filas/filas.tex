\documentclass[a4paper,11pt]{article}

\usepackage[spanish]{babel}
\usepackage[utf8]{inputenc}
\usepackage{amsmath}
\usepackage{amsfonts}
\usepackage{amssymb}
\usepackage{graphicx}
\usepackage{tikz}
\usepackage{enumitem}
\usepackage{listings}
\usepackage{xcolor}
\usepackage{booktabs}
\usepackage{multirow}

\lstset{
  language=Python,
  basicstyle=\small\ttfamily,
  keywordstyle=\color{red},
  stringstyle=\color{green!50!black},
  commentstyle=\color{gray}\itshape,
  showstringspaces=false,
  breaklines=true,
  frame=single,
  frameround=tttt,
  backgroundcolor=\color{black!5},
  extendedchars=true,
  inputencoding=utf8,
  literate={á}{{\'a}}1 {é}{{\'e}}1 {í}{{\'{\i}}}1 {ó}{{\'o}}1 {ú}{{\'u}}1 {ñ}{{\~n}}1
}

\usetikzlibrary{arrows.meta, positioning}

\setlength{\parindent}{0pt}

\begin{document}

\begin{center}
    {\LARGE \textbf{Teoría de Filas}}\\[0.5em]
    {Investigación Operativa, Universidad de San Andrés}
\end{center}

Si encuentran algún error en el documento o hay alguna duda, mandenmé un mail a rodriguezf@udesa.edu.ar y lo revisamos.

\section{Introducción a la Teoría de Filas}

La Teoría de Filas es una rama de la Investigación Operativa que estudia el comportamiento de sistemas en los que entidades (clientes) deben esperar para recibir un servicio. Estos sistemas se encuentran en múltiples contextos:

\begin{itemize}
    \item Bancos y supermercados: clientes esperando ser atendidos
    \item Sistemas de comunicaciones: paquetes de datos esperando ser transmitidos
    \item Hospitales: pacientes esperando atención médica
    \item Centros de llamadas: llamadas esperando ser atendidas
    \item Sistemas de manufactura: trabajos esperando ser procesados
\end{itemize}

\subsection{Componentes de un Sistema de Filas}

Un sistema de filas típico consta de los siguientes elementos:

\begin{enumerate}
    \item \textbf{Proceso de llegada}: Describe cómo los clientes llegan al sistema (tasa $\lambda$)
    \item \textbf{Mecanismo de servicio}: Describe cómo se atienden los clientes (tasa $\mu$)
    \item \textbf{Disciplina de la cola}: El orden en que se atienden los clientes (FIFO, LIFO, prioridades, etc.)
    \item \textbf{Capacidad del sistema}: Número máximo de clientes que puede contener
    \item \textbf{Número de servidores}: Cantidad de recursos disponibles para atender
\end{enumerate}

\subsection{Objetivos del Análisis}

El análisis de sistemas de filas busca responder preguntas como:

\begin{itemize}
    \item ¿Cuánto tiempo esperará un cliente en promedio?
    \item ¿Cuántos clientes habrá en el sistema en un momento dado?
    \item ¿Cuál es la probabilidad de que el sistema esté vacío u ocupado?
    \item ¿Cuántos servidores se necesitan para mantener un nivel de servicio aceptable?
    \item ¿Cómo optimizar el balance entre costos de servicio y costos de espera?
\end{itemize}

\section{Distribuciones}

En teoría de filas, las distribuciones de probabilidad más comunes para modelar llegadas y servicios son:

\subsection{Distribución de Poisson}

La distribución de Poisson se utiliza para modelar el \textbf{número de llegadas} en un intervalo de tiempo fijo. Si las llegadas ocurren con una tasa $\lambda$ (llegadas por unidad de tiempo), la probabilidad de que ocurran exactamente $n$ llegadas en un intervalo de tiempo es:

\[
P(N = n) = \frac{e^{-\lambda} \lambda^n}{n!}, \quad n = 0, 1, 2, \ldots
\]

\textbf{Propiedades:}
\begin{itemize}
    \item Media: $E[N] = \lambda$
    \item Varianza: $\text{Var}(N) = \lambda$
    \item Las llegadas son independientes entre sí
    \item El proceso es sin memoria (propiedad markoviana)
\end{itemize}

\subsection{Distribución Exponencial}

La distribución exponencial se utiliza para modelar los \textbf{tiempos entre llegadas} y los \textbf{tiempos de servicio}. Si el tiempo entre eventos sigue una distribución exponencial con parámetro $\lambda$, su función de densidad es:

\[
f(t) = \lambda e^{-\lambda t}, \quad t \geq 0
\]

Y su función de distribución acumulada:

\[
F(t) = P(T \leq t) = 1 - e^{-\lambda t}
\]

\textbf{Propiedades:}
\begin{itemize}
    \item Media: $E[T] = \frac{1}{\lambda}$
    \item Varianza: $\text{Var}(T) = \frac{1}{\lambda^2}$
    \item \textbf{Propiedad de falta de memoria}: $P(T > s + t \mid T > s) = P(T > t)$
    \item El mínimo de variables exponenciales independientes es exponencial
\end{itemize}

\subsection{Relación entre Poisson y Exponencial}

Existe una relación fundamental entre ambas distribuciones:

\begin{itemize}
    \item Si el número de llegadas sigue una distribución de Poisson con tasa $\lambda$, entonces los tiempos entre llegadas siguen una distribución exponencial con parámetro $\lambda$
    \item Recíprocamente, si los tiempos entre llegadas son exponenciales con parámetro $\lambda$, el número de llegadas sigue una distribución de Poisson con tasa $\lambda$
\end{itemize}

\subsection{Distribución General}

En algunos modelos (como M/G/1), el tiempo de servicio puede seguir una distribución general (no necesariamente exponencial). En estos casos se caracteriza por:

\begin{itemize}
    \item Media: $E[S] = \frac{1}{\mu}$
    \item Varianza: $\text{Var}(S) = \sigma^2$
    \item Segundo momento: $E[S^2]$
\end{itemize}

\section{Nacimiento y Muerte}

Los \textbf{procesos de nacimiento y muerte} son una clase especial de cadenas de Markov en tiempo continuo que modelan sistemas donde la población puede aumentar (nacimientos) o disminuir (muertes) de a una unidad a la vez.

\subsection{Definición}

Un proceso de nacimiento y muerte se caracteriza por:

\begin{itemize}
    \item \textbf{Estados}: $n = 0, 1, 2, \ldots$ (número de clientes en el sistema)
    \item \textbf{Tasas de nacimiento} $\lambda_n$: tasa a la que el sistema pasa del estado $n$ al estado $n+1$
    \item \textbf{Tasas de muerte} $\mu_n$: tasa a la que el sistema pasa del estado $n$ al estado $n-1$
\end{itemize}

\subsection{Diagrama de Transiciones}

\begin{center}
\shorthandoff{>}
\begin{tikzpicture}[node distance=2.5cm, auto]
    \node (s0) {$0$};
    \node (s1) [right of=s0] {$1$};
    \node (s2) [right of=s1] {$2$};
    \node (s3) [right of=s2] {$3$};
    \node (sn) [right of=s3] {$\cdots$};

    % Nacimientos (hacia la derecha)
    \draw[->, bend left=30] (s0) to node[above] {$\lambda_0$} (s1);
    \draw[->, bend left=30] (s1) to node[above] {$\lambda_1$} (s2);
    \draw[->, bend left=30] (s2) to node[above] {$\lambda_2$} (s3);
    \draw[->, bend left=30] (s3) to node[above] {$\lambda_3$} (sn);

    % Muertes (hacia la izquierda)
    \draw[->, bend left=30] (s1) to node[below] {$\mu_1$} (s0);
    \draw[->, bend left=30] (s2) to node[below] {$\mu_2$} (s1);
    \draw[->, bend left=30] (s3) to node[below] {$\mu_3$} (s2);
\end{tikzpicture}
\shorthandon{>}
\end{center}

\subsection{Ecuaciones de Balance}

En el estado estacionario, la tasa de entrada a cada estado debe igualar la tasa de salida. Esto da lugar a las \textbf{ecuaciones de balance}:

Para el estado 0:
\[
\lambda_0 \pi_0 = \mu_1 \pi_1
\]

Para el estado $n \geq 1$:
\[
(\lambda_n + \mu_n) \pi_n = \lambda_{n-1} \pi_{n-1} + \mu_{n+1} \pi_{n+1}
\]

\subsection{Solución para Distribuciones de Estado Estacionario}

Resolviendo las ecuaciones de balance, obtenemos:

\[
\pi_n = \pi_0 \prod_{i=0}^{n-1} \frac{\lambda_i}{\mu_{i+1}}, \quad n = 1, 2, 3, \ldots
\]

donde $\pi_0$ se obtiene de la condición de normalización:

\[
\sum_{n=0}^{\infty} \pi_n = 1
\]

\subsection{Aplicación a Teoría de Filas}

En teoría de filas, los procesos de nacimiento y muerte modelan:

\begin{itemize}
    \item \textbf{Nacimientos}: Llegadas de clientes (tasa $\lambda_n$)
    \item \textbf{Muertes}: Salidas de clientes después del servicio (tasa $\mu_n$)
\end{itemize}

Para sistemas M/M/1 y M/M/1/c:
\begin{itemize}
    \item $\lambda_n = \lambda$ (tasa de llegada constante)
    \item $\mu_n = \mu$ (tasa de servicio constante)
\end{itemize}

\section{Notación de Kendall}

La notación de Kendall es una notación estándar para clasificar sistemas de filas. Se expresa como:

\[
\boxed{A / B / c / K / N / D}
\]

donde cada símbolo tiene el siguiente significado:

\begin{itemize}
    \item \textbf{A}: Distribución de los tiempos entre llegadas
    \item \textbf{B (o a veces S)}: Distribución de los tiempos de servicio
    \item \textbf{c}: Número de servidores en paralelo
    \item \textbf{K}: Capacidad máxima del sistema (opcional, por defecto $\infty$)
    \item \textbf{N}: Tamaño de la población de clientes (opcional, por defecto $\infty$)
    \item \textbf{D}: Disciplina de la cola (opcional, por defecto FIFO)
\end{itemize}

\subsection{Símbolos Comunes para Distribuciones}

\begin{itemize}
    \item \textbf{M} (Markoviana): Distribución exponencial (o Poisson para llegadas)
    \item \textbf{D} (Determinística): Tiempos constantes
    \item \textbf{G} (General): Distribución arbitraria
    \item \textbf{$E_k$} (Erlang): Distribución Erlang con $k$ fases
\end{itemize}

\subsection{Ejemplos Comunes}

\begin{itemize}
    \item \textbf{M/M/1}: Llegadas Poisson, servicio exponencial, 1 servidor, capacidad infinita
    \item \textbf{M/M/1/c}: Llegadas Poisson, servicio exponencial, 1 servidor, capacidad máxima $c$
    \item \textbf{M/M/s}: Llegadas Poisson, servicio exponencial, $s$ servidores en paralelo
    \item \textbf{M/G/1}: Llegadas Poisson, servicio con distribución general, 1 servidor
    \item \textbf{M/D/1}: Llegadas Poisson, servicio determinístico, 1 servidor
\end{itemize}

\subsection{Disciplinas de Cola}

\begin{itemize}
    \item \textbf{FIFO} (First In First Out): El primero en llegar es el primero en ser atendido
    \item \textbf{LIFO} (Last In First Out): El último en llegar es el primero en ser atendido
    \item \textbf{SIRO} (Service In Random Order): Se atiende en orden aleatorio
    \item \textbf{Priority}: Se atiende según prioridades asignadas
\end{itemize}

\section{Parámetros}

\begin{itemize}
    \item $\lambda$: Tasa de llegadas.
    \item $\mu$: Tasa de servicio.
    \item $L$: Número de clientes en el sistema.
    \item $L_q$: Número de clientes en la cola.
    \item $L_s$: Número de clientes en servicio.
    \item $W$: Valor medio esperado del tiempo de espera en el sistema.
    \item $W_q$: Valor medio esperado del tiempo de espera en la cola.
    \item $W_s$: Valor medio esperado del tiempo de servicio.
\end{itemize}

\section{Fórmulas}

\subsection{$\rho$}

Tanto para modelos M/M/1 como M/M/1/c, la fórmula para $\rho$ es la misma:

\[
\rho = \frac{\lambda}{\mu}
\]

\subsection{$\pi_0$}

Para modelos M/M/1:

\[
\pi_0 = 1 - \rho
\]

Para modelos M/M/1/c:

\[
\pi_0 = \frac{1-\rho}{1-\rho^c+1}
\]

\subsection{$L$}

Para modelos M/M/1:

\[
L = \frac{\rho}{1-\rho}
\]

Para modelos M/M/1/c:

\[
L = \frac{\rho \left[1 - (c+1)\rho^c + c\rho^{c+1}\right]}{(1 - \rho^{c+1})(1-\rho)}
\]

\subsection{$L_q$}

Para modelos M/M/1:

\[
L_q = \frac{\rho^2}{1-\rho}
\]

Para modelos M/M/1/c:

\[
L_q = L - (1-\pi_0)
\]

\subsection{$L_s$}

Para modelos M/M/1:

\[
L_s = \rho
\]

Para modelos M/M/1/c:

\[
L_s = 1 - \pi_0
\]

\subsection{$W$}

Para modelos M/M/1:

\[
W = \frac{1}{\mu - \lambda}
\]

Para modelos M/M/1/c:

\[
W = \frac{L}{\lambda(1 - \pi_c)}
\]

\subsection{$W_q$}

Para modelos M/M/1:

\[
W_q = \frac{\lambda}{\mu(\mu-\lambda)}
\]

Para modelos M/M/1/c:

\[
W_q = \frac{L_q}{\lambda(1 - \pi_c)}
\]

\subsection{$W_s$}

Tanto para modelos M/M/1 como M/M/1/c:

\[
W_s = \frac{1}{\mu}
\]

\section{Ejercicios}

\subsection{Ejercicio 1}

Suponga que en una estación con un solo servidor llegan en promedio 45 clientes por hora, Se tiene capacidad para atender en promedio a 60 clientes por hora. Se sabe que los clientes esperan en promedio 3 minutos en la cola. Se solicita: 

\begin{itemize}
    \item[a)] Tiempo promedio que un cliente pasa en el sistema.
    \item[b)] Número promedio de clientes en la cola.
    \item[c)] Número promedio de clientes en el Sistema en un momento dado.
\end{itemize}

\subsubsection{Solución}

Datos:
\begin{itemize}
    \item $\lambda = 45$ clientes/hora
    \item $\mu = 60$ clientes/hora
    \item $W_q = 3$ minutos $= 0.05$ horas
\end{itemize}

Este es un modelo M/M/1.

\vspace{0.5em}

\textbf{\underline{a) Tiempo promedio en el sistema ($W$):}}

\vspace{0.5em}

Sabemos que $W = W_q + W_s$, donde $W_s = \frac{1}{\mu}$:

\[
W_s = \frac{1}{60} \text{ horas} = 1 \text{ minuto}
\]

Por lo tanto:
\[
W = W_q + W_s = 3 + 1 = 4 \text{ minutos} = \frac{1}{15} \text{ horas}
\]

Alternativamente, usando la fórmula directa:
\[
W = \frac{1}{\mu - \lambda} = \frac{1}{60-45} = \frac{1}{15} \text{ horas} = 4 \text{ minutos}
\]

\textbf{\underline{b) Número promedio de clientes en la cola ($L_q$):}}

\vspace{0.5em}

Usando la Ley de Little: $L_q = \lambda W_q$

\[
L_q = 45 \times 0.05 = 2.25 \text{ clientes}
\]

Alternativamente, usando la fórmula directa:
\[
\rho = \frac{\lambda}{\mu} = \frac{45}{60} = 0.75
\]

\[
L_q = \frac{\rho^2}{1-\rho} = \frac{(0.75)^2}{1-0.75} = \frac{0.5625}{0.25} = 2.25 \text{ clientes}
\]

\textbf{\underline{c) Número promedio de clientes en el sistema ($L$):}}

\vspace{0.5em}

Usando la Ley de Little: $L = \lambda W$

\[
L = 45 \times \frac{1}{15} = 3 \text{ clientes}
\]

Alternativamente, usando la fórmula directa:
\[
L = \frac{\rho}{1-\rho} = \frac{0.75}{1-0.75} = \frac{0.75}{0.25} = 3 \text{ clientes}
\]

\subsection{Ejercicio 2}

Suponga un restaurante de comidas rápidas al cual llegan en promedio 100 clientes por hora. Se tiene capacidad para atender en promedio a 150 clientes por hora. Calcule las medidas de desempeño del sistema

\begin{itemize}
    \item[a)] ¿Cuál es la probabilidad que el sistema esté ocioso?
    \item[b)] ¿Cuál es la probabilidad que un cliente llegue y tenga que esperar, porque el sistema está ocupado?
    \item[c)] ¿Cuál es el número promedio de clientes en la cola?
    \item[d)] ¿Cuál es la probabilidad que hayan 10 clientes en la cola?
\end{itemize}

\subsubsection{Solución}

Datos:
\begin{itemize}
    \item $\lambda = 100$ clientes/hora
    \item $\mu = 150$ clientes/hora
\end{itemize}

Este es un modelo M/M/1.

Primero calculamos $\rho$:
\[
\rho = \frac{\lambda}{\mu} = \frac{100}{150} = \frac{2}{3} \approx 0.667
\]

\textbf{\underline{a) Probabilidad de que el sistema esté ocioso ($\pi_0$):}}

\[
\pi_0 = 1 - \rho = 1 - \frac{2}{3} = \frac{1}{3} \approx 0.333
\]

La probabilidad de que el sistema esté ocioso es del 33.3\%.

\vspace{0.5em}

\textbf{\underline{b) Probabilidad de que un cliente tenga que esperar:}}

\vspace{0.5em}

Un cliente tiene que esperar cuando el sistema está ocupado, es decir, cuando hay al menos 1 cliente en el sistema. Esta probabilidad es:

\[
P(\text{esperar}) = 1 - \pi_0 = \rho = \frac{2}{3} \approx 0.667
\]

La probabilidad de que un cliente tenga que esperar es del 66.7\%.

\vspace{0.5em}

\textbf{\underline{c) Número promedio de clientes en la cola ($L_q$):}}

\vspace{0.5em}

Usando la fórmula para modelo M/M/1:

\[
L_q = \frac{\rho^2}{1-\rho} = \frac{(2/3)^2}{1-2/3} = \frac{4/9}{1/3} = \frac{4}{3} \approx 1.33 \text{ clientes}
\]

Alternativamente, primero calculamos $W_q$:

\[
W_q = \frac{\lambda}{\mu(\mu-\lambda)} = \frac{100}{150(150-100)} = \frac{100}{150 \times 50} = \frac{100}{7500} = \frac{1}{75} \text{ horas} = 0.8 \text{ minutos}
\]

Y luego aplicamos la Ley de Little:

\[
L_q = \lambda W_q = 100 \times \frac{1}{75} = \frac{100}{75} = \frac{4}{3} \approx 1.33 \text{ clientes}
\]

\textbf{\underline{d) Probabilidad de que haya 10 clientes en la cola:}}

\vspace{0.5em}

En un sistema M/M/1, la probabilidad de que haya exactamente $n$ clientes en el sistema es:

\[
\pi_n = (1-\rho)\rho^n
\]

Si hay 10 clientes en la cola, hay 11 clientes en el sistema (10 esperando + 1 siendo atendido):

\[
\pi_{11} = (1-\rho)\rho^{11} = \frac{1}{3} \times \left(\frac{2}{3}\right)^{11} \approx 0.333 \times 0.00568 \approx 0.00189
\]

La probabilidad de que haya exactamente 10 clientes en la cola es aproximadamente 0.189\%.

\subsection{Ejercicio 3}

Una cabina de peaje tiene una ventanilla y puede atender como máximo a 4 autos. Los autos llegan a razón de 20 por hora, y el tiempo medio de servicio es de 2 minutos por auto. Se desea calcular:

\begin{itemize}
    \item[a)] La probabilidad de que el sistema esté vacío.
    \item[b)] La probabilidad de que haya 4 autos (sistema lleno).
    \item[c)] El número promedio de autos en el sistema.
    \item[d)] El número promedio de autos en la cola.
    \item[e)] El tiempo promedio total y en cola que pasa un auto en el sistema.
\end{itemize}

\subsubsection{Solución}

Datos:
\begin{itemize}
    \item $\lambda = 20$ autos/hora
    \item Tiempo medio de servicio = 2 minutos $\Rightarrow \mu = 30$ autos/hora
    \item Capacidad máxima del sistema: $c = 4$ autos
\end{itemize}

Este es un modelo M/M/1/4 (llegadas Poisson, servicio exponencial, 1 servidor, capacidad máxima 4).

Primero calculamos $\rho$:
\[
\rho = \frac{\lambda}{\mu} = \frac{20}{30} = \frac{2}{3}
\]

\textbf{\underline{a) Probabilidad de que el sistema esté vacío ($\pi_0$):}}

\vspace{0.5em}

Para un modelo M/M/1/c:
\[
\pi_0 = \frac{1-\rho}{1-\rho^{c+1}} = \frac{1-\frac{2}{3}}{1-(\frac{2}{3})^5} = \frac{\frac{1}{3}}{1-\frac{32}{243}} = \frac{\frac{1}{3}}{\frac{211}{243}} = \frac{243}{3 \times 211} = \frac{81}{211} \approx 0.384
\]

La probabilidad de que el sistema esté vacío es aproximadamente 38.4\%.

\vspace{0.5em}

\textbf{\underline{b) Probabilidad de que el sistema esté lleno ($\pi_4$):}}

\vspace{0.5em}

Para un modelo M/M/1/c, la probabilidad de estado $n$ es:
\[
\pi_n = \pi_0 \rho^n
\]

Por lo tanto:
\[
\pi_4 = \pi_0 \rho^4 = \frac{81}{211} \times \left(\frac{2}{3}\right)^4 = \frac{81}{211} \times \frac{16}{81} = \frac{16}{211} \approx 0.076
\]

La probabilidad de que haya 4 autos (sistema lleno) es aproximadamente 7.6\%.

\vspace{0.5em}

\textbf{\underline{c) Número promedio de autos en el sistema ($L$):}}

\vspace{0.5em}

Para un modelo M/M/1/c:
\[
L = \frac{\rho \left[1 - (c+1)\rho^c + c\rho^{c+1}\right]}{(1 - \rho^{c+1})(1-\rho)}
\]

Sustituyendo con $\rho = \frac{2}{3}$ y $c = 4$:

\[
L = \frac{\frac{2}{3} \left[1 - 5\left(\frac{2}{3}\right)^4 + 4\left(\frac{2}{3}\right)^5\right]}{\left(1 - \left(\frac{2}{3}\right)^5\right)\left(1-\frac{2}{3}\right)}
\]

\[
L = \frac{\frac{2}{3} \left[1 - 5 \times \frac{16}{81} + 4 \times \frac{32}{243}\right]}{\frac{211}{243} \times \frac{1}{3}} = \frac{\frac{2}{3} \left[1 - \frac{80}{81} + \frac{128}{243}\right]}{\frac{211}{729}}
\]

\[
L = \frac{\frac{2}{3} \left[\frac{243 - 240 + 128}{243}\right]}{\frac{211}{729}} = \frac{\frac{2}{3} \times \frac{131}{243}}{\frac{211}{729}} = \frac{\frac{262}{729}}{\frac{211}{729}} = \frac{262}{211} \approx 1.242
\]

El número promedio de autos en el sistema es aproximadamente 1.24 autos.

\vspace{0.5em}

\textbf{\underline{d) Número promedio de autos en la cola ($L_q$):}}

\vspace{0.5em}

Para un modelo M/M/1/c:
\[
L_q = L - (1 - \pi_0) = L - L_s
\]

donde $L_s = 1 - \pi_0 = 1 - \frac{81}{211} = \frac{130}{211}$

\[
L_q = \frac{262}{211} - \frac{130}{211} = \frac{132}{211} \approx 0.626
\]

El número promedio de autos en la cola es aproximadamente 0.63 autos.

\vspace{0.5em}

\textbf{\underline{e) Tiempo promedio total y en cola:}}

\vspace{0.5em}

La tasa efectiva de llegadas es:
\[
\lambda_{ef} = \lambda(1 - \pi_c) = 20 \times \left(1 - \frac{16}{211}\right) = 20 \times \frac{195}{211} \approx 18.48 \text{ autos/hora}
\]

Tiempo promedio en el sistema:
\[
W = \frac{L}{\lambda_{ef}} = \frac{\frac{262}{211}}{20 \times \frac{195}{211}} = \frac{262}{20 \times 195} = \frac{262}{3900} \approx 0.0672 \text{ horas} \approx 4.03 \text{ minutos}
\]

Tiempo promedio en la cola:
\[
W_q = \frac{L_q}{\lambda_{ef}} = \frac{\frac{132}{211}}{20 \times \frac{195}{211}} = \frac{132}{3900} \approx 0.0338 \text{ horas} \approx 2.03 \text{ minutos}
\]

\section{M/G/1}

El modelo M/G/1 es un sistema de filas donde:
\begin{itemize}
    \item \textbf{M}: Las llegadas siguen un proceso de Poisson (distribución Markoviana)
    \item \textbf{G}: El tiempo de servicio sigue una distribución general (no necesariamente exponencial)
    \item \textbf{1}: Hay un solo servidor
\end{itemize}

Este modelo generaliza el M/M/1 al permitir cualquier distribución de tiempo de servicio.

\subsection{Parámetros del Sistema}

\begin{itemize}
    \item $\lambda$: Tasa de llegadas (llegadas por unidad de tiempo)
    \item $E[S]$: Tiempo medio de servicio
    \item $\mu = \frac{1}{E[S]}$: Tasa media de servicio
    \item $\text{Var}(S)$: Varianza del tiempo de servicio
    \item $E[S^2]$: Segundo momento del tiempo de servicio
    \item $\rho = \frac{\lambda}{\mu} = \lambda E[S]$: Factor de utilización
\end{itemize}

Para estabilidad del sistema, se requiere $\rho < 1$.

\subsection{Fórmula de Pollaczek-Khinchin}

La fórmula de Pollaczek-Khinchin proporciona el número promedio de clientes en la cola:

\[
L_q = \frac{\lambda^2 E[S^2]}{2(1-\rho)} = \frac{\rho^2 + \lambda^2 \text{Var}(S)}{2(1-\rho)}
\]

Esta fórmula también puede expresarse en términos del coeficiente de variación del tiempo de servicio $C_s^2 = \frac{\text{Var}(S)}{E[S]^2}$:

\[
L_q = \frac{\rho^2(1 + C_s^2)}{2(1-\rho)}
\]

\subsection{Otras Medidas de Desempeño}

\textbf{Tiempo promedio en la cola:}
\[
W_q = \frac{L_q}{\lambda} = \frac{\lambda E[S^2]}{2(1-\rho)}
\]

\textbf{Número promedio de clientes en el sistema:}
\[
L = L_q + \rho
\]

\textbf{Tiempo promedio en el sistema:}
\[
W = W_q + E[S] = \frac{L}{\lambda}
\]

\subsection{Casos Especiales}

\subsubsection{M/M/1 (Servicio Exponencial)}

Si el servicio es exponencial: $E[S^2] = \frac{2}{\mu^2}$ y $C_s^2 = 1$

\[
L_q = \frac{\rho^2}{1-\rho}
\]

que coincide con la fórmula del modelo M/M/1.

\subsubsection{M/D/1 (Servicio Determinístico)}

Si el servicio es determinístico (constante): $\text{Var}(S) = 0$, $E[S^2] = E[S]^2$ y $C_s^2 = 0$

\[
L_q = \frac{\rho^2}{2(1-\rho)}
\]

Nótese que para el mismo $\rho$, el sistema M/D/1 tiene la mitad de clientes en cola que M/M/1, debido a la ausencia de variabilidad en el servicio.

\subsection{Interpretación}

La fórmula de Pollaczek-Khinchin muestra que:
\begin{itemize}
    \item El número de clientes en cola aumenta con la variabilidad del tiempo de servicio
    \item Reducir la variabilidad del servicio (manteniendo la media constante) reduce la congestión
    \item Cuando $C_s^2 = 0$ (servicio determinístico), se minimiza $L_q$
    \item Cuando $C_s^2 = 1$ (servicio exponencial), se recupera el modelo M/M/1
\end{itemize}

\section{Simulaciones}

\subsection{Ejercicio 1}

Un banco recibe en promedio $(\lambda = 4)$ clientes por hora (llegadas Poisson) y atiende cada cajero a razón de $(\mu = 2)$ clientes por hora (servicio exponencial). El banco puede contratar \(s\) cajeros paralelos.

\begin{itemize}
    \item El costo de espera es de \$10 por cliente-hora en cola.
    \item El costo de servicio es de \$15 por cajero-hora.
\end{itemize}

Se busca determinar el número óptimo de cajeros \(s\) que minimice el costo total:

\[
C(s) = 10\,L_q(s) \;+\; 15\,s
\]

donde \(L_q(s)\) es el número promedio de clientes en cola en el sistema M/M/s.

\subsubsection{Código Python}

\begin{lstlisting}[language=Python]
import numpy as np
import matplotlib.pyplot as plt

# Parámetros
rate_llegada   = 4    # lambda (llegadas/hora)
rate_atencion  = 2    # mu     (servicios por servidor por hora)
s              = 2    # número de servidores
n_pasos        = 100  # número de intervalos
costo_espera   = 10   # costo por cliente en cola
costo_servidor = 15   # costo por servidor

# Historial
hist_cola      = []
hist_atendidos = []

# Estado inicial
cola = 0

for paso in range(n_pasos):
    # 1) Llegadas: Poisson(lambda * dt)
    llegadas = np.random.poisson(rate_llegada)
    cola += llegadas

    # 2) Atenciones: Poisson(s * mu * dt), hasta agotar la cola
    capacidad = np.random.poisson(s * rate_atencion)
    atendidos = min(cola, capacidad)
    cola -= atendidos

    # 3) Guardar historial
    hist_cola.append(cola)
    hist_atendidos.append(atendidos)

# 4) Costo final
cola_final = hist_cola[-1]
costo_total = costo_espera * cola_final + costo_servidor * s

print(f"Clientes en cola al final: {cola_final}")
print(f"Costo total = {costo_espera}*{cola_final} + {costo_servidor}*{s} = {costo_total:.2f}")

# 5) Gráfica
plt.figure(figsize=(10,4))
plt.plot(hist_cola,      label="Clientes en cola",    marker='o', markersize=3)
plt.plot(hist_atendidos, label="Atendidos por paso", marker='x', markersize=3)
plt.xlabel("Paso")
plt.ylabel("Número de clientes")
plt.title(f"Simulación M/M/{s} para {n_pasos} pasos")
plt.legend()
plt.grid(True)
plt.tight_layout()
plt.show()
\end{lstlisting}

\subsubsection{Solución} 

\begin{itemize}
    \item Clientes en cola al final: 13
    \item Costo total = 10*13 + 15*2 = 160.00
\end{itemize}

\begin{center}
    \includegraphics[width=0.9\textwidth]{img/output_simulacion_uno.png}
\end{center}

\subsection{¿Y si tengo diferentes S?}

\begin{lstlisting}[language=Python]
import numpy as np
import matplotlib.pyplot as plt

# Parámetros base
rate_llegada   = 4    # lambda (llegadas/hora)
rate_atencion  = 2    # mu     (servicios por servidor por hora)
s_max          = 5    # número máximo de servidores a probar
n_pasos        = 100  # número de intervalos
costo_espera   = 10   # costo por cliente en cola (por intervalo)
costo_servidor = 15   # costo por servidor (por intervalo)

def M_M_s(rate_llegada, rate_atencion, s, n_pasos, costo_espera, costo_servidor):
    """
    Simula un sistema M/M/s discretizado en n_pasos.
    Retorna:
        hist_cola      : lista con la longitud de cola en cada paso
        hist_atendidos : lista con atendidos en cada paso
        cola_final     : longitud de cola al final de la simulación
        costo_total    : costo = costo_espera * media_cola + costo_servidor * s
    """
    hist_cola      = []
    hist_atendidos = []
    cola = 0

    for paso in range(n_pasos):
        # 1) Llegadas: Poisson(rate_llegada)
        llegadas = np.random.poisson(rate_llegada)
        cola += llegadas

        # 2) Atenciones: Poisson(s * rate_atencion)
        capacidad = np.random.poisson(s * rate_atencion)
        atendidos = min(cola, capacidad)
        cola -= atendidos

        # 3) Guardar historial
        hist_cola.append(cola)
        hist_atendidos.append(atendidos)

    media_cola   = np.mean(hist_cola)
    costo_total  = costo_espera * media_cola + costo_servidor * s

    return hist_cola, hist_atendidos, costo_total

# Simular para s = 1 ... s_max
costos_totales = []
for i in range(1, s_max + 1):
    hist_cola, hist_atendidos, costo_total = M_M_s(
        rate_llegada, rate_atencion, i, n_pasos, costo_espera, costo_servidor)
    costos_totales.append(costo_total)

    # Gráfico de evolución para cada s
    plt.figure(figsize=(10,4))
    plt.plot(hist_cola,      label="Clientes en cola",    marker='o', markersize=3)
    plt.axhline(np.mean(hist_cola), color='r', linestyle='--',
                label=f"Media cola = {np.mean(hist_cola):.2f}")
    plt.plot(hist_atendidos, label="Atendidos por paso", marker='x', markersize=3)
    plt.axhline(np.mean(hist_atendidos), color='g', linestyle='--',
                label=f"Media atendidos = {np.mean(hist_atendidos):.2f}")
    plt.xlabel("Paso")
    plt.ylabel("Número de clientes")
    plt.title(f"Simulación M/M/{i} para {n_pasos} pasos")
    plt.legend()
    plt.grid(True)
    plt.tight_layout()
    plt.show()

# Gráfico de costos totales vs número de servidores
plt.figure(figsize=(8,4))
plt.plot(range(1, s_max+1), costos_totales, '-o')
plt.xlabel("Número de servidores $s$")
plt.ylabel("Costo total por intervalo")
plt.title("Costo total vs. número de servidores")
plt.grid(True)
plt.tight_layout()
plt.show()
\end{lstlisting}

\begin{center}
    \includegraphics[width=0.9\textwidth]{img/output_simulacion_dos_1.png}
\end{center}

\begin{center}
    \includegraphics[width=0.9\textwidth]{img/output_simulacion_dos_2.png}
\end{center}

\begin{center}
    \includegraphics[width=0.9\textwidth]{img/output_simulacion_dos_3.png}
\end{center}

\begin{center}
    \includegraphics[width=0.9\textwidth]{img/output_simulacion_dos_4.png}
\end{center}

\begin{center}
    \includegraphics[width=0.9\textwidth]{img/output_simulacion_dos_5.png}
\end{center}

\begin{center}
    \includegraphics[width=0.9\textwidth]{img/output_simulacion_dos_6.png}
\end{center}

\section{¿Y si hacemos Montecarlo?}

\begin{lstlisting}[language=Python]
import numpy as np
import matplotlib.pyplot as plt

# Parámetros base
rate_llegada   = 4
rate_atencion  = 2
s              = 3     # fijamos s=3
n_pasos        = 100
costo_espera   = 10
costo_servidor = 15
n_rep          = 1000  # número de réplicas Monte Carlo

# Función simulación
def M_M_s(rate_llegada, rate_atencion, s, n_pasos, costo_espera, costo_servidor):
    hist_cola = []
    hist_atendidos = []
    cola = 0

    for _ in range(n_pasos):
        # Llegadas y atenciones como procesos Poisson
        llegadas  = np.random.poisson(rate_llegada)
        capacidad = np.random.poisson(s * rate_atencion)
        cola += llegadas
        atendidos = min(cola, capacidad)
        cola -= atendidos

        hist_cola.append(cola)
        hist_atendidos.append(atendidos)

    media_cola  = np.mean(hist_cola)
    costo_total = costo_espera * media_cola + costo_servidor * s
    return np.array(hist_cola), np.array(hist_atendidos), costo_total

# Almacenar todos los históricos
all_colas      = np.zeros((n_rep, n_pasos))
all_atendidos  = np.zeros((n_rep, n_pasos))
all_costos     = np.zeros(n_rep)

for i in range(n_rep):
    hc, ha, ctot = M_M_s(rate_llegada, rate_atencion, s, n_pasos, costo_espera, costo_servidor)
    all_colas[i]     = hc
    all_atendidos[i] = ha
    all_costos[i]    = ctot

# Calcular estadísticos
mean_cola     = all_colas.mean(axis=0)
p5_cola, p95_cola = np.percentile(all_colas, [5,95], axis=0)

mean_atendidos    = all_atendidos.mean(axis=0)
p5_att, p95_att   = np.percentile(all_atendidos, [5,95], axis=0)

# 1) Trayectorias promedio con bandas 5-95%
fig, axes = plt.subplots(2,1, figsize=(10,8), sharex=True)

# Cola
axes[0].plot(mean_cola, color='C0', label='Media cola')
axes[0].fill_between(range(n_pasos), p5_cola, p95_cola, color='C0', alpha=0.2, label='5-95\%')
axes[0].set_ylabel('Clientes en cola')
axes[0].legend()
axes[0].grid(True)

# Atendidos
axes[1].plot(mean_atendidos, color='C1', label='Media atendidos')
axes[1].fill_between(range(n_pasos), p5_att, p95_att, color='C1', alpha=0.2, label='5-95\%')
axes[1].set_xlabel('Paso')
axes[1].set_ylabel('Clientes atendidos')
axes[1].legend()
axes[1].grid(True)

plt.suptitle(f'Simulación Monte Carlo M/M/{s} ({n_rep} réplicas)')
plt.tight_layout(rect=[0,0,1,0.95])
plt.show()

# 2) Histograma de costos
plt.figure(figsize=(8,4))
plt.hist(all_costos, bins=30, edgecolor='k', alpha=0.7)
plt.axvline(all_costos.mean(), color='r', linestyle='--', label=f'Media = {all_costos.mean():.2f}')
plt.xlabel('Costo total')
plt.ylabel('Frecuencia')
plt.title('Distribución de costos totales (Monte Carlo)')
plt.legend()
plt.grid(True)
plt.tight_layout()
plt.show()
\end{lstlisting}

\begin{center}
    \includegraphics[width=0.9\textwidth]{img/output_montecarlo_uno_1.png}
\end{center}

\begin{center}
    \includegraphics[width=0.9\textwidth]{img/output_montecarlo_uno_2.png}
\end{center}

\subsection{¿Y con diferentes S?}

\begin{lstlisting}[language=Python]
import numpy as np
import matplotlib.pyplot as plt

# Parámetros base
rate_llegada   = 4
rate_atencion  = 2
s_max          = 5
n_pasos        = 100
costo_espera   = 10
costo_servidor = 15
n_rep          = 1000

def M_M_s(rate_llegada, rate_atencion, s, n_pasos, costo_espera, costo_servidor):
    hist_cola = np.zeros(n_pasos)
    hist_att  = np.zeros(n_pasos)
    cola = 0
    for t in range(n_pasos):
        # Llegadas
        cola += np.random.poisson(rate_llegada)
        # Atenciones
        atend = min(cola, np.random.poisson(s * rate_atencion))
        cola -= atend
        hist_cola[t] = cola
        hist_att[t]  = atend
    media_cola = hist_cola.mean()
    costo_total = costo_espera * media_cola + costo_servidor * s
    return hist_cola, hist_att, costo_total

# Arrays para almacenar resultados
cola_means   = np.zeros((s_max, n_pasos))
att_means    = np.zeros((s_max, n_pasos))
costos_means = np.zeros(s_max)

# Bucle sobre s
for s in range(1, s_max + 1):
    all_colas = np.zeros((n_rep, n_pasos))
    all_atts  = np.zeros((n_rep, n_pasos))
    all_costs = np.zeros(n_rep)

    # Réplicas Monte Carlo
    for i in range(n_rep):
        hc, ha, ctot = M_M_s(rate_llegada, rate_atencion, s, n_pasos, costo_espera, costo_servidor)
        all_colas[i] = hc
        all_atts[i]  = ha
        all_costs[i] = ctot

    # Medias temporales (sobre réplicas)
    cola_means[s-1]   = all_colas.mean(axis=0)
    att_means[s-1]    = all_atts.mean(axis=0)
    costos_means[s-1] = all_costs.mean()

# 1) Plot de medias de cola a lo largo del tiempo, un curve por cada s
plt.figure(figsize=(8,4))
for idx in range(s_max):
    plt.plot(cola_means[idx], label=f's={idx+1}')
plt.xlabel('Paso')
plt.ylabel('Media clientes en cola')
plt.title('Evolución media de la cola para distintos s')
plt.legend()
plt.grid(True)
plt.tight_layout()

# 2) Plot de medias de atendidos a lo largo del tiempo
plt.figure(figsize=(8,4))
for idx in range(s_max):
    plt.plot(att_means[idx], label=f's={idx+1}')
plt.xlabel('Paso')
plt.ylabel('Media clientes atendidos')
plt.title('Evolución media de atendidos para distintos s')
plt.legend()
plt.grid(True)
plt.tight_layout()

# 3) Plot del costo medio final vs s
plt.figure(figsize=(6,4))
plt.plot(range(1, s_max+1), costos_means, '-o')
plt.xlabel('Número de servidores $s$')
plt.ylabel('Costo medio por intervalo')
plt.title('Costo medio vs número de servidores')
plt.grid(True)
plt.tight_layout()

plt.show()
\end{lstlisting}

\begin{center}
    \includegraphics[width=0.9\textwidth]{img/output_montecarlo_dos_1.png}
\end{center}

\begin{center}
    \includegraphics[width=0.9\textwidth]{img/output_montecarlo_dos_2.png}
\end{center}

\begin{center}
    \includegraphics[width=0.9\textwidth]{img/output_montecarlo_dos_3.png}
\end{center}

\end{document}