\documentclass[12pt]{article}

\usepackage[utf8]{inputenc}
\usepackage[T1]{fontenc}
\usepackage{lmodern}
\usepackage[spanish]{babel}
\usepackage{booktabs}
\usepackage{amsmath}
\usepackage{amssymb}
\usepackage{amsthm}
\usepackage{forest}
\usepackage{float}
\usepackage{listings}
\usepackage{xcolor}
\usepackage{tikz}
\usepackage{pgfplots}
\pgfplotsset{compat=1.18}
\usepackage{graphicx}
\usepackage{hyperref}
\usepackage{geometry}
\usepackage{enumitem}
\usepackage{multicol}
\usepackage{siunitx}

\definecolor{codegreen}{rgb}{0,0.6,0}
\definecolor{codegray}{rgb}{0.5,0.5,0.5}
\definecolor{codepurple}{rgb}{0.58,0,0.82}
\definecolor{backcolour}{rgb}{0.95,0.95,0.92}

\lstdefinestyle{mystyle}{
    backgroundcolor=\color{backcolour},   
    commentstyle=\color{codegreen},
    keywordstyle=\color{magenta},
    numberstyle=\tiny\color{codegray},
    stringstyle=\color{codepurple},
    basicstyle=\ttfamily\footnotesize,
    breakatwhitespace=false,         
    breaklines=true,                 
    captionpos=b,                    
    keepspaces=true,                 
    numbers=left,                    
    numbersep=5pt,                  
    showspaces=false,                
    showstringspaces=false,
    showtabs=false,                  
    tabsize=2
}

\lstset{style=mystyle}

\sloppy
\setlength{\parindent}{0pt}

\begin{document}

\begin{center}
    {\LARGE \textbf{Guía de Ejercicios \\Python para I. O.}}\\[0.5em]
    {Investigación Operativa, Universidad de San Andrés}
\end{center}

Si encuentran algún error en el documento o hay alguna duda, mandenmé un mail a rodriguezf@udesa.edu.ar y lo revisamos.

\section{Ejercicios}

\subsection{Ejercicio 1}
Escribir un código que imprima en la consola las siguientes frases o el resultado de las operaciones matemáticas:
\begin{enumerate}[label=\alph*)]
    \item “Alo mundo!”
    \item 2 + 3
    \item 2 * 3
    \item $2^3$
    \item $\frac{2}{3}$
    \item Resto de la división 2/3
\end{enumerate}
    
\subsection{Ejercicio 2}
Escribir un código que imprima todos los números pares entre 0 y 31, utilizando \textbf{for} loops.
    
\subsection{Ejercicio 3}
Escribir un código que compute el promedio de la lista de números [1,32,53,14,55,36,27]. Hacerlo de dos maneras distintas:
\begin{enumerate}[label=\alph*)]
    \item Mediante for loops
    \item Usando la función \texttt{np.mean( )}
\end{enumerate}
    
\subsection{Ejercicio 4}
Escribir una función que tome como input dos números $x_1,x_2$ e imprima a la consola la suma de esos dos números.
    
\subsection{Ejercicio 5}

\begin{enumerate}[label=\alph*)]
    \item Importar la librería numpy con el comando ``\texttt{import numpy as np}''
    \item Considerar la matriz  
    \[
    A = \begin{bmatrix}
    1 & 2 & 0\\[4pt]
    3 & 0 & 4\\[4pt]
    1 & 0 & 3
    \end{bmatrix}
    \]
    \item Transformar en un array de numpy con el comando ``\texttt{A = np.array(A)}''
    \item Corroborar que los comandos \texttt{A[0][0]} y \texttt{A[0,0]} devuelven el primer elemento en la primera fila, y primera columna.
    \item Corroborar que el comando \texttt{A[:,1]} devuelve la segunda columna.
    \item Corroborar que los comandos \texttt{A[1]} y \texttt{A[1,:]} devuelven la segunda fila.
    \item Corroborar que el comando \texttt{A[:, -1]} devuelve la última columna.
    \item Corroborar que \texttt{A[0:2]} y \texttt{A[:,0:2]} devuelven las primeras dos filas y las primeras dos columnas respectivamente.
    \item ¿Qué devuelven los comandos \texttt{A[-1, -1]}, \texttt{A[0:2]}, \texttt{A[0:2, 0]}, \texttt{A[0:2, 0:2]}?
\end{enumerate}

    
\subsection{Ejercicio 6}
\begin{enumerate}[label=\alph*)]
    \item Escribir un for loop que dadas las dos listas:
    \[
    A = [2,\, 10,\, 16,\, 2,\, 4,\, 12,\, 24,\, 100]
    \]
    \[
    B = [5,\, 2,\, 5,\, 2,\, 1,\, 2,\, 1,\, 0.5]
    \]
    sume la multiplicación coordenada a coordenada de todos sus elementos, es decir:
    \[
    A[0]*B[0] + A[1]*B[1] + A[2]*B[2] + \dots = 2*5 + 10*2 + 16*5 \dots
    \]
    \textbf{Nota: }Esta operación es llamada producto interno entre dos vectores o listas.
    \item Realizar la cuenta a mano y verificar que el resultado es el mismo que en Python.
    \item Importar la librería numpy y transformar ambas listas en arrays de numpy usando np.array
    \item Corroborar que ahora el comando ``\texttt{np.dot(A, B)}'' da el mismo resultado que en (a) y (b).
\end{enumerate}
    
\subsection{Ejercicio 7}
\begin{enumerate}[label=\alph*)]
    \item Considerar las matrices
    \[
    A = \begin{bmatrix}
    1 & 2 & 0\\[4pt]
    3 & 0 & 4\\[4pt]
    1 & 0 & 3
    \end{bmatrix}, \quad
    B = \begin{bmatrix}
    3 & 1 & 1\\[4pt]
    1 & 2 & 0\\[4pt]
    2 & 4 & 1
    \end{bmatrix}
    \]
    \item Transformarlas en arrays de numpy
    \textbf{Definición:} Se define a la multiplicación de matrices $A*B$ como la matriz que en la coordenada $i,j$ (fila $i$, columna $j$) tiene al producto interno de la fila $i$ de $A$ con la columna $j$ de $B$. En este caso la matriz $A*B$ también es de tamaño $3 \times 3$ (son 9 operaciones de producto interno).
    \item Realizar a mano la multiplicación de las matrices $A$ y $B$ del punto (a).
    \item Hacer un doble \textbf{for} loop y utilizar lo aprendido en el ejercicio anterior (el 6) para construir la matriz $A*B$ en Python. Corroborar que da lo mismo que a mano.
    \item Multiplicar las matrices utilizando el comando abreviado de numpy `A@B'. Corroborar que también da lo mismo.
\end{enumerate}
    
\subsection{Ejercicio 8}
Escribir una función que tome como input dos números $x_1, x_2$, y determine si están en el conjunto factible definido por las siguientes restricciones lineales. El output debe ser un booleano \texttt{True/False}.\\[0.5em]
    \[
    \begin{aligned}
        2x_1 + 3x_2 &\leq 24 \\
        x_1 &\geq 0 \\
        x_2 &\geq 0
        \end{aligned}
    \]

Usando esta función escribir código que determine si los siguientes puntos están en el conjunto factible.
\begin{enumerate}[label=\alph*)]
    \item $(x_1,x_2) = (1,1)$
    \item $(x_1,x_2) = (12,8)$
    \item $(x_1,x_2) = (4,8)$
    \item $(x_1,x_2) = (-1,0)$
    \item $(x_1,x_2) = (-2,2)$
\end{enumerate}

\newpage

\section{Anexo: Soluciones}

% Ejercicio 1
\subsection{Ejercicio 1}
\begin{lstlisting}[language=Python]
print("Alo mundo!")
print(2 + 3)
print(2 * 3)
print(2 ** 3)
print(2 / 3)
print(2 % 3)
\end{lstlisting}

% Ejercicio 2
\subsection{Ejercicio 2}
\begin{lstlisting}[language=Python]
for i in range(0, 32, 2):
    print(i)
\end{lstlisting}

% Ejercicio 3
\subsection{Ejercicio 3}
\begin{lstlisting}[language=Python]
# a)
nums = [1, 32, 53, 14, 55, 36, 27]
suma = 0
for n in nums:
    suma += n
promedio = suma / len(nums)
print(promedio)

# b)
import numpy as np
nums = [1, 32, 53, 14, 55, 36, 27]
print(np.mean(nums))
\end{lstlisting}

% Ejercicio 4
\subsection{Ejercicio 4}
\begin{lstlisting}[language=Python]
def suma(x1, x2):
    print(x1 + x2)
\end{lstlisting}

% Ejercicio 5
\subsection{Ejercicio 5}
\begin{lstlisting}[language=Python]
# a)
import numpy as np
A = [[1, 2, 0], [3, 0, 4], [1, 0, 3]]
A = np.array(A)

# d)
print(A[0][0])
print(A[0,0])

# e)
print(A[:,1])

# f)
print(A[1])
print(A[1,:])

# g)
print(A[:,-1])

# h)
print(A[0:2])
print(A[:,0:2])

# i)
print(A[-1,-1])
print(A[0:2])
print(A[0:2,0])
print(A[0:2,0:2])
\end{lstlisting}

% Ejercicio 6
\subsection{Ejercicio 6}
\begin{lstlisting}[language=Python]
# a)
A = [2, 10, 16, 2, 4, 12, 24, 100]
B = [5, 2, 5, 2, 1, 2, 1, 0.5]
res = 0

for i in range(len(A)):
    res += A[i] * B[i]
print(res)

# c)
import numpy as np

A = np.array([2, 10, 16, 2, 4, 12, 24, 100])
B = np.array([5, 2, 5, 2, 1, 2, 1, 0.5])
print(np.dot(A, B))
\end{lstlisting}

% Ejercicio 7
\subsection{Ejercicio 7}
\begin{lstlisting}[language=Python]
# a)
import numpy as np
A = np.array([[1, 2, 0], [3, 0, 4], [1, 0, 3]])
B = np.array([[3, 1, 1], [1, 2, 0], [2, 4, 1]])

# d)
C = np.zeros((3,3))
for i in range(3):
    for j in range(3):
        C[i,j] = np.dot(A[i,:], B[:,j])
print(C)

# e)
print(A @ B)
\end{lstlisting}

% Ejercicio 8
\subsection{Ejercicio 8}
\begin{lstlisting}[language=Python]
def factible(x1, x2):
    return 2*x1 + 3*x2 <= 24 and x1 >= 0 and x2 >= 0

print(factible(1,1))
print(factible(12,8))
print(factible(4,8))
print(factible(-1,0))
print(factible(-2,2))
\end{lstlisting}

\end{document}
