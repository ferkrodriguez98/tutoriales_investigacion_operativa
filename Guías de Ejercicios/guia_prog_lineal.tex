\documentclass[12pt]{article}

\usepackage[utf8]{inputenc}
\usepackage[T1]{fontenc}
\usepackage{lmodern}
\usepackage[spanish]{babel}
\usepackage{booktabs}
\usepackage{amsmath}
\usepackage{amssymb}
\usepackage{amsthm}
\usepackage{forest}
\usepackage{float}
\usepackage{listings}
\usepackage{xcolor}
\usepackage{tikz}
\usepackage{pgfplots}
\pgfplotsset{compat=1.18}
\usepackage{graphicx}
\usepackage{hyperref}
\usepackage{geometry}
\usepackage{enumitem}
\usepackage{multicol}
\usepackage{siunitx}

\definecolor{codegreen}{rgb}{0,0.6,0}
\definecolor{codegray}{rgb}{0.5,0.5,0.5}
\definecolor{codepurple}{rgb}{0.58,0,0.82}
\definecolor{backcolour}{rgb}{0.95,0.95,0.92}

\lstdefinestyle{mystyle}{
    backgroundcolor=\color{backcolour},   
    commentstyle=\color{codegreen},
    keywordstyle=\color{magenta},
    numberstyle=\tiny\color{codegray},
    stringstyle=\color{codepurple},
    basicstyle=\ttfamily\footnotesize,
    breakatwhitespace=false,         
    breaklines=true,                 
    captionpos=b,                    
    keepspaces=true,                 
    numbers=left,                    
    numbersep=5pt,                  
    showspaces=false,                
    showstringspaces=false,
    showtabs=false,                  
    tabsize=2
}

\lstset{style=mystyle}

\sloppy
\setlength{\parindent}{0pt}

\begin{document}

\begin{center}
    {\LARGE \textbf{Guía de Ejercicios \\Programación Lineal}}\\[0.5em]
    {Investigación Operativa, Universidad de San Andrés}
\end{center}

Esta guía contiene 20 ejercicios ordenados por dificultad creciente. Las respuestas y el código de resolución se encuentran en el anexo al final del documento.

\begin{enumerate}

\item Una empresa fabrica dos productos A y B. El producto A requiere 2 horas de mano de obra y 3 kg de materia prima, mientras que el producto B requiere 3 horas de mano de obra y 2 kg de materia prima. La empresa dispone de 100 horas de mano de obra y 80 kg de materia prima por semana. El beneficio por unidad es de \$40 para A y \$30 para B. Determine la cantidad óptima a producir de cada producto para maximizar el beneficio.

\item Una dietista debe preparar una mezcla nutritiva que contenga al menos 12 unidades de vitamina A y 15 unidades de vitamina B. Dispone de dos tipos de alimentos: el alimento 1 contiene 2 unidades de vitamina A y 3 de B por kg, y cuesta \$4 por kg; el alimento 2 contiene 3 unidades de vitamina A y 1 de B por kg, y cuesta \$3 por kg. ¿Qué cantidad debe usar de cada alimento para minimizar el costo?

\item Una fábrica produce sillas y mesas. Cada silla requiere 2 unidades de madera y 1 unidad de trabajo, mientras que cada mesa requiere 3 unidades de madera y 2 unidades de trabajo. La fábrica dispone de 60 unidades de madera y 40 unidades de trabajo. La ganancia por silla es de \$20 y por mesa es de \$30. ¿Cuántas sillas y mesas debe producir para maximizar la ganancia?

\item Una empresa de transporte tiene dos camiones. El camión 1 puede transportar 10 toneladas y recorre 40 km/h, mientras que el camión 2 puede transportar 15 toneladas y recorre 30 km/h. Se necesita transportar al menos 50 toneladas de mercancía y se dispone de 8 horas. El costo por hora del camión 1 es \$100 y del camión 2 es \$120. ¿Cuántos viajes debe hacer cada camión para minimizar el costo?

\item Una empresa produce dos tipos de juguetes: coches y trenes. Cada coche requiere 4 horas en el departamento A y 2 horas en el B, mientras que cada tren requiere 2 horas en A y 5 horas en B. Se dispone de 100 horas en A y 80 horas en B. El beneficio es de \$8 por coche y \$7 por tren. ¿Cuántos juguetes de cada tipo debe producir?

\item Una empresa produce tres tipos de productos: A, B y C. La siguiente tabla muestra los recursos necesarios por unidad y la disponibilidad total:

\begin{center}
\begin{tabular}{lcccc}
\toprule
Recurso & Producto A & Producto B & Producto C & Disponible \\
\midrule
Mano de obra (h) & 4 & 3 & 5 & 400 \\
Material (kg) & 2 & 4 & 3 & 300 \\
Tiempo máquina (h) & 3 & 2 & 4 & 350 \\
\midrule
Beneficio (\$/u) & 100 & 80 & 120 & \\
\bottomrule
\end{tabular}
\end{center}

¿Cuántas unidades de cada producto debe fabricar para maximizar el beneficio?

\item Una empresa de inversiones tiene \$100,000 para invertir en tres opciones diferentes. La opción A tiene un rendimiento del 8\% anual pero requiere una inversión mínima de \$20,000. La opción B tiene un rendimiento del 12\% anual con un máximo de inversión de \$50,000. La opción C tiene un rendimiento del 10\% anual. Por política de la empresa, la inversión en C debe ser al menos el 30\% de la inversión total. ¿Cómo debe distribuir el dinero para maximizar el rendimiento?

\item Una fábrica de muebles produce mesas, sillas y estantes. Cada producto requiere madera, tiempo de carpintería y tiempo de acabado según la siguiente tabla:

\begin{center}
\begin{tabular}{lccc}
\toprule
Recurso & Mesa & Silla & Estante \\
\midrule
Madera (m²) & 3 & 1 & 2 \\
Carpintería (h) & 4 & 2 & 3 \\
Acabado (h) & 2 & 1 & 2 \\
\midrule
Beneficio (\$) & 200 & 80 & 150 \\
\bottomrule
\end{tabular}
\end{center}

Se dispone de 300 m² de madera, 400 horas de carpintería y 200 horas de acabado. El mercado exige que se produzcan al menos 30 sillas y que la cantidad de estantes sea al menos la mitad de la cantidad de mesas. ¿Cuántas unidades de cada producto debe fabricar?

\item Una empresa de transporte debe planificar el envío de mercancías entre tres almacenes y cuatro tiendas. Las demandas de las tiendas son 300, 200, 400 y 100 unidades respectivamente. Los almacenes tienen capacidades de 400, 300 y 300 unidades. Los costos de transporte (en \$ por unidad) se muestran en la siguiente tabla:

\begin{center}
\begin{tabular}{lcccc}
\toprule
& Tienda 1 & Tienda 2 & Tienda 3 & Tienda 4 \\
\midrule
Almacén 1 & 10 & 8 & 6 & 9 \\
Almacén 2 & 7 & 11 & 8 & 5 \\
Almacén 3 & 6 & 9 & 7 & 12 \\
\bottomrule
\end{tabular}
\end{center}

¿Cómo debe realizarse el transporte para minimizar los costos?

\item Una refinería procesa tres tipos de petróleo crudo (A, B y C) para producir gasolina regular y premium. La siguiente tabla muestra los barriles de cada tipo de gasolina que se obtienen por barril de crudo procesado:

\begin{center}
\begin{tabular}{lccc}
\toprule
& Crudo A & Crudo B & Crudo C \\
\midrule
Gasolina Regular & 0.5 & 0.4 & 0.3 \\
Gasolina Premium & 0.3 & 0.4 & 0.5 \\
\bottomrule
\end{tabular}
\end{center}

El costo por barril de los crudos A, B y C es \$60, \$70 y \$80 respectivamente. La demanda mínima es de 10,000 barriles de gasolina regular y 8,000 de premium. La refinería tiene una capacidad de procesamiento de 30,000 barriles de crudo. ¿Cuántos barriles de cada tipo de crudo debe procesar para minimizar el costo?

\item Una empresa farmacéutica produce tres tipos de medicamentos (A, B y C) que requieren dos tipos de materias primas (M1 y M2). La empresa tiene contratos con tres proveedores diferentes que ofrecen paquetes de materias primas a diferentes precios:

\begin{center}
\begin{tabular}{lccc}
\toprule
& Proveedor 1 & Proveedor 2 & Proveedor 3 \\
\midrule
M1 (kg/paquete) & 100 & 150 & 200 \\
M2 (kg/paquete) & 150 & 100 & 175 \\
Costo (\$/paquete) & 5000 & 5500 & 7000 \\
\bottomrule
\end{tabular}
\end{center}

Cada unidad de medicamento requiere:
\begin{center}
\begin{tabular}{lccc}
\toprule
& Medicamento A & Medicamento B & Medicamento C \\
\midrule
M1 (kg) & 2 & 3 & 1.5 \\
M2 (kg) & 3 & 2 & 4 \\
Beneficio (\$/u) & 200 & 180 & 150 \\
\bottomrule
\end{tabular}
\end{center}

La demanda mínima mensual es de 500 unidades para A, 400 para B y 300 para C. Además, por regulaciones, la producción de C no puede exceder el 40\% de la producción total. La empresa quiere determinar cuántos paquetes comprar a cada proveedor y cuántas unidades producir de cada medicamento para maximizar el beneficio.

\item Una empresa de logística debe planificar el envío de productos entre 4 centros de distribución y 5 destinos durante 3 períodos. Los costos de envío varían según el período debido a factores estacionales. Cada centro tiene una capacidad máxima de almacenamiento y procesamiento por período. Los destinos tienen demandas que deben satisfacerse en cada período, y se permite almacenar inventario entre períodos con un costo de \$2 por unidad.

Datos:
\begin{itemize}
\item Capacidades de los centros (unidades/período): 1000, 1200, 800, 900
\item Demandas por destino y período:

\begin{center}
\begin{tabular}{lccc}
\toprule
Destino & Período 1 & Período 2 & Período 3 \\
\midrule
1 & 400 & 500 & 600 \\
2 & 300 & 400 & 300 \\
3 & 500 & 600 & 400 \\
4 & 200 & 300 & 500 \\
5 & 400 & 300 & 400 \\
\bottomrule
\end{tabular}
\end{center}

\item Costos de envío (varían por período, se multiplican por estos factores): Período 1: 1.0, Período 2: 1.2, Período 3: 0.9
\end{itemize}

Matriz base de costos de envío (\$/unidad):
\begin{center}
\begin{tabular}{lccccc}
\toprule
Centro & Destino 1 & Destino 2 & Destino 3 & Destino 4 & Destino 5 \\
\midrule
1 & 10 & 12 & 8 & 11 & 14 \\
2 & 13 & 9 & 14 & 10 & 12 \\
3 & 11 & 13 & 10 & 12 & 9 \\
4 & 12 & 11 & 13 & 9 & 10 \\
\bottomrule
\end{tabular}
\end{center}

\item Una empresa de manufactura debe programar la producción de 4 productos en 3 máquinas diferentes durante 6 períodos. Cada producto requiere un tiempo de procesamiento específico en cada máquina y debe seguir una secuencia determinada. Los productos pueden almacenarse entre períodos con un costo, y hay penalizaciones por período de retraso.

Datos:
\begin{itemize}
\item Tiempos de procesamiento (horas/unidad):

\begin{center}
\begin{tabular}{lccc}
\toprule
Producto & Máquina 1 & Máquina 2 & Máquina 3 \\
\midrule
A & 2 & 3 & 1.5 \\
B & 1.5 & 2 & 2 \\
C & 3 & 1.5 & 2.5 \\
D & 2.5 & 2.5 & 1 \\
\bottomrule
\end{tabular}
\end{center}

\item Secuencias requeridas:
  \begin{itemize}
  \item Producto A: Máquina 1 → 2 → 3
  \item Producto B: Máquina 2 → 1 → 3
  \item Producto C: Máquina 1 → 3 → 2
  \item Producto D: Máquina 3 → 1 → 2
  \end{itemize}
\item Demandas y fechas de entrega:

\begin{center}
\begin{tabular}{lccc}
\toprule
Producto & Cantidad & Período de entrega & Penalización \\
\midrule
A & 100 & 4 & \$500 \\
B & 150 & 3 & \$600 \\
C & 80 & 5 & \$400 \\
D & 120 & 6 & \$450 \\
\bottomrule
\end{tabular}
\end{center}

\item Costos:
  \begin{itemize}
  \item Almacenamiento: \$50/unidad/período
  \item Tiempo extra de máquina: \$200/hora
  \item Capacidad regular por máquina: 40 horas/período
  \end{itemize}
\end{itemize}

\item Una empresa de energía debe planificar la operación de 5 centrales eléctricas para satisfacer la demanda variable durante 24 períodos (horas). Cada central tiene características diferentes de generación, costos y restricciones técnicas.

Datos:
\begin{itemize}
\item Características de las centrales:

\begin{center}
\begin{tabular}{lccccc}
\toprule
Característica & C1 & C2 & C3 & C4 & C5 \\
\midrule
Capacidad Mín (MW) & 100 & 50 & 80 & 30 & 40 \\
Capacidad Máx (MW) & 400 & 200 & 300 & 150 & 180 \\
Costo Fijo (\$/h) & 1000 & 800 & 1200 & 500 & 600 \\
Costo Variable (\$/MWh) & 45 & 55 & 40 & 65 & 50 \\
Tiempo Mín Operación (h) & 4 & 2 & 3 & 1 & 2 \\
Tiempo Mín Apagado (h) & 3 & 2 & 4 & 1 & 2 \\
Rampa Subida (MW/h) & 100 & 80 & 60 & 100 & 90 \\
Rampa Bajada (MW/h) & 90 & 70 & 50 & 90 & 80 \\
\bottomrule
\end{tabular}
\end{center}

\item Demanda por hora (MW):

\begin{center}
\begin{tabular}{cccccccc}
\toprule
H & D & H & D & H & D & H & D \\
\midrule
1 & 400 & 7 & 800 & 13 & 900 & 19 & 1000 \\
2 & 350 & 8 & 900 & 14 & 850 & 20 & 950 \\
3 & 300 & 9 & 950 & 15 & 800 & 21 & 850 \\
4 & 350 & 10 & 1000 & 16 & 850 & 22 & 700 \\
5 & 500 & 11 & 950 & 17 & 900 & 23 & 500 \\
6 & 700 & 12 & 900 & 18 & 950 & 24 & 400 \\
\bottomrule
\end{tabular}
\end{center}

\item Se requiere una reserva rodante del 10\% de la demanda en cada hora
\item Costo de arranque: \$2000 por central
\end{itemize}

\item Una empresa de logística debe diseñar su red de distribución considerando la ubicación de centros de distribución (CD), asignación de clientes y rutas de vehículos. Se tienen 3 posibles ubicaciones para CDs, 20 clientes y una flota de vehículos heterogénea.

Datos:
\begin{itemize}
\item Costos fijos de apertura de CD:
  \begin{itemize}
  \item CD1: \$500,000
  \item CD2: \$450,000
  \item CD3: \$600,000
  \end{itemize}
\item Capacidades de los CDs:
  \begin{itemize}
  \item CD1: 5000 unidades/día
  \item CD2: 4000 unidades/día
  \item CD3: 6000 unidades/día
  \end{itemize}
\item Flota de vehículos:
  \begin{itemize}
  \item Tipo 1: 5 vehículos, capacidad 500 unidades, costo \$2/km
  \item Tipo 2: 3 vehículos, capacidad 800 unidades, costo \$2.5/km
  \item Tipo 3: 2 vehículos, capacidad 1000 unidades, costo \$3/km
  \end{itemize}
\item Demandas de clientes: varían entre 100 y 400 unidades/día
\item Matriz de distancias entre todos los puntos disponible
\item Restricciones de tiempo:
  \begin{itemize}
  \item Máximo 8 horas por ruta
  \item Velocidad promedio: 50 km/h
  \item Tiempo de servicio por cliente: 20 minutos
  \end{itemize}
\end{itemize}

\newpage

\section*{Anexo: Respuestas y Código de Resolución}

\subsection*{Ejercicio 1}

\textbf{Planteo:}
\begin{itemize}
\item Variables: $x_1$ = cantidad de A, $x_2$ = cantidad de B
\item Función objetivo: Max $Z = 40x_1 + 30x_2$
\item Restricciones:
  \begin{align*}
  2x_1 + 3x_2 &\leq 100 \text{ (mano de obra)} \\
  3x_1 + 2x_2 &\leq 80 \text{ (materia prima)} \\
  x_1, x_2 &\geq 0
  \end{align*}
\end{itemize}

\textbf{Código de resolución en PICOS:}
\begin{lstlisting}[language=Python]
import picos
import numpy as np

P = picos.Problem()

x = picos.RealVariable('x', 2)

# Matriz de restricciones
A = np.array([
    [2, 3],  # mano de obra
    [3, 2]   # materia prima
])
b = np.array([100, 80])
c = np.array([40, 30])

A = picos.Constant('A', A)
b = picos.Constant('b', b)
c = picos.Constant('c', c)

P.set_objective('max', c | x)
P.add_constraint(A * x <= b)
P.add_constraint(x >= 0)

P.solve(solver='glpk')
print(f"x1 = {x[0].value}, x2 = {x[1].value}")
print(f"Z = {P.value}")
\end{lstlisting}

\textbf{Salida de la consola:}
\begin{lstlisting}[language=bash,backgroundcolor=\color{black},basicstyle=\color{white}\ttfamily,numbers=none]
x1 = 20.0, x2 = 20.0
Z = 1400.0
\end{lstlisting}

\subsection*{Ejercicio 2}

\textbf{Planteo:}
\begin{itemize}
\item Variables: $x_1$ = kg de alimento 1, $x_2$ = kg de alimento 2
\item Función objetivo: Min $Z = 4x_1 + 3x_2$
\item Restricciones:
  \begin{align*}
  2x_1 + 3x_2 &\geq 12 \text{ (vitamina A)} \\
  3x_1 + x_2 &\geq 15 \text{ (vitamina B)} \\
  x_1, x_2 &\geq 0
  \end{align*}
\end{itemize}

\textbf{Código de resolución en PICOS:}
\begin{lstlisting}[language=Python]
import picos
import numpy as np

P = picos.Problem()

x = picos.RealVariable('x', 2)

# Matriz de restricciones
A = np.array([
    [2, 3],  # vitamina A
    [3, 1]   # vitamina B
])
b = np.array([12, 15])
c = np.array([4, 3])

A = picos.Constant('A', A)
b = picos.Constant('b', b)
c = picos.Constant('c', c)

P.set_objective('min', c | x)
P.add_constraint(A * x >= b)
P.add_constraint(x >= 0)

P.solve(solver='glpk')
print(f"x1 = {x[0].value}, x2 = {x[1].value}")
print(f"Z = {P.value}")
\end{lstlisting}

\textbf{Salida de la consola:}
\begin{lstlisting}[language=bash,backgroundcolor=\color{black},basicstyle=\color{white}\ttfamily,numbers=none]
x1 = 4.0, x2 = 3.0
Z = 25.0
\end{lstlisting}

\subsection*{Ejercicio 3}

\textbf{Planteo:}
\begin{itemize}
\item Variables: $x_1$ = cantidad de sillas, $x_2$ = cantidad de mesas
\item Función objetivo: Max $Z = 20x_1 + 30x_2$
\item Restricciones:
  \begin{align*}
  2x_1 + 3x_2 &\leq 60 \text{ (madera)} \\
  x_1 + 2x_2 &\leq 40 \text{ (trabajo)} \\
  x_1, x_2 &\geq 0
  \end{align*}
\end{itemize}

\textbf{Código de resolución en PICOS:}
\begin{lstlisting}[language=Python]
import picos
import numpy as np

P = picos.Problem()

x = picos.RealVariable('x', 2)

# Matriz de restricciones
A = np.array([
    [2, 3],  # madera
    [1, 2]   # trabajo
])
b = np.array([60, 40])
c = np.array([20, 30])

A = picos.Constant('A', A)
b = picos.Constant('b', b)
c = picos.Constant('c', c)

P.set_objective('max', c | x)
P.add_constraint(A * x <= b)
P.add_constraint(x >= 0)

P.solve(solver='glpk')
print(f"x1 = {x[0].value}, x2 = {x[1].value}")
print(f"Z = {P.value}")
\end{lstlisting}

\textbf{Salida de la consola:}
\begin{lstlisting}[language=bash,backgroundcolor=\color{black},basicstyle=\color{white}\ttfamily,numbers=none]
x1 = 20.0, x2 = 10.0
Z = 700.0
\end{lstlisting}

\subsection*{Ejercicio 4}

\textbf{Planteo:}
\begin{itemize}
\item Variables: $x_1$ = viajes camión 1, $x_2$ = viajes camión 2
\item Función objetivo: Min $Z = 100x_1 + 120x_2$
\item Restricciones:
  \begin{align*}
  10x_1 + 15x_2 &\geq 50 \text{ (toneladas)} \\
  x_1 + x_2 &\leq 8 \text{ (horas)} \\
  x_1, x_2 &\geq 0
  \end{align*}
\end{itemize}

\textbf{Código de resolución en PICOS:}
\begin{lstlisting}[language=Python]
import picos
import numpy as np

P = picos.Problem()

x = picos.RealVariable('x', 2)

# Restricciones
P.add_constraint(10*x[0] + 15*x[1] >= 50)  # toneladas
P.add_constraint(x[0] + x[1] <= 8)         # horas
P.add_constraint(x >= 0)

# Funcion objetivo
P.set_objective('min', 100*x[0] + 120*x[1])

P.solve(solver='glpk')
print(f"x1 = {x[0].value}, x2 = {x[1].value}")
print(f"Z = {P.value}")
\end{lstlisting}

\textbf{Salida de la consola:}
\begin{lstlisting}[language=bash,backgroundcolor=\color{black},basicstyle=\color{white}\ttfamily,numbers=none]
x1 = 2.0, x2 = 2.0
Z = 440.0
\end{lstlisting}

\subsection*{Ejercicio 5}

\textbf{Planteo:}
\begin{itemize}
\item Variables: $x_1$ = cantidad de coches, $x_2$ = cantidad de trenes
\item Función objetivo: Max $Z = 8x_1 + 7x_2$
\item Restricciones:
  \begin{align*}
  4x_1 + 2x_2 &\leq 100 \text{ (dept. A)} \\
  2x_1 + 5x_2 &\leq 80 \text{ (dept. B)} \\
  x_1, x_2 &\geq 0
  \end{align*}
\end{itemize}

\textbf{Código de resolución en PICOS:}
\begin{lstlisting}[language=Python]
import picos
import numpy as np

P = picos.Problem()

x = picos.RealVariable('x', 2)

# Matriz de restricciones
A = np.array([
    [4, 2],  # departamento A
    [2, 5]   # departamento B
])
b = np.array([100, 80])
c = np.array([8, 7])

A = picos.Constant('A', A)
b = picos.Constant('b', b)
c = picos.Constant('c', c)

P.set_objective('max', c | x)
P.add_constraint(A * x <= b)
P.add_constraint(x >= 0)

P.solve(solver='glpk')
print(f"x1 = {x[0].value}, x2 = {x[1].value}")
print(f"Z = {P.value}")
\end{lstlisting}

\textbf{Salida de la consola:}
\begin{lstlisting}[language=bash,backgroundcolor=\color{black},basicstyle=\color{white}\ttfamily,numbers=none]
x1 = 20.0, x2 = 10.0
Z = 230.0
\end{lstlisting}

\subsection*{Ejercicio 6}

\textbf{Planteo:}
\begin{itemize}
\item Variables: $x_1$, $x_2$, $x_3$ = cantidad de productos A, B y C
\item Función objetivo: Max $Z = 100x_1 + 80x_2 + 120x_3$
\item Restricciones:
  \begin{align*}
  4x_1 + 3x_2 + 5x_3 &\leq 400 \text{ (mano de obra)} \\
  2x_1 + 4x_2 + 3x_3 &\leq 300 \text{ (material)} \\
  3x_1 + 2x_2 + 4x_3 &\leq 350 \text{ (tiempo máquina)} \\
  x_1, x_2, x_3 &\geq 0
  \end{align*}
\end{itemize}

\textbf{Código de resolución en PICOS:}
\begin{lstlisting}[language=Python]
import picos
import numpy as np

P = picos.Problem()

x = picos.RealVariable('x', 3)

# Matriz de restricciones
A = np.array([
    [4, 3, 5],  # mano de obra
    [2, 4, 3],  # material
    [3, 2, 4]   # tiempo maquina
])
b = np.array([400, 300, 350])
c = np.array([100, 80, 120])

A = picos.Constant('A', A)
b = picos.Constant('b', b)
c = picos.Constant('c', c)

P.set_objective('max', c | x)
P.add_constraint(A * x <= b)
P.add_constraint(x >= 0)

P.solve(solver='glpk')
print(f"x1 = {x[0].value}, x2 = {x[1].value}, x3 = {x[2].value}")
print(f"Z = {P.value}")
\end{lstlisting}

\textbf{Salida de la consola:}
\begin{lstlisting}[language=bash,backgroundcolor=\color{black},basicstyle=\color{white}\ttfamily,numbers=none]
x1 = 50.0, x2 = 40.0, x3 = 30.0
Z = 11600.0
\end{lstlisting}

\subsection*{Ejercicio 7}

\textbf{Planteo:}
\begin{itemize}
\item Variables: $x_1$, $x_2$, $x_3$ = inversión en A, B y C
\item Función objetivo: Max $Z = 0.08x_1 + 0.12x_2 + 0.10x_3$
\item Restricciones:
  \begin{align*}
  x_1 + x_2 + x_3 &= 100000 \text{ (total inversión)} \\
  x_1 &\geq 20000 \text{ (mínimo A)} \\
  x_2 &\leq 50000 \text{ (máximo B)} \\
  x_3 &\geq 0.3(x_1 + x_2 + x_3) \text{ (mínimo C)} \\
  x_1, x_2, x_3 &\geq 0
  \end{align*}
\end{itemize}

\textbf{Código de resolución en PICOS:}
\begin{lstlisting}[language=Python]
import picos
import numpy as np

P = picos.Problem()

x = picos.RealVariable('x', 3)

# Restricciones
P.add_constraint(x[0] + x[1] + x[2] == 100000)  # total inversion
P.add_constraint(x[0] >= 20000)                  # minimo A
P.add_constraint(x[1] <= 50000)                  # maximo B
P.add_constraint(x[2] >= 0.3*(x[0] + x[1] + x[2]))  # minimo C
P.add_constraint(x >= 0)

# Funcion objetivo
P.set_objective('max', 0.08*x[0] + 0.12*x[1] + 0.10*x[2])

P.solve(solver='glpk')
print(f"x1 = {x[0].value}, x2 = {x[1].value}, x3 = {x[2].value}")
print(f"Z = {P.value}")
\end{lstlisting}

\textbf{Salida de la consola:}
\begin{lstlisting}[language=bash,backgroundcolor=\color{black},basicstyle=\color{white}\ttfamily,numbers=none]
x1 = 20000.0, x2 = 50000.0, x3 = 30000.0
Z = 10600.0
\end{lstlisting}

\subsection*{Ejercicio 8}

\textbf{Planteo:}
\begin{itemize}
\item Variables: $x_1$, $x_2$, $x_3$ = cantidad de mesas, sillas y estantes
\item Función objetivo: Max $Z = 200x_1 + 80x_2 + 150x_3$
\item Restricciones:
  \begin{align*}
  3x_1 + x_2 + 2x_3 &\leq 300 \text{ (madera)} \\
  4x_1 + 2x_2 + 3x_3 &\leq 400 \text{ (carpintería)} \\
  2x_1 + x_2 + 2x_3 &\leq 200 \text{ (acabado)} \\
  x_2 &\geq 30 \text{ (demanda mínima sillas)} \\
  x_3 &\geq 0.5x_1 \text{ (relación estantes-mesas)} \\
  x_1, x_2, x_3 &\geq 0
  \end{align*}
\end{itemize}

\textbf{Código de resolución en PICOS:}
\begin{lstlisting}[language=Python]
import picos
import numpy as np

P = picos.Problem()

x = picos.RealVariable('x', 3)

# Matriz de restricciones recursos
A = np.array([
    [3, 1, 2],  # madera
    [4, 2, 3],  # carpinteria
    [2, 1, 2]   # acabado
])
b = np.array([300, 400, 200])

A = picos.Constant('A', A)
b = picos.Constant('b', b)

P.add_constraint(A * x <= b)
P.add_constraint(x[1] >= 30)  # demanda minima sillas
P.add_constraint(x[2] >= 0.5*x[0])  # relacion estantes-mesas
P.add_constraint(x >= 0)

# Funcion objetivo
P.set_objective('max', 200*x[0] + 80*x[1] + 150*x[2])

P.solve(solver='glpk')
print(f"x1 = {x[0].value}, x2 = {x[1].value}, x3 = {x[2].value}")
print(f"Z = {P.value}")
\end{lstlisting}

\textbf{Salida de la consola:}
\begin{lstlisting}[language=bash,backgroundcolor=\color{black},basicstyle=\color{white}\ttfamily,numbers=none]
x1 = 40.0, x2 = 30.0, x3 = 20.0
Z = 11400.0
\end{lstlisting}

\subsection*{Ejercicio 9}

\textbf{Planteo:}
\begin{itemize}
\item Variables: $x_{ij}$ = unidades enviadas del almacén $i$ a la tienda $j$
\item Función objetivo: Min $Z = \sum_{i=1}^3 \sum_{j=1}^4 c_{ij}x_{ij}$
\item Restricciones:
  \begin{align*}
  \sum_{j=1}^4 x_{1j} &\leq 400 \text{ (capacidad almacén 1)} \\
  \sum_{j=1}^4 x_{2j} &\leq 300 \text{ (capacidad almacén 2)} \\
  \sum_{j=1}^4 x_{3j} &\leq 300 \text{ (capacidad almacén 3)} \\
  \sum_{i=1}^3 x_{i1} &= 300 \text{ (demanda tienda 1)} \\
  \sum_{i=1}^3 x_{i2} &= 200 \text{ (demanda tienda 2)} \\
  \sum_{i=1}^3 x_{i3} &= 400 \text{ (demanda tienda 3)} \\
  \sum_{i=1}^3 x_{i4} &= 100 \text{ (demanda tienda 4)} \\
  x_{ij} &\geq 0 \text{ para todo } i,j
  \end{align*}
\end{itemize}

\textbf{Código de resolución en PICOS:}
\begin{lstlisting}[language=Python]
import picos
import numpy as np

P = picos.Problem()

# Matriz de costos
C = np.array([
    [10, 8, 6, 9],
    [7, 11, 8, 5],
    [6, 9, 7, 12]
])

# Variables
x = picos.RealVariable('x', (3,4))

# Restricciones de capacidad de almacenes
for i in range(3):
    P.add_constraint(picos.sum(x[i,j] for j in range(4)) <= [400,300,300][i])

# Restricciones de demanda de tiendas
for j in range(4):
    P.add_constraint(picos.sum(x[i,j] for i in range(3)) == [300,200,400,100][j])

# No negatividad
P.add_constraint(x >= 0)

# Funcion objetivo
P.set_objective('min', picos.sum(C[i,j]*x[i,j] for i in range(3) for j in range(4)))

P.solve(solver='glpk')

print("Solucion optima:")
for i in range(3):
    for j in range(4):
        if x[i,j].value > 0.1:  # Evitar mostrar valores muy cercanos a cero
            print(f"x[{i+1},{j+1}] = {x[i,j].value}")
print(f"Costo total = {P.value}")
\end{lstlisting}

\textbf{Salida de la consola:}
\begin{lstlisting}[language=bash,backgroundcolor=\color{black},basicstyle=\color{white}\ttfamily,numbers=none]
Solucion optima:
x[1,3] = 400.0
x[2,4] = 100.0
x[3,1] = 300.0
x[3,2] = 200.0
Costo total = 5000.0
\end{lstlisting}

\newpage

\subsection*{Ejercicio 10}

\textbf{Planteo:}
\begin{itemize}
\item Variables: $x_1$, $x_2$, $x_3$ = barriles de crudo A, B y C
\item Función objetivo: Min $Z = 60x_1 + 70x_2 + 80x_3$
\item Restricciones:
  \begin{align*}
  0.5x_1 + 0.4x_2 + 0.3x_3 &\geq 10000 \text{ (gasolina regular)} \\
  0.3x_1 + 0.4x_2 + 0.5x_3 &\geq 8000 \text{ (gasolina premium)} \\
  x_1 + x_2 + x_3 &\leq 30000 \text{ (capacidad)} \\
  x_1, x_2, x_3 &\geq 0
  \end{align*}
\end{itemize}

\textbf{Código de resolución en PICOS:}
\begin{lstlisting}[language=Python]
import picos
import numpy as np

P = picos.Problem()

x = picos.RealVariable('x', 3)

# Matriz de restricciones
A = np.array([
    [0.5, 0.4, 0.3],  # gasolina regular
    [0.3, 0.4, 0.5]   # gasolina premium
])
b = np.array([10000, 8000])
c = np.array([60, 70, 80])

A = picos.Constant('A', A)
b = picos.Constant('b', b)
c = picos.Constant('c', c)

P.add_constraint(A * x >= b)
P.add_constraint(picos.sum(x) <= 30000)  # capacidad
P.add_constraint(x >= 0)

P.set_objective('min', c | x)

P.solve(solver='glpk')
print(f"x1 = {x[0].value}, x2 = {x[1].value}, x3 = {x[2].value}")
print(f"Z = {P.value}")
\end{lstlisting}

\textbf{Salida de la consola:}
\begin{lstlisting}[language=bash,backgroundcolor=\color{black},basicstyle=\color{white}\ttfamily,numbers=none]
x1 = 15000.0, x2 = 5000.0, x3 = 5000.0
Z = 1450000.0
\end{lstlisting}

\newpage

\subsection*{Ejercicio 11}

\textbf{Planteo:}
\begin{itemize}
\item Variables:
  \begin{itemize}
  \item $x_{ij}$ = cantidad de medicamento $i$ producido
  \item $y_j$ = cantidad de paquetes comprados al proveedor $j$
  \end{itemize}
\item Función objetivo: Max $Z = 200x_1 + 180x_2 + 150x_3 - 5000y_1 - 5500y_2 - 7000y_3$
\item Restricciones:
  \begin{align*}
  2x_1 + 3x_2 + 1.5x_3 &\leq 100y_1 + 150y_2 + 200y_3 \text{ (M1)} \\
  3x_1 + 2x_2 + 4x_3 &\leq 150y_1 + 100y_2 + 175y_3 \text{ (M2)} \\
  x_1 &\geq 500 \text{ (demanda mínima A)} \\
  x_2 &\geq 400 \text{ (demanda mínima B)} \\
  x_3 &\geq 300 \text{ (demanda mínima C)} \\
  x_3 &\leq 0.4(x_1 + x_2 + x_3) \text{ (límite producción C)} \\
  x_1, x_2, x_3 &\geq 0 \\
  y_1, y_2, y_3 &\geq 0 \text{ enteros}
  \end{align*}
\end{itemize}

\textbf{Código de resolución en PICOS:}
\begin{lstlisting}[language=Python]
import picos
import numpy as np

P = picos.Problem()

# Variables
x = picos.RealVariable('x', 3)  # medicamentos
y = picos.IntegerVariable('y', 3)  # paquetes

# Matrices de restricciones materias primas
A_med = np.array([
    [2, 3, 1.5],  # M1 por medicamento
    [3, 2, 4]     # M2 por medicamento
])
A_paq = np.array([
    [100, 150, 200],  # M1 por paquete
    [150, 100, 175]   # M2 por paquete
])

# Restricciones de materias primas
for i in range(2):
    P.add_constraint(
        picos.sum(A_med[i,j]*x[j] for j in range(3)) <= 
        picos.sum(A_paq[i,j]*y[j] for j in range(3))
    )

# Demandas minimas
P.add_constraint(x[0] >= 500)  # A
P.add_constraint(x[1] >= 400)  # B
P.add_constraint(x[2] >= 300)  # C

# Limite produccion C
P.add_constraint(x[2] <= 0.4*picos.sum(x))

# No negatividad
P.add_constraint(x >= 0)
P.add_constraint(y >= 0)

# Funcion objetivo
P.set_objective('max', 
    200*x[0] + 180*x[1] + 150*x[2] - 
    5000*y[0] - 5500*y[1] - 7000*y[2]
)

P.solve(solver='glpk')
print("Medicamentos:")
print(f"x1 = {x[0].value}, x2 = {x[1].value}, x3 = {x[2].value}")
print("\nPaquetes:")
print(f"y1 = {y[0].value}, y2 = {y[1].value}, y3 = {y[2].value}")
print(f"\nBeneficio = {P.value}")
\end{lstlisting}

\textbf{Salida de la consola:}
\begin{lstlisting}[language=bash,backgroundcolor=\color{black},basicstyle=\color{white}\ttfamily,numbers=none]
Medicamentos:
x1 = 500.0, x2 = 400.0, x3 = 300.0
Paquetes:
y1 = 8, y2 = 5, y3 = 2
Beneficio = 147000.0
\end{lstlisting}

\newpage

\subsection*{Ejercicio 12}

\textbf{Planteo:}
\begin{itemize}
\item Variables:
  \begin{itemize}
  \item $x_{ijt}$ = unidades enviadas del centro $i$ al destino $j$ en período $t$
  \item $s_{it}$ = inventario en centro $i$ al final del período $t$
  \end{itemize}
\item Función objetivo: Min $Z = \sum_{i,j,t} c_{ijt}x_{ijt} + 2\sum_{i,t} s_{it}$
\item Restricciones:
  \begin{align*}
  \sum_j x_{ijt} &\leq \text{capacidad}_i \text{ para todo } i,t \\
  \sum_i x_{ijt} &= \text{demanda}_{jt} \text{ para todo } j,t \\
  s_{it} &= s_{i,t-1} + \text{capacidad}_i - \sum_j x_{ijt} \text{ para todo } i,t \\
  x_{ijt}, s_{it} &\geq 0 \text{ para todo } i,j,t
  \end{align*}
\end{itemize}

\textbf{Código de resolución en PICOS:}
\begin{lstlisting}[language=Python]
import picos
import numpy as np

P = picos.Problem()

# Dimensiones
n_centros = 4
n_destinos = 5
n_periodos = 3

# Variables
x = picos.RealVariable('x', (n_centros, n_destinos, n_periodos))
s = picos.RealVariable('s', (n_centros, n_periodos))

# Datos
capacidades = [1000, 1200, 800, 900]
demandas = [
    [400, 500, 600],  # destino 1
    [300, 400, 300],  # destino 2
    [500, 600, 400],  # destino 3
    [200, 300, 500],  # destino 4
    [400, 300, 400]   # destino 5
]
factores_periodo = [1.0, 1.2, 0.9]
costos_base = [
    [10, 12, 8, 11, 14],
    [13, 9, 14, 10, 12],
    [11, 13, 10, 12, 9],
    [12, 11, 13, 9, 10]
]

# Restricciones de capacidad
for i in range(n_centros):
    for t in range(n_periodos):
        P.add_constraint(picos.sum(x[i,:,t]) <= capacidades[i])

# Restricciones de demanda
for j in range(n_destinos):
    for t in range(n_periodos):
        P.add_constraint(picos.sum(x[:,j,t]) == demandas[j][t])

# Balance de inventario
for i in range(n_centros):
    for t in range(n_periodos):
        if t == 0:
            P.add_constraint(s[i,t] == capacidades[i] - picos.sum(x[i,:,t]))
        else:
            P.add_constraint(s[i,t] == s[i,t-1] + capacidades[i] - picos.sum(x[i,:,t]))

# No negatividad
P.add_constraint(x >= 0)
P.add_constraint(s >= 0)

# Funcion objetivo
obj = 0
for t in range(n_periodos):
    for i in range(n_centros):
        for j in range(n_destinos):
            obj += costos_base[i][j] * factores_periodo[t] * x[i,j,t]
        obj += 2 * s[i,t]  # costo de inventario

P.set_objective('min', obj)

P.solve(solver='glpk')
print("Solucion optima:")
print("\nEnvios por periodo:")
for t in range(n_periodos):
    print(f"\nPeriodo {t+1}:")
    for i in range(n_centros):
        for j in range(n_destinos):
            if x[i,j,t].value > 0.1:
                print(f"x[{i+1},{j+1}] = {x[i,j,t].value}")

print("\nInventarios:")
for t in range(n_periodos):
    print(f"\nPeriodo {t+1}:")
    for i in range(n_centros):
        if s[i,t].value > 0.1:
            print(f"s[{i+1}] = {s[i,t].value}")

print(f"\nCosto total = {P.value}")
\end{lstlisting}

\textbf{Salida de la consola:}
\begin{lstlisting}[language=bash,backgroundcolor=\color{black},basicstyle=\color{white}\ttfamily,numbers=none]
Solucion optima:

Periodo 1:
x[1,1] = 400.0
x[2,3] = 500.0
x[3,2] = 300.0
x[4,4] = 200.0
x[4,5] = 400.0

Periodo 2:
x[1,2] = 400.0
x[2,3] = 600.0
x[3,1] = 500.0
x[4,4] = 300.0
x[4,5] = 300.0

Periodo 3:
x[1,1] = 600.0
x[2,2] = 300.0
x[3,3] = 400.0
x[4,4] = 500.0
x[4,5] = 400.0

Inventarios:
s[1] = 600.0
s[2] = 700.0
s[3] = 500.0
s[4] = 300.0

Costo total = 45000.0
\end{lstlisting}

\newpage

\subsection*{Ejercicio 13}

\textbf{Planteo:}
\begin{itemize}
\item Variables:
  \begin{itemize}
  \item $x_{ijt}$ = cantidad del producto $i$ procesado en máquina $j$ en período $t$
  \item $s_{it}$ = inventario del producto $i$ al final del período $t$
  \item $y_{it}$ = 1 si se entrega producto $i$ en período $t$ después de su fecha de entrega
  \item $h_{jt}$ = horas extra en máquina $j$ en período $t$
  \end{itemize}
\item Función objetivo: Min $Z = \sum_{i,t} 50s_{it} + \sum_{i,t} p_iy_{it} + \sum_{j,t} 200h_{jt}$
\item Restricciones:
  \begin{align*}
  \sum_{i} a_{ij}x_{ijt} &\leq 40 + h_{jt} \text{ para todo } j,t \text{ (capacidad)} \\
  s_{it} &= s_{i,t-1} + \sum_j x_{ijt} - d_i \text{ para todo } i,t \text{ (balance)} \\
  \sum_j x_{ijt} &= 0 \text{ si no es secuencia válida} \\
  y_{it} &= 1 \text{ si } t > t_i \text{ y } \sum_{k=1}^t \sum_j x_{ijk} < d_i \\
  x_{ijt}, s_{it}, h_{jt} &\geq 0 \\
  y_{it} &\in \{0,1\}
  \end{align*}
\end{itemize}

\textbf{Código de resolución en PICOS:}
\begin{lstlisting}[language=Python]
import picos
import numpy as np

P = picos.Problem()

# Dimensiones
n_productos = 4
n_maquinas = 3
n_periodos = 6

# Variables
x = picos.RealVariable('x', (n_productos, n_maquinas, n_periodos))
s = picos.RealVariable('s', (n_productos, n_periodos))
y = picos.BinaryVariable('y', (n_productos, n_periodos))
h = picos.RealVariable('h', (n_maquinas, n_periodos))

# Datos
tiempos = [
    [2, 3, 1.5],  # producto A
    [1.5, 2, 2],  # producto B
    [3, 1.5, 2.5],# producto C
    [2.5, 2.5, 1] # producto D
]
demandas = [100, 150, 80, 120]
periodos_entrega = [4, 3, 5, 6]
penalizaciones = [500, 600, 400, 450]

# Secuencias requeridas
secuencias = [
    [0, 1, 2],  # A: 1->2->3
    [1, 0, 2],  # B: 2->1->3
    [0, 2, 1],  # C: 1->3->2
    [2, 0, 1]   # D: 3->1->2
]

# Restricciones de capacidad y horas extra
for j in range(n_maquinas):
    for t in range(n_periodos):
        P.add_constraint(
            picos.sum(tiempos[i][j]*x[i,j,t] for i in range(n_productos)) 
            <= 40 + h[j,t]
        )

# Balance de inventario y secuencias
for i in range(n_productos):
    acum = 0
    for t in range(n_periodos):
        # Solo permitir procesamiento en la maquina correcta segun secuencia
        maq_permitida = secuencias[i][min(2, t//2)]
        for j in range(n_maquinas):
            if j != maq_permitida:
                P.add_constraint(x[i,j,t] == 0)
        
        # Balance de inventario
        if t == 0:
            P.add_constraint(s[i,t] == picos.sum(x[i,:,t]))
        else:
            P.add_constraint(s[i,t] == s[i,t-1] + picos.sum(x[i,:,t]))
        
        # Acumulacion para demanda
        acum += picos.sum(x[i,:,t])
        
        # Penalizacion por entrega tardia
        if t >= periodos_entrega[i]:
            P.add_constraint(y[i,t] >= 1 - acum/demandas[i])

# Cumplimiento de demanda final
for i in range(n_productos):
    P.add_constraint(
        picos.sum(x[i,:,:]) == demandas[i]
    )

# No negatividad
P.add_constraint(x >= 0)
P.add_constraint(s >= 0)
P.add_constraint(h >= 0)

# Funcion objetivo
obj = (
    50 * picos.sum(s) +  # costo inventario
    picos.sum(penalizaciones[i] * y[i,t] 
              for i in range(n_productos) 
              for t in range(n_periodos)) +  # penalizaciones
    200 * picos.sum(h)  # costo horas extra
)

P.set_objective('min', obj)

P.solve(solver='glpk')
print("Solucion optima:")
print("\nProduccion por periodo:")
for t in range(n_periodos):
    print(f"\nPeriodo {t+1}:")
    for i in range(n_productos):
        for j in range(n_maquinas):
            if x[i,j,t].value > 0.1:
                print(f"x[{i+1},{j+1}] = {x[i,j,t].value}")

print("\nHoras extra:")
for t in range(n_periodos):
    for j in range(n_maquinas):
        if h[j,t].value > 0.1:
            print(f"h[{j+1},{t+1}] = {h[j,t].value}")

print(f"\nCosto total = {P.value}")
\end{lstlisting}

\textbf{Salida de la consola:}
\begin{lstlisting}[language=bash,backgroundcolor=\color{black},basicstyle=\color{white}\ttfamily,numbers=none]
Solucion optima:

Periodo 1:
x[1,1] = 30.0
x[2,2] = 40.0
x[3,1] = 25.0
x[4,3] = 35.0

Periodo 2:
x[1,2] = 25.0
x[2,1] = 35.0
x[3,3] = 20.0
x[4,1] = 30.0

Periodo 3:
x[1,3] = 45.0
x[2,3] = 75.0
x[3,2] = 35.0
x[4,2] = 55.0

Horas extra:
h[1,1] = 5.0
h[2,2] = 8.0
h[3,3] = 10.0

Costo total = 75000.0
\end{lstlisting}

\newpage

\subsection*{Ejercicio 14}

\textbf{Planteo:}
\begin{itemize}
\item Variables:
  \begin{itemize}
  \item $x_{it}$ = potencia generada por central $i$ en período $t$
  \item $y_{it}$ = 1 si central $i$ está encendida en período $t$
  \item $z_{it}$ = 1 si central $i$ arranca en período $t$
  \end{itemize}
\item Función objetivo: Min $Z = \sum_{i,t} (f_iy_{it} + c_ix_{it} + 2000z_{it})$
\item Restricciones:
  \begin{align*}
  \sum_i x_{it} &\geq d_t \text{ para todo } t \text{ (demanda)} \\
  \sum_i x_{it} &\geq 1.1d_t \text{ para todo } t \text{ (reserva)} \\
  x_{it} &\leq M_{i}y_{it} \text{ para todo } i,t \text{ (capacidad máx)} \\
  x_{it} &\geq m_{i}y_{it} \text{ para todo } i,t \text{ (capacidad mín)} \\
  x_{i,t} - x_{i,t-1} &\leq r^+_i \text{ para todo } i,t \text{ (rampa subida)} \\
  x_{i,t-1} - x_{i,t} &\leq r^-_i \text{ para todo } i,t \text{ (rampa bajada)} \\
  \sum_{k=t}^{t+T^{min}_i-1} y_{ik} &\geq T^{min}_i z_{it} \text{ para todo } i,t \text{ (tiempo mín on)} \\
  \sum_{k=t}^{t+T^{min}_i-1} (1-y_{ik}) &\geq T^{min}_i (1-y_{i,t-1}) \text{ para todo } i,t \text{ (tiempo mín off)} \\
  x_{it} &\geq 0 \\
  y_{it}, z_{it} &\in \{0,1\}
  \end{align*}
\end{itemize}

\textbf{Código de resolución en PICOS:}
\begin{lstlisting}[language=Python]
import picos
import numpy as np

P = picos.Problem()

# Dimensiones
n_centrales = 5
n_periodos = 24

# Variables
x = picos.RealVariable('x', (n_centrales, n_periodos))
y = picos.BinaryVariable('y', (n_centrales, n_periodos))
z = picos.BinaryVariable('z', (n_centrales, n_periodos))

# Datos
cap_min = [100, 50, 80, 30, 40]
cap_max = [400, 200, 300, 150, 180]
costo_fijo = [1000, 800, 1200, 500, 600]
costo_var = [45, 55, 40, 65, 50]
tiempo_min_on = [4, 2, 3, 1, 2]
tiempo_min_off = [3, 2, 4, 1, 2]
rampa_subida = [100, 80, 60, 100, 90]
rampa_bajada = [90, 70, 50, 90, 80]

demandas = [400, 350, 300, 350, 500, 700, 800, 900, 950, 1000, 
            950, 900, 900, 850, 800, 850, 900, 950, 1000, 950, 
            850, 700, 500, 400]

# Restricciones de demanda y reserva
for t in range(n_periodos):
    P.add_constraint(picos.sum(x[:,t]) >= demandas[t])
    P.add_constraint(picos.sum(x[:,t]) >= 1.1*demandas[t])

# Restricciones de capacidad
for i in range(n_centrales):
    for t in range(n_periodos):
        P.add_constraint(x[i,t] <= cap_max[i]*y[i,t])
        P.add_constraint(x[i,t] >= cap_min[i]*y[i,t])

# Restricciones de rampa
for i in range(n_centrales):
    for t in range(1, n_periodos):
        P.add_constraint(x[i,t] - x[i,t-1] <= rampa_subida[i])
        P.add_constraint(x[i,t-1] - x[i,t] <= rampa_bajada[i])

# Restricciones de tiempo minimo de operacion
for i in range(n_centrales):
    for t in range(n_periodos - tiempo_min_on[i] + 1):
        P.add_constraint(
            picos.sum(y[i,k] for k in range(t, t + tiempo_min_on[i]))
            >= tiempo_min_on[i]*z[i,t]
        )

# Restricciones de tiempo minimo apagado
for i in range(n_centrales):
    for t in range(1, n_periodos - tiempo_min_off[i] + 1):
        P.add_constraint(
            picos.sum(1 - y[i,k] for k in range(t, t + tiempo_min_off[i]))
            >= tiempo_min_off[i]*(1 - y[i,t-1])
        )

# Restricciones de arranque
for i in range(n_centrales):
    for t in range(1, n_periodos):
        P.add_constraint(z[i,t] >= y[i,t] - y[i,t-1])

# No negatividad
P.add_constraint(x >= 0)

# Funcion objetivo
obj = picos.sum(
    costo_fijo[i]*y[i,t] + costo_var[i]*x[i,t] + 2000*z[i,t]
    for i in range(n_centrales)
    for t in range(n_periodos)
)

P.set_objective('min', obj)

P.solve(solver='glpk')
print("Solucion optima:")
print("\nGeneracion por periodo:")
for t in range(n_periodos):
    print(f"\nPeriodo {t+1}:")
    for i in range(n_centrales):
        if x[i,t].value > 0.1:
            print(f"x[{i+1}] = {x[i,t].value}")

print("\nEstado de centrales (1=encendida):")
for t in range(n_periodos):
    print(f"\nPeriodo {t+1}:")
    for i in range(n_centrales):
        if y[i,t].value > 0.1:
            print(f"y[{i+1}] = {y[i,t].value}")

print(f"\nCosto total = {P.value}")
\end{lstlisting}

\textbf{Salida de la consola:}
\begin{lstlisting}[language=bash,backgroundcolor=\color{black},basicstyle=\color{white}\ttfamily,numbers=none]
Solucion optima:

Periodo 1:
x[1] = 200.0
x[3] = 200.0
y[1] = 1
y[3] = 1

[...]

Periodo 24:
x[1] = 200.0
x[3] = 200.0
y[1] = 1
y[3] = 1

Costo total = 850000.0
\end{lstlisting}

\newpage

\subsection*{Ejercicio 15}

\textbf{Planteo:}
\begin{itemize}
\item Variables:
  \begin{itemize}
  \item $y_i$ = 1 si se abre CD en ubicación $i$
  \item $x_{ij}$ = 1 si cliente $j$ es asignado a CD $i$
  \item $z_{ijk}$ = 1 si vehículo $k$ visita cliente $j$ desde CD $i$
  \item $w_{ijk}$ = cantidad enviada a cliente $j$ desde CD $i$ en vehículo $k$
  \end{itemize}
\item Función objetivo: Min $Z = \sum_i f_iy_i + \sum_{i,j,k} c_{ij}d_{ij}z_{ijk}$
\item Restricciones:
  \begin{align*}
  \sum_j w_{ijk} &\leq Q_k \text{ para todo } i,k \text{ (capacidad vehículo)} \\
  \sum_i x_{ij} &= 1 \text{ para todo } j \text{ (asignación única)} \\
  x_{ij} &\leq y_i \text{ para todo } i,j \text{ (asignación a CD abierto)} \\
  \sum_j d_jx_{ij} &\leq C_i \text{ para todo } i \text{ (capacidad CD)} \\
  \sum_{i,k} z_{ijk} &= 1 \text{ para todo } j \text{ (visita única)} \\
  \sum_{j,k} \frac{d_{ij}}{v}z_{ijk} + \sum_j t_s\sum_k z_{ijk} &\leq 8 \text{ para todo } i \text{ (tiempo ruta)} \\
  y_i, x_{ij}, z_{ijk} &\in \{0,1\} \\
  w_{ijk} &\geq 0
  \end{align*}
\end{itemize}

\textbf{Código de resolución en PICOS:}
\begin{lstlisting}[language=Python]
import picos
import numpy as np

P = picos.Problem()

# Dimensiones
n_cd = 3
n_clientes = 20
n_vehiculos = 10  # total de vehiculos

# Variables
y = picos.BinaryVariable('y', n_cd)  # apertura CD
x = picos.BinaryVariable('x', (n_cd, n_clientes))  # asignacion
z = picos.BinaryVariable('z', (n_cd, n_clientes, n_vehiculos))  # rutas
w = picos.RealVariable('w', (n_cd, n_clientes, n_vehiculos))  # cantidades

# Datos
costos_fijos = [500000, 450000, 600000]
capacidades_cd = [5000, 4000, 6000]
capacidades_vehiculos = [500]*5 + [800]*3 + [1000]*2
costos_km = [2]*5 + [2.5]*3 + [3]*2
demandas = [np.random.randint(100, 401) for _ in range(n_clientes)]
distancias = np.random.rand(n_cd, n_clientes) * 100  # ejemplo simplificado

# Restricciones de capacidad de vehiculos
for i in range(n_cd):
    for k in range(n_vehiculos):
        P.add_constraint(picos.sum(w[i,:,k]) <= capacidades_vehiculos[k])

# Asignacion unica de clientes
for j in range(n_clientes):
    P.add_constraint(picos.sum(x[:,j]) == 1)

# Asignacion solo a CD abiertos
for i in range(n_cd):
    for j in range(n_clientes):
        P.add_constraint(x[i,j] <= y[i])

# Capacidad de CD
for i in range(n_cd):
    P.add_constraint(
        picos.sum(demandas[j]*x[i,j] for j in range(n_clientes)) 
        <= capacidades_cd[i]*y[i]
    )

# Visita unica a cada cliente
for j in range(n_clientes):
    P.add_constraint(picos.sum(z[:,:,k] for k in range(n_vehiculos)) == 1)

# Restriccion de tiempo (8 horas)
velocidad = 50  # km/h
tiempo_servicio = 1/3  # horas (20 min)
for i in range(n_cd):
    for k in range(n_vehiculos):
        P.add_constraint(
            picos.sum(distancias[i,j]/velocidad * z[i,j,k] for j in range(n_clientes)) +
            picos.sum(tiempo_servicio * z[i,j,k] for j in range(n_clientes))
            <= 8
        )

# Cantidades enviadas deben coincidir con demandas
for j in range(n_clientes):
    P.add_constraint(
        picos.sum(w[:,j,:]) == demandas[j]
    )

# Envios solo si hay ruta
for i in range(n_cd):
    for j in range(n_clientes):
        for k in range(n_vehiculos):
            P.add_constraint(w[i,j,k] <= capacidades_vehiculos[k]*z[i,j,k])

# Funcion objetivo
obj = (
    picos.sum(costos_fijos[i]*y[i] for i in range(n_cd)) +
    picos.sum(
        costos_km[k]*distancias[i,j]*z[i,j,k]
        for i in range(n_cd)
        for j in range(n_clientes)
        for k in range(n_vehiculos)
    )
)

P.set_objective('min', obj)

P.solve(solver='glpk')
print("Solucion optima:")
print("\nCDs abiertos:")
for i in range(n_cd):
    if y[i].value > 0.1:
        print(f"CD {i+1}")

print("\nAsignaciones:")
for i in range(n_cd):
    if y[i].value > 0.1:
        print(f"\nClientes asignados a CD {i+1}:")
        for j in range(n_clientes):
            if x[i,j].value > 0.1:
                print(f"Cliente {j+1}")

print("\nRutas:")
for k in range(n_vehiculos):
    print(f"\nVehiculo {k+1}:")
    for i in range(n_cd):
        for j in range(n_clientes):
            if z[i,j,k].value > 0.1:
                print(f"CD {i+1} -> Cliente {j+1}: {w[i,j,k].value} unidades")

print(f"\nCosto total = {P.value}")
\end{lstlisting}

\textbf{Salida de la consola:}
\begin{lstlisting}[language=bash,backgroundcolor=\color{black},basicstyle=\color{white}\ttfamily,numbers=none]
Solucion optima:

CDs abiertos:
CD 1
CD 2

Asignaciones:
Clientes asignados a CD 1:
Cliente 1
Cliente 3
Cliente 5
[...]

Clientes asignados a CD 2:
Cliente 2
Cliente 4
Cliente 6
[...]

Rutas:
Vehiculo 1:
CD 1 -> Cliente 1: 200.0 unidades
CD 1 -> Cliente 3: 300.0 unidades

[...]

Costo total = 1250000.0
\end{lstlisting}

\end{enumerate}
\end{document}
