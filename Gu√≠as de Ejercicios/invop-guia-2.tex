\documentclass{article}
\usepackage[utf8]{inputenc}
\usepackage{xcolor}               % Cargar xcolor antes de babel y ulem
\usepackage[spanish]{babel}
\usepackage{amsmath, amssymb}
\usepackage[normalem]{ulem}       % Evita que ulem modifique \emph y otros
\usepackage{booktabs}
\usepackage{array}
\usepackage{graphicx}
\usepackage{multirow}
\usepackage{tikz}
\usepackage{pstricks}
\usepackage{pst-node} % Para \circlenode y conexiones
\usepackage{asymptote}
\usetikzlibrary{arrows, positioning}

\begin{document}

\begin{center}
    \underline{\textbf{\large Guía de Ejercicios 2}} \\[5pt]
    \underline{\textbf{\large Programación Lineal}}
\end{center}

\section*{Desestructurado}

\begin{enumerate}
    \item Una empresa tiene sólo tres empleados (Doug, Linda y Bob) que hacen dos tipos de ventanas a mano: con marco de madera y con marco de aluminio. La ganancia es de \$180 por cada ventana con marco de madera y de \$90 por cada una con marco de aluminio. Doug hace marcos de madera y puede terminar 6 al día. Linda hace 4 marcos de aluminio por día. Bob forma y corta el vidrio y puede hacer 48 pies cuadrados de vidrio por día. Cada ventana con marco de madera emplea 6 pies cuadrados de vidrio y cada una de aluminio, 8 pies cuadrados.  
    ¿Cuántas ventanas de cada tipo debe producir al día para maximizar la ganancia total?  

    \textbf{NOTA:} siendo este un TP de Programación Lineal, en este caso no es necesario que las variables de decisión sean enteras (es decir, pueden ser variables de decisión reales), a pesar de que en la realidad no tiene sentido fabricar una cantidad fraccional de ventanas.
\end{enumerate}

\section*{Transporte}

\begin{enumerate}
    \setcounter{enumi}{1} % Continuamos numeración (se inicia en 2)
    \item En un problema de transporte con 5 nodos de origen y 6 de destino, el costo de transporte y los requisitos de la demanda y oferta están resumidos en la siguiente tabla.  
    Determinar si el problema es factible (o sea, si la oferta puede suplir a la demanda), y de ser así determinar la distribución de transporte óptima.  
    Para facilitar la resolución recomendamos dibujar el gráfico de transporte con las 5 fuentes y los 6 destinos.  
    En la tabla \(M = 1000\) (o sea, un número muy grande).  

    \textbf{NOTA:} utilizar una matriz de decisión, en lugar de un vector de decisión, puede hacer más sencillo el código de Python resultante.

    \begin{table}[h]
        \centering
        \renewcommand{\arraystretch}{1.3} % Espaciado vertical
        \begin{tabular}{c|cccccc|c}
            \hline
            & \multicolumn{6}{c|}{\textbf{Destination}} & \textbf{Supply} \\
            \hline
            & \textbf{1} & \textbf{2} & \textbf{3} & \textbf{4} & \textbf{5} & \textbf{6} &  \\
            \hline
            \textbf{Source} & & & & & & & \\
            1 & 13 & 10 & 22 & 29 & 18 & 0 & 5 \\
            2 & 14 & 13 & 16 & 21 & $M$ & 0 & 6 \\
            3 & 3 & 0 & $M$ & 11 & 6 & 0 & 7 \\
            4 & 18 & 9 & 19 & 23 & 11 & 0 & 4 \\
            5 & 30 & 24 & 34 & 36 & 28 & 0 & 3 \\
            \hline
            \textbf{Demand} & 3 & 5 & 4 & 5 & 6 & 2 & \\
            \hline
        \end{tabular}
    \end{table}
    
    \item Una empresa constructora debe llevar cemento a tres sitios de construcción. Tiene dos proveedores de cemento, uno al Norte y otro al Sur, que le venden cemento a distintos valores por tonelada, y además el costo de transporte de cada uno de estos proveedores a los sitios de construcción es distinto. Puede comprar hasta 18 toneladas a una cantera ubicada al Norte de la ciudad y 14 toneladas a una del Sur.  
    Necesita 10, 5 y 10 toneladas en las respectivas construcciones 1, 2 y 3.  
    Los costos de compra y transporte se resumen en la siguiente tabla.  

    Formular el problema como uno de transporte, y encontrar la estrategia de compra y transporte óptima, que minimice el costo total (costo de transporte + costo de compra).

    \begin{table}[h]
        \centering
        \renewcommand{\arraystretch}{1.3} 
        \setlength{\tabcolsep}{10pt}
        \begin{tabular}{c|>{\centering}p{1.5cm} >{\centering}p{1.5cm} >{\centering}p{1.5cm}|c}
            \toprule
            & \multicolumn{3}{c|}{\textbf{Hauling Cost per Ton at Site}} & \textbf{Price per Ton} \\
            \cmidrule(lr){2-4} 
            \textbf{Pit} & \textbf{1} & \textbf{2} & \textbf{3} & \\
            \midrule
            North & \$100 & \$190 & \$160 & \$300 \\
            South & 180 & 110 & 140 & 420 \\
            \bottomrule
        \end{tabular}
    \end{table}
    
    \item Una empresa ha decidido producir tres nuevos productos. Tiene cinco plantas de producción con capacidad ociosa, donde quiere producir estos nuevos productos.  
    El costo unitario de producción del producto 1 es \$31, \$29, \$32, \$28 y \$20 en las plantas 1, 2, 3, 4 y 5 respectivamente.  
    El costo unitario de producción del producto 2 es \$45, \$41, \$46, \$42 y \$43 en las plantas 1, 2, 3, 4 y 5 respectivamente.  
    El costo unitario de producción del producto 3 es \$38, \$35, \$40 en las plantas 1, 2 y 3 respectivamente, pero no es posible producir este producto en las plantas 4 y 5 por falta de entrenamiento del personal.  

    El estudio de mercado indica que se tendrán que producir 600, 1000 y 800 unidades por día de los productos 1, 2 y 3 respectivamente.  
    Las plantas tienen una capacidad de producción de hasta 400, 600, 400, 600 y 1000 unidades por día, independientemente de qué producto sea.

    \begin{enumerate}
        \item ¿Cuál debe ser la estrategia de producción si se quiere cumplir con los pedidos de producción, pero minimizando el costo total de producción?
        \item Supongamos que la demanda proyectada fuera de 1000, 1500 y 900 por día de los productos 1, 2 y 3 respectivamente, y que la pérdida por demanda insatisfecha es de \$150, \$200 y \$300 por unidad de los productos 1, 2 y 3 respectivamente.  
        ¿Cuál es la estrategia de producción óptima?  
        ¿Queda algún producto con demanda insatisfecha?  
        De ser así, ¿cuál(es) y cuánta es la demanda insatisfecha?
    \end{enumerate}
\end{enumerate}

\section*{Transhipment}

\begin{enumerate}
    \setcounter{enumi}{4} % Se continúa la numeración (item 5 en adelante)
    \item Una empresa produce un producto, y debe producir suficiente para satisfacer los contratos de compra-venta firmados para los próximos tres meses.  
    Las capacidades de producción, costos de producción y costos de almacenaje varían mes a mes.  
    Debido a esto puede ser beneficioso sobreproducir en ciertos meses, almacenar unidades y venderlas en futuros meses.  
    La planta puede producir una cierta cantidad durante horas regulares, o de ser necesario puede producir otro tanto en horas extra, a un costo mayor.  

    \textbf{Objetivo:} determinar el plan de producción óptimo.

    \begin{table}[h]
        \centering
        \resizebox{\textwidth}{!}{%
        \renewcommand{\arraystretch}{1.3}%
        \setlength{\tabcolsep}{8pt} 
        \begin{tabular}{c|cc|cc|c|c}
            \toprule
            \textbf{Mes} & \multicolumn{2}{c|}{\textbf{Capacidad de Producción (unidades/mes)}} & \multicolumn{2}{c|}{\textbf{Costo de Producción (\$/unidad)}} & \textbf{Costo de Almacenaje (\$/unidad)} & \textbf{Ventas Comprometidas (unidades)} \\
            \cmidrule(lr){2-3} \cmidrule(lr){4-5}
            & Horas Regulares & Horas Extra & Horas Regulares & Horas Extra & & \\
            \midrule
            1 & 10 & 3 & 31 & 38 & 3 & 8 \\
            2 & 8 & 2 & 32 & 38 & 3 & 10 \\
            3 & 10 & 3 & 36 & 44 & 3 & 16 \\
            \bottomrule
        \end{tabular}%
        }
    \end{table}

    \item Una empresa produce un producto en dos plantas y lo vende en tres locales de venta. Luego de producirlos, los productos son enviados a uno de sus dos warehouses hasta que sean requeridos por los locales de venta.  

    Se usan camiones para transportar los productos de sus dos plantas de producción a los warehouses, y de allí a uno de sus tres locales de venta.  
    La siguiente tabla muestra la capacidad de producción de cada una de sus plantas, los costos de transporte a cada uno de los warehouses, y la cantidad máxima que se puede transportar a cada uno de los warehouses.

    \textbf{Objetivo:} Plantear el problema como un problema de transhipment y encontrar el cronograma de transporte óptimo.

    \begin{table}[h]
        \centering
        \resizebox{\textwidth}{!}{%
        \renewcommand{\arraystretch}{1.3}%
        \begin{tabular}{c|c|cc|cc}
            \hline
            & \textbf{Capacidad de Producción} & \multicolumn{2}{c|}{\textbf{Costo unitario de transporte}} & \multicolumn{2}{c}{\textbf{Capacidad de transporte}} \\
            \hline
            & & \textbf{Warehouse 1} & \textbf{Warehouse 2} & \textbf{Warehouse 1} & \textbf{Warehouse 2} \\
            \hline
            \textbf{Planta 1} & 200 & 425 & 560 & 125 & 150 \\
            \textbf{Planta 2} & 300 & 510 & 600 & 175 & 200 \\
            \hline
        \end{tabular}%
        }
    \end{table}

    \item Considere la siguiente red de distribución de productos, en donde \(A\) es el nodo origen y \(F\) el nodo de demanda, mientras que las capacidades de cada ruta son los números que se muestran junto a los arcos dirigidos.

    \begin{enumerate}
        \item ¿Cuál es la máxima cantidad de productos que se pueden transportar del Nodo \(A\) al Nodo \(F\) a través de esta red de distribución?
        \item La empresa quiere entender cómo cambia la capacidad de transporte de la red si se incrementa la capacidad de transporte del vínculo \(B-D\).  
        Realizar un barrido paramétrico del parámetro 
        \[
        u_{BD} = [0,1,2,3,4,5,6,7,8,9,10,11]
        \]
        y graficar cómo cambia la capacidad de transporte de la red versus este parámetro.
    \end{enumerate}

    \begin{center}
        \begin{asy}
            size(10cm); // Tamaño de la figura
            
            pair A=(0,1), B=(2,2), C=(2,0), D=(4,2), E=(4,0), F=(6,2);
            
            // Dibujamos círculos para cada nodo y les ponemos etiqueta
            draw(Circle(A,0.2)); label("A", A, S);
            draw(Circle(B,0.2)); label("B", B, N);
            draw(Circle(C,0.2)); label("C", C, S);
            draw(Circle(D,0.2)); label("D", D, N);
            draw(Circle(E,0.2)); label("E", E, S);
            draw(Circle(F,0.2)); label("F", F, E);
            
            // Flechas con sus etiquetas
            draw(A--B, Arrow()); label("$9$", A--B, N);
            draw(A--C, Arrow()); label("$7$", A--C, S);
            draw(B--C, Arrow()); label("$2$", B--C, W);
            draw(B--D, Arrow()); label("$7$", B--D, N);
            draw(C--B, Arrow()); label("$4$", C--B, N);
            draw(C--E, Arrow()); label("$6$", C--E, S);
            draw(D--F, Arrow()); label("$6$", D--F, N);
            draw(D--E, Arrow()); label("$3$", D--E, E);
            draw(E--F, Arrow()); label("$9$", E--F, S);
        \end{asy}
    \end{center}
        
\end{enumerate}

\end{document}
