\documentclass{beamer}
\usetheme{metropolis}

\usepackage[spanish]{babel}
\usepackage[utf8]{inputenc}
\usepackage{graphicx}
\usepackage{tikz}
\usepackage{xcolor}
\usepackage{amsmath}
\usepackage{listings}
\usepackage{pgfplots}
\pgfplotsset{compat=1.18}
\usepackage{booktabs}
\usepackage{fontawesome5}

% Definición de colores personalizados
\definecolor{primary}{RGB}{46, 204, 113}
\definecolor{secondary}{RGB}{52, 152, 219}
\definecolor{accent}{RGB}{231, 76, 60}
\definecolor{background}{RGB}{236, 240, 241}
\definecolor{gradient1}{RGB}{255, 107, 107}
\definecolor{gradient2}{RGB}{255, 159, 67}

% Configuración del tema
\setbeamercolor{normal text}{fg=black,bg=background}
\setbeamercolor{structure}{fg=primary}
\setbeamercolor{alerted text}{fg=accent}

\definecolor{lightgray}{rgb}{0.95,0.95,0.95}
\definecolor{darkgreen}{rgb}{0,0.5,0}
\definecolor{darkblue}{rgb}{0,0,0.5}

\lstset{
  backgroundcolor=\color{lightgray},
  basicstyle=\tiny\ttfamily,
  keywordstyle=\color{darkblue}\bfseries,
  commentstyle=\color{darkgreen},
  stringstyle=\color{red},
  numbers=left,
  numberstyle=\tiny\color{gray},
  stepnumber=1,
  numbersep=5pt,
  showspaces=false,
  showstringspaces=false,
  showtabs=false,
  frame=single,
  tabsize=2,
  language=Python,
  breaklines=true,
  breakatwhitespace=true
}

\title{\Huge\textbf{Programación Lineal - Pt 1}}
\author{Investigación Operativa}
\date{}

\begin{document}

\begin{frame}
    \titlepage
    \begin{tikzpicture}[remember picture,overlay]
        \node[anchor=south west,inner sep=30pt] at (current page.south west) {
            \includegraphics[height=1cm]{../misc/UdeSA.png}
        };
    \end{tikzpicture}
\end{frame}

\begin{frame}{¿Qué es la Programación Lineal?}
    \textbf{Definición:}
    \begin{itemize}
        \item Técnica de optimización matemática
        \item Busca la mejor solución a problemas con función objetivo y restricciones lineales
        \item Herramienta fundamental para toma de decisiones
        \item Optimización de recursos limitados
    \end{itemize}
\end{frame}

\begin{frame}{Aplicaciones de la Programación Lineal}
    \textbf{Aplicaciones:}
    \begin{itemize}
        \item[{\makebox[1em][c]{\textcolor{black}{\faIndustry}}}] Producción y manufactura
        \item[{\makebox[1em][c]{\textcolor{black}{\faTruck}}}] Logística y transporte
        \item[{\makebox[1em][c]{\textcolor{black}{\faMoneyBill}}}] Finanzas y inversiones
    \end{itemize}
\end{frame}

\begin{frame}{Problema de PL General - Componentes}
    \textbf{Elementos del modelo:}
    \begin{itemize}
        \item<1-> \textbf{Variables de decisión} ($x_j$): Nivel de cada actividad
        \item<2-> \textbf{Función objetivo} ($Z$): Medida de performance a optimizar
        \item<3-> \textbf{Parámetros:}
        \begin{itemize}
            \item $c_j$: Aumento de $Z$ por unidad de $x_j$
            \item $b_i$: Cantidad del recurso $i$ disponible
            \item $a_{ij}$: Cantidad del recurso $i$ que consume cada unidad de actividad $j$
        \end{itemize}
    \end{itemize}
\end{frame}

\begin{frame}{Problema de PL General - Formulación}
    \textbf{Forma estándar:}
    \begin{align*}
        &\textbf{Min } Z \\
        &\text{Dado} \\
        &\qquad Z = c x^t \\
        &\qquad a x^t \leq b \\
        &\qquad x \geq 0
    \end{align*}
    
    Donde:
    \begin{itemize}
        \item $x = (x_1, x_2, \ldots, x_n)$ - vector de variables
        \item $c = (c_1, c_2, \ldots, c_n)$ - vector de coeficientes
        \item $b = (b_1, b_2, \ldots, b_m)$ - vector de recursos
        \item $a$ - matriz de consumo de recursos por unidad de actividad
    \end{itemize}
\end{frame}

\begin{frame}{Ejemplo 1: Panes y Tortas}
    \textbf{Situación:}
    Jorge tiene una panadería que produce panes y tortas.
    
    \vspace{1em}
    \textbf{Recursos por unidad:}
    \begin{itemize}
        \item \textbf{Pan:} 1 kg harina, 0.2 kg levadura
        \item \textbf{Torta:} 0.5 kg harina, 0.1 kg levadura, 0.2 kg azúcar
    \end{itemize}
    
    \vspace{1em}
    \textbf{Disponibilidad diaria:}
    \begin{itemize}
        \item 50 kg de harina
        \item 15 kg de levadura  
        \item 10 kg de azúcar
    \end{itemize}
    
    \vspace{1em}
    \textbf{Ganancias:} Pan \$2.5, Torta \$1.8
\end{frame}

\begin{frame}{Ejemplo 1: Formulación}
    \onslide<1->{
    \textbf{Variables de decisión:}
    \begin{itemize}
        \item $x_1$: Cantidad de panes a producir
        \item $x_2$: Cantidad de tortas a producir
    \end{itemize}
    }
    
    \vspace{1em}
    \onslide<2->{
    \textbf{Función objetivo:}
    \begin{align*}
        \text{Maximizar: } Z = 2.5x_1 + 1.8x_2
    \end{align*}
    }
    
    \onslide<3->{
    \textbf{Restricciones:}
    \begin{align*}
        x_1 + 0.5x_2 &\leq 50 \quad \text{(harina)} \\
        0.2x_1 + 0.1x_2 &\leq 15 \quad \text{(levadura)} \\
        0.2x_2 &\leq 10 \quad \text{(azúcar)} \\
        20 \leq x_1 &\leq 50, \quad 15 \leq x_2 \leq 20
    \end{align*}
    }
\end{frame}

\begin{frame}{Ejemplo 2: Wyndor Glass Co.}
    \textbf{Productos:}
    \begin{itemize}
        \item \textbf{Producto 1:} Ventana 2m con marco de aluminio
        \item \textbf{Producto 2:} Ventana colgante 3m con marco de madera
    \end{itemize}
    
    \vspace{1em}
    \textbf{Recursos de producción:}
    \begin{table}[H]
        \centering
        \tiny
        \begin{tabular}{l|cc|c}
            \textbf{Planta} & \textbf{Producto 1} & \textbf{Producto 2} & \textbf{Tiempo disponible} \\
            \midrule
            1 & 1 & 0 & 4 \\
            2 & 0 & 2 & 12 \\
            3 & 3 & 2 & 18 \\
        \end{tabular}
    \end{table}
    
    \vspace{1em}
    \textbf{Ganancias:} Producto 1: \$3000, Producto 2: \$5000
\end{frame}

\begin{frame}{Wyndor Glass Co. - Formulación}
    \onslide<1->{
    \textbf{Variables de decisión:}
    \begin{itemize}
        \item $x_1$: Lotes por semana del Producto 1
        \item $x_2$: Lotes por semana del Producto 2
    \end{itemize}
    }
    \vspace{1em}
    \onslide<2->{
    \textbf{Función objetivo:}
    \begin{align*}
        \text{Maximizar: } Z = 3000x_1 + 5000x_2
    \end{align*}
    }
    
    \onslide<3->{
    \textbf{Restricciones:}
    \begin{align*}
        x_1 &\leq 4 \quad \text{(Planta 1)} \\
        2x_2 &\leq 12 \quad \text{(Planta 2)} \\
        3x_1 + 2x_2 &\leq 18 \quad \text{(Planta 3)} \\
        x_1, x_2 &\geq 0
    \end{align*}
    }
\end{frame}

\begin{frame}{Método Simplex - Concepto}
    \textbf{¿Que es el Metodo Simplex?}
    \begin{itemize}
        \item Es un algoritmo iterativo para resolver problemas de programacion lineal
        \item Se basa en que la solucion optima esta en un vertice de la region factible
        \item Se mueve de vertice en vertice mejorando la funcion objetivo
        \item Se detiene cuando no hay vertices adyacentes mejores
    \end{itemize}
\end{frame}

\begin{frame}{Método Simplex - Pasos del algoritmo}
    \textbf{Pasos del algoritmo:}
    \begin{itemize}
        \item<1->[{\makebox[1em][c]{\textcolor{black}{\faSearch}}}] Encuentra un vertice inicial factible
        \item<2->[{\makebox[1em][c]{\textcolor{black}{\faMap}}}] Evalua vertices adyacentes
        \item<3->[{\makebox[1em][c]{\textcolor{black}{\faArrowRight}}}] Se mueve al mejor vertice adyacente
        \item<4->[{\makebox[1em][c]{\textcolor{black}{\faRandom}}}] Repite hasta encontrar el optimo
    \end{itemize}
\end{frame}

\begin{frame}{Método Simplex - Paso 1: Restricciones de No Negatividad}
    \begin{center}
    \begin{tikzpicture}
    \begin{axis}[
        xlabel=$x_1$,
        ylabel=$x_2$,
        xmin=-1, xmax=7,
        ymin=-1, ymax=7,
        grid=both,
        grid style={line width=.1pt, draw=gray!10},
        major grid style={line width=.2pt,draw=gray!50},
        axis lines=middle,
        samples=100,
        domain=-1:7,
        legend pos=outer north east
    ]
    \addplot[color=red, thick] coordinates {(0,-1) (0,7)};
    \addlegendentry{$x_1 \geq 0$}
    \addplot[color=blue, thick] coordinates {(-1,0) (7,0)};
    \addlegendentry{$x_2 \geq 0$}
    \fill[gray!50,opacity=0.3] (0,0) -- (0,7) -- (7,7) -- (7,0) -- cycle;
    \end{axis}
    \end{tikzpicture}
    \end{center}
\end{frame}

\begin{frame}{Método Simplex - Paso 2: Agregamos $x_1 \leq 4$}
    \begin{center}
    \begin{tikzpicture}
    \begin{axis}[
        xlabel=$x_1$,
        ylabel=$x_2$,
        xmin=-1, xmax=7,
        ymin=-1, ymax=7,
        grid=both,
        grid style={line width=.1pt, draw=gray!10},
        major grid style={line width=.2pt,draw=gray!50},
        axis lines=middle,
        samples=100,
        domain=-1:7,
        legend pos=outer north east
    ]
    \addplot[color=red, thick] coordinates {(0,-1) (0,7)};
    \addlegendentry{$x_1 \geq 0$}
    \addplot[color=blue, thick] coordinates {(-1,0) (7,0)};
    \addlegendentry{$x_2 \geq 0$}
    \addplot[color=green, thick] coordinates {(4,-1) (4,7)};
    \addlegendentry{$x_1 \leq 4$}
    \fill[gray!50,opacity=0.3] (0,0) -- (0,7) -- (4,7) -- (4,0) -- cycle;
    \end{axis}
    \end{tikzpicture}
    \end{center}
\end{frame}

\begin{frame}{Método Simplex - Paso 3: Agregamos $2x_2 \leq 12$}
    \begin{center}
    \begin{tikzpicture}
    \begin{axis}[
        xlabel=$x_1$,
        ylabel=$x_2$,
        xmin=-1, xmax=7,
        ymin=-1, ymax=7,
        grid=both,
        grid style={line width=.1pt, draw=gray!10},
        major grid style={line width=.2pt,draw=gray!50},
        axis lines=middle,
        samples=100,
        domain=-1:7,
        legend pos=outer north east
    ]
    \addplot[color=red, thick] coordinates {(0,-1) (0,7)};
    \addlegendentry{$x_1 \geq 0$}
    \addplot[color=blue, thick] coordinates {(-1,0) (7,0)};
    \addlegendentry{$x_2 \geq 0$}
    \addplot[color=green, thick] coordinates {(4,-1) (4,7)};
    \addlegendentry{$x_1 \leq 4$}
    \addplot[color=purple, thick] coordinates {(-1,6) (7,6)};
    \addlegendentry{$2x_2 \leq 12$}
    \fill[gray!50,opacity=0.3] (0,0) -- (0,6) -- (4,6) -- (4,0) -- cycle;
    \end{axis}
    \end{tikzpicture}
    \end{center}
\end{frame}

\begin{frame}{Método Simplex - Paso 4: Agregamos $3x_1 + 2x_2 \leq 18$}
    \begin{center}
\begin{tikzpicture}
\begin{axis}[
    xlabel=$x_1$,
    ylabel=$x_2$,
    xmin=-1, xmax=7,
    ymin=-1, ymax=7,
    grid=both,
    grid style={line width=.1pt, draw=gray!10},
    major grid style={line width=.2pt,draw=gray!50},
    axis lines=middle,
    samples=100,
    domain=-1:7,
    legend pos=outer north east
]
\addplot[color=red, thick] coordinates {(0,-1) (0,7)};
\addlegendentry{$x_1 \geq 0$}
\addplot[color=blue, thick] coordinates {(-1,0) (7,0)};
\addlegendentry{$x_2 \geq 0$}
\addplot[color=green, thick] coordinates {(4,-1) (4,7)};
\addlegendentry{$x_1 \leq 4$}
\addplot[color=purple, thick] coordinates {(-1,6) (7,6)};
\addlegendentry{$2x_2 \leq 12$}
\addplot[color=orange, thick, domain=-1:7] { (18 - 3*x)/2 };
\addlegendentry{$3x_1 + 2x_2 \leq 18$}
\fill[gray!50,opacity=0.3] (0,0) -- (0,6) -- (2,6) -- (4,3) -- (4,0) -- cycle;
\end{axis}
\end{tikzpicture}
\end{center}
\end{frame}

\begin{frame}{Wyndor Glass Co. - Solución Óptima}
    \textbf{Evaluación en vértices:}
    \begin{itemize}
        \item<1-> $(0,0)$: $Z = 0$
        \item<2-> $(0,6)$: $Z = 30000$
        \item<3-> $(2,6)$: $Z = 36000$ \textcolor{accent}{\textbf{← ÓPTIMO}}
        \item<4-> $(4,3)$: $Z = 27000$
        \item<5-> $(4,0)$: $Z = 12000$
    \end{itemize}
    
    \vspace{1em}
    \onslide<6->{
    \textbf{Solución óptima:}
    \begin{itemize}
        \item Producir 2 lotes del Producto 1
        \item Producir 6 lotes del Producto 2
        \item Ganancia máxima: \$36000
    \end{itemize}
    }
\end{frame}

\begin{frame}{Implementación con PICOS}
    \textbf{¿Qué es PICOS?}
    \begin{itemize}
        \item Interfaz de Python para solucionadores de optimización
        \item Simplifica la formulación de problemas de PL
        \item Se conecta con solucionadores como GLPK, CPLEX, Gurobi
    \end{itemize}
    
    \vspace{1em}
    \onslide<2->{
    \textbf{Ventajas:}
    \begin{itemize}
        \item Sintaxis clara y legible
        \item Manejo automático de matrices
        \item Múltiples solucionadores disponibles
        \item Integración con NumPy
    \end{itemize}
    }
\end{frame}

\begin{frame}[fragile]{PICOS - Ejemplo Wyndor}
    \begin{lstlisting}[language=Python]
import picos
import numpy as np

# Crear problema
P = picos.Problem()

# Definir variables
x = picos.RealVariable('x', 2)

# Funcion objetivo
P.set_objective('max', 3000*x[0] + 5000*x[1])

# Restricciones
P.add_constraint(x[0] <= 4)
P.add_constraint(2*x[1] <= 12)
P.add_constraint(3*x[0] + 2*x[1] <= 18)

# Resolver
P.solve(solver='glpk')
print(x)  # [2.0, 6.0]
print(P.value)  # 36000.0
    \end{lstlisting}
\end{frame}

\begin{frame}[fragile]{PICOS - Forma Matricial}
    \begin{lstlisting}[language=Python]
P = picos.Problem()
x = picos.RealVariable('x', 2)

# Definir matrices
A = np.array([[1,0], [0,2], [3,2]])
c = np.array([3000,5000])
b = np.array([4,12,18])

# Convertir a constantes PICOS
c = picos.Constant('c', c)
A = picos.Constant('A', A)
b = picos.Constant('b', b)

# Funcion objetivo y restricciones
P.set_objective('max', c|x)
P.add_constraint(A*x <= b)

P.solve(solver='glpk')
    \end{lstlisting}
\end{frame}

\begin{frame}{Casos Especiales en PL}
    \textbf{Tipos de soluciones:}
    \begin{itemize}
        \item<1-> \textbf{Solución única:} El óptimo está en un único vértice
        \item<2-> \textbf{Múltiples soluciones:} El óptimo está en una arista completa
        \item<3-> \textbf{Problema no acotado:} La función objetivo puede crecer infinitamente
        \item<4-> \textbf{Problema infactible:} No existe solución que satisfaga todas las restricciones
    \end{itemize}
    
    \vspace{1em}
    \onslide<5->{
    \textbf{¿Cómo identificarlos?}
    \begin{itemize}
        \item El solucionador nos lo indica
        \item Análisis gráfico (en 2D)
        \item Verificación de las restricciones
    \end{itemize}
    }
\end{frame}

\begin{frame}{Próximos Pasos}
    \textbf{En la siguiente clase veremos:}
    \begin{itemize}
        \item Problemas de Redes
        \item Problema de Transporte (balanceado y desbalanceado)
        \item Problema de Transshipment
        \item Problema de Flujo Máximo
    \end{itemize}
    
    \vspace{1em}
    \textbf{Para practicar:}
    \begin{itemize}
        \item Resolver los ejercicios de lámparas y dieta óptima
        \item Resolver los ejercicios de la guía
        \item Experimentar con PICOS
        \item Identificar problemas de PL en situaciones reales
    \end{itemize}
\end{frame}

\begin{frame}{Terminamos}
    \begin{center}
        \Large{\textbf{¿Dudas?\\¿Consultas?}}
    \end{center}
    \begin{tikzpicture}[remember picture,overlay]
        \node[anchor=south,inner sep=30pt] at (current page.south) {
            \includegraphics[height=1cm]{../misc/UdeSA.png}
        };
    \end{tikzpicture}
\end{frame}

\end{document}
