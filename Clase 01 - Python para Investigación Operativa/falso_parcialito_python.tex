\documentclass[12pt]{article}

\usepackage[utf8]{inputenc}
\usepackage[T1]{fontenc}
\usepackage{lmodern}
\usepackage[spanish]{babel}
\usepackage{booktabs}
\usepackage{amsmath}
\usepackage{forest}
\usepackage{float}
\usepackage{listings}
\usepackage{xcolor}
\usepackage{tikz}
\usepackage{enumitem}
\usepackage{ulem}

\definecolor{codegreen}{rgb}{0,0.6,0}
\definecolor{codegray}{rgb}{0.5,0.5,0.5}
\definecolor{codepurple}{rgb}{0.58,0,0.82}
\definecolor{backcolour}{rgb}{0.95,0.95,0.92}

\lstdefinestyle{mystyle}{
    backgroundcolor=\color{backcolour},   
    commentstyle=\color{codegreen},
    keywordstyle=\color{magenta},
    numberstyle=\tiny\color{codegray},
    stringstyle=\color{codepurple},
    basicstyle=\ttfamily\footnotesize,
    breakatwhitespace=false,         
    breaklines=true,                 
    captionpos=b,                    
    keepspaces=true,                 
    numbers=left,                    
    numbersep=5pt,                  
    showspaces=false,                
    showstringspaces=false,
    showtabs=false,                  
    tabsize=2
}

\lstset{style=mystyle}

\sloppy
\setlength{\parindent}{0pt}

\begin{document}

% Título y materia
\begin{center}
  {\LARGE \textbf{Falso Parcialito}}\\[0.5em]
  {Investigación Operativa, Universidad de San Andrés}
\end{center}

\textbf{Nombre:} \dotuline{\hspace{\fill}} \\
\textbf{¿Cursaste IPC? ¿Cuándo?} \dotuline{\hspace{\fill}}

\section{Ciclos y Condicionales}

Una tienda registra las ventas diarias de la semana. Se pide crear un programa que analice estos datos.

\begin{lstlisting}[language=Python]
ventas_semana = [1200, 1500, 800, 2000, 1800, 2200, 1600]
dias = ['Lunes', 'Martes', 'Miercoles', 'Jueves', 'Viernes', 'Sabado', 'Domingo']
\end{lstlisting}

Hacer una función por cada uno de los siguientes puntos:
\begin{enumerate}[label=\alph*)]
    \item Calcular el promedio de ventas de la semana.
    \item Encontrar el día con mayor venta y mostrar su nombre.
    \item Contar cuántos días tuvieron ventas superiores a 1500.
    \item Crear una lista con los días que tuvieron ventas menores al promedio.
\end{enumerate}

\section{Análisis de Código}

Analiza el siguiente código y responde las preguntas:
\begin{lstlisting}[language=Python]
def procesar_ventas(productos, ventas_diarias):
    resultado = {}
    total_ventas = 0
    
    for producto in productos:
        resultado[producto] = {'cantidad': 0, 'ganancia': 0}
    
    for venta in ventas_diarias:
        if venta['cantidad'] > 0:
            producto = venta['producto']
            if producto in resultado:
                resultado[producto]['cantidad'] += venta['cantidad']
                resultado[producto]['ganancia'] += venta['precio'] * venta['cantidad']
                total_ventas += venta['cantidad']
            else:
                break
    
    return resultado, total_ventas

productos = ['laptop', 'mouse', 'teclado']
ventas = [
    {'producto': 'laptop', 'cantidad': 2, 'precio': 500},
    {'producto': 'mouse', 'cantidad': 5, 'precio': 25},
    {'producto': 'teclado', 'cantidad': 3, 'precio': 50},
    {'producto': 'laptop', 'cantidad': 1, 'precio': 500},
    {'producto': 'monitor', 'cantidad': 1, 'precio': 200}
]

resultado_final, total = procesar_ventas(productos, ventas)
\end{lstlisting}

\begin{enumerate}[label=\alph*)]
    \item ¿Cuál es el valor de \texttt{resultado\_final} al final de la ejecución?
    \item ¿Cuál es el valor de \texttt{total}?
    \item ¿Por qué el producto 'monitor' no aparece en el resultado final?
    \item Si cambiamos el último elemento de \texttt{ventas} por \texttt{'producto': 'mouse', 'cantidad': 2, 'precio': 25}, ¿cuál sería el nuevo valor de \texttt{total}?
\end{enumerate}

\section{Algoritmos de Investigación Operativa}

Un sistema de inventarios necesita calcular el punto de reorden óptimo. Se pide crear un algoritmo que simule el comportamiento del inventario.

\begin{enumerate}[label=\alph*)]
    \item Crear una función \texttt{simular\_inventario(demanda\_diaria, stock\_inicial, punto\_reorden, dias\_simulacion)} que:
    \begin{itemize}
        \item Simule el comportamiento del inventario día a día generando una demanda aleatoria entre 0 y \texttt{demanda\_diaria}
        \item Si el stock cae por debajo del punto de reorden, haga un pedido de 50 unidades
        \item Retorne una lista con el stock de cada día
    \end{itemize}
\end{enumerate}

\end{document}
