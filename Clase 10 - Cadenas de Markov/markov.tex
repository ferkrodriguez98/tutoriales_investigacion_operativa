\documentclass[a4paper,11pt]{article}

\usepackage[spanish]{babel}
\usepackage[utf8]{inputenc}
\usepackage{amsmath}
\usepackage{amsfonts}
\usepackage{amssymb}
\usepackage{graphicx}
\usepackage{tikz}
\usepackage{enumitem}

\setlength{\parindent}{0pt}

\begin{document}

\begin{center}
    {\LARGE \textbf{Cadenas de Markov}}\\[0.5em]
    {Investigación Operativa, Universidad de San Andrés}
\end{center}

Si encuentran algún error en el documento o hay alguna duda, mandenmé un mail a rodriguezf@udesa.edu.ar y lo revisamos.

\section{Introducción a las Cadenas de Markov}

Una cadena de Markov es un proceso estocástico que cumple con la propiedad de Markov, es decir, la probabilidad de cualquier estado futuro depende únicamente del estado presente y no de la secuencia de eventos que le precedieron.

\subsection{Definición Formal}

Sea $\{X_n, n \geq 0\}$ un proceso estocástico con espacio de estados $S$. Decimos que es una cadena de Markov si para todo $n \geq 0$ y para todos los estados $i_0, i_1, ..., i_n, j \in S$:

\[P(X_{n+1} = j | X_n = i_n, X_{n-1} = i_{n-1}, ..., X_0 = i_0) = P(X_{n+1} = j | X_n = i_n)\]

\subsection{Matriz de Transición}

La matriz de transición $P = (p_{ij})$ contiene las probabilidades de transición de un estado a otro, donde:

\[p_{ij} = P(X_{n+1} = j | X_n = i)\]

Propiedades importantes:
\begin{itemize}
    \item $0 \leq p_{ij} \leq 1$ para todo $i,j \in S$
    \item $\sum_{j \in S} p_{ij} = 1$ para todo $i \in S$
\end{itemize}

\section{Ejercicios Resueltos}

\subsection{Ejercicio 1: Predicción del Clima}

Un meteorólogo estudia el clima en Buenos Aires y clasifica los días en tres estados posibles: Soleado (S), Nublado (N) y Lluvioso (L). Las probabilidades de transición son:

\begin{itemize}
    \item Si hoy está soleado:
        \begin{itemize}
            \item 70\% de probabilidad de que mañana esté soleado
            \item 20\% de probabilidad de que esté nublado
            \item 10\% de probabilidad de que llueva
        \end{itemize}
    \item Si hoy está nublado:
        \begin{itemize}
            \item 30\% de probabilidad de que mañana esté soleado
            \item 40\% de probabilidad de que siga nublado
            \item 30\% de probabilidad de que llueva
        \end{itemize}
    \item Si hoy llueve:
        \begin{itemize}
            \item 20\% de probabilidad de que mañana esté soleado
            \item 40\% de probabilidad de que esté nublado
            \item 40\% de probabilidad de que siga lloviendo
        \end{itemize}
\end{itemize}

\begin{itemize}
    \item[a)] Construir la matriz de transición $P$ que representa este sistema.
    \item[b)] ¿Cuál es la probabilidad de que llueva dentro de dos días si hoy está soleado?
\end{itemize}

\textbf{Solución:}

\vspace{0.5em}

a) La matriz de transición $P$ es:

\[P = \begin{pmatrix}
0.7 & 0.2 & 0.1 \\
0.3 & 0.4 & 0.3 \\
0.2 & 0.4 & 0.4
\end{pmatrix}\]

b) Si hoy está soleado, la probabilidad de que dentro de dos días llueva es:

\[P(X_2 = L | X_0 = S) = \sum_{j \in S} P(X_1 = j | X_0 = S)P(X_2 = L | X_1 = j)\]

Para calcular la probabilidad de que llueva en dos días partiendo de un día soleado, necesitamos considerar todos los caminos posibles:

\begin{itemize}
    \item Si mañana está soleado (0.7) y luego llueve (0.1)
    \item Si mañana está nublado (0.2) y luego llueve (0.3)
    \item Si mañana llueve (0.1) y sigue lloviendo (0.4)
\end{itemize}

\[
\begin{aligned}
P(X_2 = L | X_0 = S) &= (0.7)(0.1) + (0.2)(0.3) + (0.1)(0.4) \\
&= 0.07 + 0.06 + 0.04 \\
&= 0.17
\end{aligned}
\]

Por lo tanto, hay un 17\% de probabilidad de que llueva dentro de dos días si hoy está soleado.

\subsection{Ejercicio 2: Puntos en Tenis}

Final de Roland Garros 2025. Francisco Cerúndolo está contra Holger Rune con match point. Siendo el analista de tenis de Cerúndolo, se sabe las siguientes probabilidades:

\begin{itemize}
    \item Hay un 50\% de chances de ganar el punto de un ace o un saque no devuelto
    \item Tiene 30\% de chances de que la pelota entre con el primer saque y se arme el punto
    \item Si erra el primer saque, tiene 90\% de chances de meter el segundo
    \item Hay un 2\% de chances de que gane el punto directamente con el segundo saque
    \item En cualquier peloteo (sea con primer o segundo saque), tiene 55\% de chances de ganar el punto
\end{itemize}

Resuelva los siguientes items:

\begin{itemize}
    \item[a)] Identifique los estados posibles del sistema y construya el diagrama de transición.
    \item[b)] Calcule la probabilidad de que Cerúndolo gane el punto y se consagre campeón.
    \item[c)] Si se jugaran infinitos puntos con estas probabilidades, ¿qué porcentaje ganaría cada jugador?
\end{itemize}

\textbf{Solución:}

\vspace{0.5em}

a) Los estados posibles son:
\begin{itemize}
    \item S: Primer saque de Cerúndolo
    \item D: Segundo saque de Cerúndolo
    \item P: Punto en juego (peloteo)
    \item W: Punto ganado por Cerúndolo
    \item L: Punto perdido por Cerúndolo (ganado por Rune)
\end{itemize}

\vspace{0.5em}

b) La matriz de transición es:

\[P = \begin{pmatrix}
0 & 0.20 & 0.30 & 0.50 & 0 \\
0 & 0 & 0.90 & 0.02 & 0.08 \\
0 & 0 & 0 & 0.55 & 0.45 \\
0 & 0 & 0 & 1 & 0 \\
0 & 0 & 0 & 0 & 1
\end{pmatrix}\]

\begin{center}
Donde las filas y columnas siguen el orden S, D, P, W, L.
\end{center}

\vspace{0.5em}
Analicemos cada fila:
\begin{itemize}
    \item Primera fila (S): 0.20 de ir a D (errar primer saque), 0.30 de ir a P (peloteo), 0.50 de ir a W (ganar directo)
    \item Segunda fila (D): 0.90 de ir a P (peloteo), 0.02 de ir a W (ganar directo), 0.08 de ir a L (doble falta)
    \item Tercera fila (P): 0.55 de ir a W (ganar el punto), 0.45 de ir a L (perder el punto)
    \item Cuarta y quinta filas (W y L): estados absorbentes (probabilidad 1 de quedarse en el mismo estado)
\end{itemize}

La probabilidad de que Cerúndolo gane el punto se puede calcular considerando todos los caminos posibles:
\begin{itemize}
    \item Ganar directo con el primer saque (0.50)
    \item Ir a peloteo con primer saque (0.30) y ganar el peloteo (0.55)
    \item Errar primer saque (0.20), meter segundo (0.90) y ganar el peloteo (0.55)
    \item Errar primer saque (0.20), ganar directo con el segundo saque (0.02)
\end{itemize}

Resolvemos entonces la ecuación:
\[
\begin{split}
P(\text{Cerúndolo gana}) &= 0.50 + (0.30 \cdot 0.55) + (0.20 \cdot 0.90 \cdot 0.55) + (0.20 \cdot 0.02) \\
&= 0.50 + 0.165 + 0.099 + 0.004 \\
&= 0.768
\end{split}
\]

Por lo tanto, Cerúndolo tiene aproximadamente un 76.8\% de probabilidad de ganar el punto.

\vspace{0.5em}

c) A largo plazo, considerando que cada punto es independiente y las probabilidades se mantienen constantes:
\begin{itemize}
    \item Cerúndolo ganaría aproximadamente el 76.8\% de los puntos
    \item Rune ganaría aproximadamente el 23.2\% de los puntos
\end{itemize}

\subsection{Ejercicio 3: Máquina de Juguetes}

En una fábrica de juguetes, hay una máquina que produce muñecos de peluche. La máquina puede estar en tres estados:

\begin{itemize}
    \item F: Funcionando perfectamente (0.80)
    \item M: Mal funcionamiento (0.15)
    \item R: Rota (0.05)
\end{itemize}

Cuando la máquina está funcionando perfectamente, produce 100 muñecos por hora. En mal funcionamiento, produce 50 muñecos por hora. Cuando está rota, no produce nada.

\vspace{0.5em}

El técnico de mantenimiento puede:
\begin{itemize}
    \item Reparar la máquina (0.90)
    \item Empeorar la situación (0.10)
\end{itemize}

Resuelva los siguientes items:

\begin{itemize}
    \item[a)] Construya la matriz de transición considerando los estados de la máquina y las acciones del técnico.
    \item[b)] Si la máquina comienza funcionando perfectamente, ¿cuál es la probabilidad de que esté rota después de dos intervenciones del técnico?
    \item[c)] ¿Cuál es la probabilidad de que la máquina funcione perfectamente durante tres horas consecutivas?
\end{itemize}

\textbf{Ayuda:}
\begin{itemize}
    \item La máquina puede cambiar de estado por sí sola o por la intervención del técnico
    \item El técnico solo interviene cuando la máquina no está funcionando perfectamente
    \item Las probabilidades son independientes del tiempo
\end{itemize}

\textbf{Resolución:}

\vspace{0.5em}

a) La matriz de transición es:

\[P = \begin{pmatrix}
0.80 & 0.15 & 0.05 \\
0.90 & 0 & 0.10 \\
0.90 & 0 & 0.10
\end{pmatrix}\]

\begin{center}
    Donde las filas y columnas siguen el orden F, M, R.
\end{center}

Explicación de los valores:
\begin{itemize}
    \item Primera fila: Desde F, la máquina tiene 80\% de probabilidad de seguir en F, 15\% de pasar a M y 5\% de pasar a R
    \item Segunda fila: Desde M, el técnico tiene 90\% de probabilidad de arreglarla (F), 0\% de dejarla igual (M) y 10\% de empeorarla (R)
    \item Tercera fila: Desde R, el técnico tiene 90\% de probabilidad de arreglarla (F), 0\% de dejarla en M y 10\% de dejarla igual (R)
\end{itemize}

b) Para calcular la probabilidad de que esté rota después de dos intervenciones, partiendo de F, analizamos todas las posibles combinaciones válidas:

\vspace{0.5em}
Posibles caminos, sin considerar que la arregla porque si no sería infinito:
\begin{itemize}
    \item F $\rightarrow$ M $\rightarrow$ M $\rightarrow$ R
    \item F $\rightarrow$ R $\rightarrow$ R $\rightarrow$ R  
    \item F $\rightarrow$ R $\rightarrow$ M $\rightarrow$ R
    \item F $\rightarrow$ M $\rightarrow$ R $\rightarrow$ R
\end{itemize}

Calculando cada probabilidad:
\[
\begin{aligned}[c]
P(F \rightarrow M \rightarrow M \rightarrow R) &= 0.15 \cdot 0 \cdot 0.10 = 0 \\
P(F \rightarrow R \rightarrow R \rightarrow R) &= 0.05 \cdot 0.10 \cdot 0.10 = 0.0005 \\
P(F \rightarrow R \rightarrow M \rightarrow R) &= 0.05 \cdot 0 \cdot 0.10 = 0 \\
P(F \rightarrow M \rightarrow R \rightarrow R) &= 0.15 \cdot 0.10 \cdot 0.10 = 0.0015
\end{aligned}
\]

Sumando todas las probabilidades:
\[
\begin{split}
P(\text{Rota después de 2 intervenciones}) &= 0 + 0.0005 + 0 + 0.0015 \\
&= 0.002
\end{split}
\]

\vspace{0.5em}

c) Para que funcione perfectamente durante tres horas consecutivas:
\[
\begin{split}
P(\text{F durante 3 horas}) &= 0.80 \cdot 0.80 \cdot 0.80 \\
&= 0.512
\end{split}
\]

\end{document}
