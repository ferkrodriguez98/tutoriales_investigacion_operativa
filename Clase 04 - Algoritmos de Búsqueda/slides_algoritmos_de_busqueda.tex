\documentclass{beamer}
\usetheme{metropolis}

\usepackage[spanish]{babel}
\usepackage[utf8]{inputenc}
\usepackage{graphicx}
\usepackage{tikz}
\usepackage{xcolor}
\usepackage{amsmath}
\usepackage{listings}
\usepackage{fontawesome5}

\usetikzlibrary{positioning,shapes.multipart,calc,arrows,shapes.geometric}

% Definición de colores personalizados
\definecolor{primary}{RGB}{46, 204, 113}
\definecolor{secondary}{RGB}{52, 152, 219}
\definecolor{accent}{RGB}{231, 76, 60}
\definecolor{background}{RGB}{236, 240, 241}
\definecolor{gradient1}{RGB}{255, 107, 107}
\definecolor{gradient2}{RGB}{255, 159, 67}

% Configuración del tema
\setbeamercolor{normal text}{fg=black,bg=background}
\setbeamercolor{structure}{fg=primary}
\setbeamercolor{alerted text}{fg=accent}

\definecolor{lightgray}{rgb}{0.95,0.95,0.95}
\definecolor{darkgreen}{rgb}{0,0.5,0}
\definecolor{darkblue}{rgb}{0,0,0.5}

\lstset{
  backgroundcolor=\color{lightgray},
  basicstyle=\tiny\ttfamily,
  keywordstyle=\color{darkblue}\bfseries,
  commentstyle=\color{darkgreen},
  stringstyle=\color{red},
  numbers=left,
  numberstyle=\tiny\color{gray},
  stepnumber=1,
  numbersep=5pt,
  showspaces=false,
  showstringspaces=false,
  showtabs=false,
  frame=single,
  tabsize=2,
  language=Python,
  breaklines=true,
  breakatwhitespace=true
}

\title{\Huge\textbf{Algoritmos de Búsqueda}}
\author{Investigación Operativa}
\date{}

\begin{document}

\begin{frame}
    \titlepage
    \begin{tikzpicture}[remember picture,overlay]
        \node[anchor=south west,inner sep=30pt] at (current page.south west) {
            \includegraphics[height=1cm]{../misc/UdeSA.png}
        };
    \end{tikzpicture}
\end{frame}

\begin{frame}{¿Qué son los algoritmos de búsqueda?}
    \textbf{Herramientas fundamentales} en la resolución de problemas de optimización y toma de decisiones.
    
    \vspace{0.5cm}
    
    \begin{center}
        \begin{tabular}{cc}
            \begin{minipage}{4cm}
                \centering
                {\Large \faSitemap}\\[0.2cm]
                \textbf{Árboles y Grafos}\\[0.1cm]
                {\scriptsize BFS, DFS, Dijkstra, A*}
            \end{minipage} &
            \begin{minipage}{4cm}
                \centering
                {\Large \faList}\\[0.2cm]
                \textbf{Listas}\\[0.1cm]
                {\scriptsize Secuencial, Binaria}
            \end{minipage}
        \end{tabular}
    \end{center}
\end{frame}

\begin{frame}{Tipos de algoritmos de búsqueda}
    \only<1>{
        \textbf{Algoritmos para árboles y grafos:}
        \begin{itemize}
            \item \textbf{Búsqueda Ciega}: BFS, DFS
            \item \textbf{Búsqueda Informada}: A*, Greedy Best-First
            \item \textbf{Dijkstra}: Búsqueda de costo uniforme
        \end{itemize}
    }
    \only<2>{
        \textbf{Algoritmos para listas:}
        \begin{itemize}
            \item \textbf{Búsqueda Secuencial}: Listas desordenadas
            \item \textbf{Búsqueda Binaria}: Listas ordenadas
        \end{itemize}
    }
\end{frame}

\begin{frame}{Conceptos básicos de grafos}
    \begin{center}
        \begin{tabular}{ccc}
            \begin{minipage}{2.5cm}
                \centering
                \textbf{Nodo}\\[0.1cm]
                {\scriptsize Elemento fundamental}
            \end{minipage} &
            \begin{minipage}{2.5cm}
                \centering
                \textbf{Arista}\\[0.1cm]
                {\scriptsize Conexion entre nodos}
            \end{minipage} &
            \begin{minipage}{2.5cm}
                \centering
                \textbf{Camino}\\[0.1cm]
                {\scriptsize Secuencia de nodos}
            \end{minipage}
        \end{tabular}
        
        \vspace{0.8cm}
        
        \begin{tabular}{cc}
            \begin{minipage}{2.5cm}
                \centering
                \textbf{No Dirigido}\\[0.1cm]
                {\scriptsize Sin direccion}
            \end{minipage} &
            \begin{minipage}{2.5cm}
                \centering
                \textbf{Ponderado}\\[0.1cm]
                {\scriptsize Con pesos}
            \end{minipage}
        \end{tabular}
    \end{center}
\end{frame}

\begin{frame}{Notación Big O}
    \textbf{Complejidades más comunes:}
    \begin{itemize}
        \item<1-> \textbf{$O(1)$ - Constante}: No depende del tamaño
        \item<2-> \textbf{$O(\log n)$ - Logarítmica}: Muy eficiente
        \item<3-> \textbf{$O(n)$ - Lineal}: Proporcional al tamaño
        \item<4-> \textbf{$O(n^2)$ - Cuadrática}: Como ordenamiento simple
        \item<5-> \textbf{$O(2^n)$ - Exponencial}: Muy ineficiente
    \end{itemize}
\end{frame}

\begin{frame}{Búsquedas No Informadas (Ciegas)}
    \begin{columns}
        \begin{column}{0.6\textwidth}
            \textbf{Características:}
            \begin{itemize}
                \item No tienen información adicional
                \item No tienen función de evaluación
                \item Exploran sistemáticamente
                \item Garantizan encontrar la solución
            \end{itemize}
        \end{column}
        \begin{column}{0.4\textwidth}
            \centering
            {\fontsize{60}{60}\selectfont \faBlind}
        \end{column}
    \end{columns}
\end{frame}

\begin{frame}{Breadth-First Search (BFS)}
    \textbf{Características principales:}
    \begin{itemize}
        \item Explora nivel por nivel
        \item Usa cola FIFO
        \item Garantiza camino mas corto
        \item Complejidad: $O(V + E)$
    \end{itemize}
    
    \vspace{0.5cm}
    
    \textbf{Orden de exploración:}
    \begin{center}
        \begin{tikzpicture}[
            node distance=1.2cm,
            every node/.style={circle, draw, minimum size=0.6cm, font=\footnotesize}
        ]
            \node (0) at (0,0) {0};
            \node (1) at (-1.2,-1.2) {1};
            \node (2) at (1.2,-1.2) {2};
            \node (3) at (-2,-2.4) {3};
            \node (4) at (-0.4,-2.4) {4};
            \node (5) at (0.4,-2.4) {5};
            \node (6) at (2,-2.4) {6};
            
            \draw[thick] (0) -- (1);
            \draw[thick] (0) -- (2);
            \draw[thick] (1) -- (3);
            \draw[thick] (1) -- (4);
            \draw[thick] (2) -- (5);
            \draw[thick] (2) -- (6);
        \end{tikzpicture}
    \end{center}
\end{frame}

\begin{frame}{Ejemplo BFS: Buscando el premio}
    \textbf{Problema:} Tenemos una caja con cajas adentro que no sabemos cuántas hay, ni qué tantas cajas y subcajas tenemos adentro. En algún momento hay un regalo, pero no sabemos cuándo vamos a llegar a él.
    
    \vspace{0.5cm}
    
    \textbf{¿Cómo funciona BFS?}
    \begin{itemize}
        \item Abrimos todas las cajas del primer nivel
        \item Luego todas las cajas del segundo nivel
        \item Y así sucesivamente nivel por nivel
    \end{itemize}
\end{frame}

\begin{frame}{BFS}
    \begin{center}
        \begin{tikzpicture}
            \node[draw, circle, minimum size=1cm, line width=1pt] (regalo) at (0,0) {\faBox};
        \end{tikzpicture}
    \end{center}
\end{frame}

\begin{frame}{BFS}
    \begin{center}
        \begin{tikzpicture}
            \node[draw, circle, minimum size=1cm, line width=1pt] (regalo) at (0,0) {\faBox};
            \node[draw, circle, minimum size=1cm, line width=1pt] (caja1) at (-2,-2) {\faBox};
            \node[draw, circle, minimum size=1cm, line width=1pt, opacity=0] (caja2) at (0,-2) {\faBox};
            \node[draw, circle, minimum size=1cm, line width=1pt, opacity=0] (caja3) at (2,-2) {\faBox};

            \draw (regalo) -- (caja1);
            \draw[opacity=0] (regalo) -- (caja2);
            \draw[opacity=0] (regalo) -- (caja3);
        \end{tikzpicture}
    \end{center}
\end{frame}

\begin{frame}{BFS}
    \begin{center}
        \begin{tikzpicture}
            \node[draw, circle, minimum size=1cm, line width=1pt] (regalo) at (0,0) {\faBox};
            \node[draw, circle, minimum size=1cm, line width=1pt] (caja1) at (-2,-2) {\faBox};
            \node[draw, circle, minimum size=1cm, line width=1pt] (caja2) at (0,-2) {\faBox};
            \node[draw, circle, minimum size=1cm, line width=1pt, opacity=0] (caja3) at (2,-2) {\faBox};

            \draw (regalo) -- (caja1);
            \draw (regalo) -- (caja2);
            \draw[opacity=0] (regalo) -- (caja3);
        \end{tikzpicture}
    \end{center}
\end{frame}

\begin{frame}{BFS}
    \begin{center}
        \begin{tikzpicture}
            \node[draw, circle, minimum size=1cm, line width=1pt] (regalo) at (0,0) {\faBox};
            \node[draw, circle, minimum size=1cm, line width=1pt] (caja1) at (-2,-2) {\faBox};
            \node[draw, circle, minimum size=1cm, line width=1pt] (caja2) at (0,-2) {\faBox};
            \node[draw, circle, minimum size=1cm, line width=1pt] (caja3) at (2,-2) {\faBox};

            \draw (regalo) -- (caja1);
            \draw (regalo) -- (caja2);
            \draw (regalo) -- (caja3);
        \end{tikzpicture}
    \end{center}
\end{frame}

\begin{frame}{BFS}
    \begin{center}
        \begin{tikzpicture}
            \node[draw, circle, minimum size=1cm, line width=1pt] (regalo) at (0,0) {\faBox};
            \node[draw, circle, minimum size=1cm, line width=1pt] (caja1) at (-2,-2) {\faBox};
            \node[draw, circle, minimum size=1cm, line width=1pt] (caja2) at (0,-2) {\faBox};
            \node[draw, circle, minimum size=1cm, line width=1pt] (caja3) at (2,-2) {\faBox};
            \node[draw, circle, minimum size=1cm, line width=1pt] (caja4) at (-2.5,-4) {\faBox};

            \draw (regalo) -- (caja1);
            \draw (regalo) -- (caja2);
            \draw (regalo) -- (caja3);

            \draw (caja1) -- (caja4);
        \end{tikzpicture}
    \end{center}
\end{frame}

\begin{frame}{BFS}
    \begin{center}
        \begin{tikzpicture}
            \node[draw, circle, minimum size=1cm, line width=1pt] (regalo) at (0,0) {\faBox};
            \node[draw, circle, minimum size=1cm, line width=1pt] (caja1) at (-2,-2) {\faBox};
            \node[draw, circle, minimum size=1cm, line width=1pt] (caja2) at (0,-2) {\faBox};
            \node[draw, circle, minimum size=1cm, line width=1pt] (caja3) at (2,-2) {\faBox};
            \node[draw, circle, minimum size=1cm, line width=1pt] (caja4) at (-2.5,-4) {\faBox};
            \node[draw, circle, minimum size=1cm, line width=1pt] (caja5) at (-1.5,-4) {\faBox};

            \draw (regalo) -- (caja1);
            \draw (regalo) -- (caja2);
            \draw (regalo) -- (caja3);

            \draw (caja1) -- (caja4);
            \draw (caja1) -- (caja5);
        \end{tikzpicture}
    \end{center}
\end{frame}

\begin{frame}{BFS}
    \begin{center}
        \begin{tikzpicture}
            \node[draw, circle, minimum size=1cm, line width=1pt] (regalo) at (0,0) {\faBox};
            \node[draw, circle, minimum size=1cm, line width=1pt] (caja1) at (-2,-2) {\faBox};
            \node[draw, circle, minimum size=1cm, line width=1pt] (caja2) at (0,-2) {\faBox};
            \node[draw, circle, minimum size=1cm, line width=1pt] (caja3) at (2,-2) {\faBox};
            \node[draw, circle, minimum size=1cm, line width=1pt] (caja4) at (-2.5,-4) {\faBox};
            \node[draw, circle, minimum size=1cm, line width=1pt] (caja5) at (-1.5,-4) {\faBox};
            \node[draw, circle, minimum size=1cm, line width=1pt] (caja6) at (-0.5,-4) {\faBox};

            \draw (regalo) -- (caja1);
            \draw (regalo) -- (caja2);
            \draw (regalo) -- (caja3);

            \draw (caja1) -- (caja4);
            \draw (caja1) -- (caja5);
            \draw (caja2) -- (caja6);
        \end{tikzpicture}
    \end{center}
\end{frame}

\begin{frame}{BFS}
    \begin{center}
        \begin{tikzpicture}
            \node[draw, circle, minimum size=1cm, line width=1pt] (regalo) at (0,0) {\faBox};
            \node[draw, circle, minimum size=1cm, line width=1pt] (caja1) at (-2,-2) {\faBox};
            \node[draw, circle, minimum size=1cm, line width=1pt] (caja2) at (0,-2) {\faBox};
            \node[draw, circle, minimum size=1cm, line width=1pt] (caja3) at (2,-2) {\faBox};
            \node[draw, circle, minimum size=1cm, line width=1pt] (caja4) at (-2.5,-4) {\faBox};
            \node[draw, circle, minimum size=1cm, line width=1pt] (caja5) at (-1.5,-4) {\faBox};
            \node[draw, circle, minimum size=1cm, line width=1pt] (caja6) at (-0.5,-4) {\faBox};
            \node[draw, circle, minimum size=1cm, line width=1pt, text=green] (caja7) at (0.5,-4) {\faGift};

            \draw (regalo) -- (caja1);
            \draw (regalo) -- (caja2);
            \draw (regalo) -- (caja3);

            \draw (caja1) -- (caja4);
            \draw (caja1) -- (caja5);
            \draw (caja2) -- (caja6);
            \draw (caja2) -- (caja7);
        \end{tikzpicture}
    \end{center}
\end{frame}

\begin{frame}{Depth-First Search (DFS)}
    \textbf{Características principales:}
    \begin{itemize}
        \item Explora una rama completa
        \item Usa pila LIFO
        \item No garantiza camino mas corto
        \item Complejidad: $O(V + E)$
    \end{itemize}
    
    \vspace{0.5cm}
    
    \textbf{Orden de exploración:}
    \begin{center}
        \begin{tikzpicture}[
            node distance=1.2cm,
            every node/.style={circle, draw, minimum size=0.6cm, font=\footnotesize}
        ]
            \node (0) at (0,0) {0};
            \node (1) at (-1.2,-1.2) {1};
            \node (2) at (1.2,-1.2) {4};
            \node (3) at (-2,-2.4) {2};
            \node (4) at (-0.4,-2.4) {3};
            \node (5) at (0.4,-2.4) {5};
            \node (6) at (2,-2.4) {6};
            
            \draw[thick] (0) -- (1);
            \draw[thick] (0) -- (2);
            \draw[thick] (1) -- (3);
            \draw[thick] (1) -- (4);
            \draw[thick] (2) -- (5);
            \draw[thick] (2) -- (6);
        \end{tikzpicture}
    \end{center}
\end{frame}

\begin{frame}{Ejemplo DFS: Buscando el premio}
    \textbf{Problema:} Tenemos una caja con cajas adentro que no sabemos cuántas hay, ni qué tantas cajas y subcajas tenemos adentro. En algún momento hay un regalo, pero no sabemos cuándo vamos a llegar a él.
    
    \vspace{0.5cm}
    
    \textbf{¿Cómo funciona DFS?}
    \begin{itemize}
        \item Abrimos una caja y nos metemos en ella
        \item Si tiene más cajas, seguimos metiéndonos
        \item Solo retrocedemos cuando llegamos al final de una rama
    \end{itemize}
\end{frame}

\begin{frame}{DFS}
    \begin{center}
        \begin{tikzpicture}
            \node[draw, circle, minimum size=1cm, line width=1pt] (regalo) at (0,0) {\faBox};
        \end{tikzpicture}
    \end{center}
\end{frame}

\begin{frame}{DFS}
    \begin{center}
        \begin{tikzpicture}
            \node[draw, circle, minimum size=1cm, line width=1pt] (regalo) at (0,0) {\faBox};
            \node[draw, circle, minimum size=1cm, line width=1pt] (caja1) at (0,-2) {\faBox};

            \draw (regalo) -- (caja1);
        \end{tikzpicture}
    \end{center}
\end{frame}

\begin{frame}{DFS}
    \begin{center}
        \begin{tikzpicture}
            \node[draw, circle, minimum size=1cm, line width=1pt] (regalo) at (0,0) {\faBox};
            \node[draw, circle, minimum size=1cm, line width=1pt] (caja1) at (0,-2) {\faBox};
            \node[draw, circle, minimum size=1cm, line width=1pt] (caja2) at (-1,-4) {\faBox};

            \draw (regalo) -- (caja1);
            \draw (caja1) -- (caja2);
        \end{tikzpicture}
    \end{center}
\end{frame}

\begin{frame}{DFS}
    \begin{center}
        \begin{tikzpicture}
            \node[draw, circle, minimum size=1cm, line width=1pt] (regalo) at (0,0) {\faBox};
            \node[draw, circle, minimum size=1cm, line width=1pt] (caja1) at (0,-2) {\faBox};
            \node[draw, circle, minimum size=1cm, line width=1pt, text=red] (caja2) at (-1,-4) {\faTimes};

            \draw (regalo) -- (caja1);
            \draw (caja1) -- (caja2);
        \end{tikzpicture}
    \end{center}
\end{frame}

\begin{frame}{DFS}
    \begin{center}
        \begin{tikzpicture}
            \node[draw, circle, minimum size=1cm, line width=1pt] (regalo) at (0,0) {\faBox};
            \node[draw, circle, minimum size=1cm, line width=1pt] (caja1) at (0,-2) {\faBox};
            \node[draw, circle, minimum size=1cm, line width=1pt, text=red] (caja2) at (-1,-4) {\faTimes};
            \node[draw, circle, minimum size=1cm, line width=1pt] (caja3) at (1,-4) {\faBox};

            \draw (regalo) -- (caja1);
            \draw (caja1) -- (caja2);
            \draw (caja1) -- (caja3);
        \end{tikzpicture}
    \end{center}
\end{frame}

\begin{frame}{DFS}
    \begin{center}
        \begin{tikzpicture}
            \node[draw, circle, minimum size=1cm, line width=1pt] (regalo) at (0,0) {\faBox};
            \node[draw, circle, minimum size=1cm, line width=1pt] (caja1) at (0,-2) {\faBox};
            \node[draw, circle, minimum size=1cm, line width=1pt, text=red] (caja2) at (-1,-4) {\faTimes};
            \node[draw, circle, minimum size=1cm, line width=1pt, text=red] (caja3) at (1,-4) {\faTimes};

            \draw (regalo) -- (caja1);
            \draw (caja1) -- (caja2);
            \draw (caja1) -- (caja3);
        \end{tikzpicture}
    \end{center}
\end{frame}

\begin{frame}{DFS}
    \begin{center}
        \begin{tikzpicture}
            \node[draw, circle, minimum size=1cm, line width=1pt] (regalo) at (0,0) {\faBox};
            \node[draw, circle, minimum size=1cm, line width=1pt] (caja1) at (-1,-2) {\faBox};
            \node[draw, circle, minimum size=1cm, line width=1pt, text=red] (caja2) at (-1.5,-4) {\faTimes};
            \node[draw, circle, minimum size=1cm, line width=1pt, text=red] (caja3) at (-0.5,-4) {\faTimes};
            \node[draw, circle, minimum size=1cm, line width=1pt] (caja4) at (1,-2) {\faBox};

            \draw (regalo) -- (caja1);
            \draw (caja1) -- (caja2);
            \draw (caja1) -- (caja3);
            \draw (regalo) -- (caja4);
        \end{tikzpicture}
    \end{center}
\end{frame}

\begin{frame}{DFS}
    \begin{center}
        \begin{tikzpicture}
            \node[draw, circle, minimum size=1cm, line width=1pt] (regalo) at (0,0) {\faBox};
            \node[draw, circle, minimum size=1cm, line width=1pt] (caja1) at (-1,-2) {\faBox};
            \node[draw, circle, minimum size=1cm, line width=1pt, text=red] (caja2) at (-1.5,-4) {\faTimes};
            \node[draw, circle, minimum size=1cm, line width=1pt, text=red] (caja3) at (-0.5,-4) {\faTimes};
            \node[draw, circle, minimum size=1cm, line width=1pt] (caja4) at (1,-2) {\faBox};
            \node[draw, circle, minimum size=1cm, line width=1pt] (caja5) at (1,-4) {\faBox};

            \draw (regalo) -- (caja1);
            \draw (caja1) -- (caja2);
            \draw (caja1) -- (caja3);
            \draw (regalo) -- (caja4);
            \draw (caja4) -- (caja5);
        \end{tikzpicture}
    \end{center}
\end{frame}

\begin{frame}{DFS}
    \begin{center}
        \begin{tikzpicture}
            \node[draw, circle, minimum size=1cm, line width=1pt] (regalo) at (0,0) {\faBox};
            \node[draw, circle, minimum size=1cm, line width=1pt] (caja1) at (-1,-2) {\faBox};
            \node[draw, circle, minimum size=1cm, line width=1pt, text=red] (caja2) at (-1.5,-4) {\faTimes};
            \node[draw, circle, minimum size=1cm, line width=1pt, text=red] (caja3) at (-0.5,-4) {\faTimes};
            \node[draw, circle, minimum size=1cm, line width=1pt] (caja4) at (1,-2) {\faBox};
            \node[draw, circle, minimum size=1cm, line width=1pt] (caja5) at (1,-4) {\faBox};
            \node[draw, circle, minimum size=1cm, line width=1pt] (caja6) at (1,-6) {\faBox};

            \draw (regalo) -- (caja1);
            \draw (caja1) -- (caja2);
            \draw (caja1) -- (caja3);
            \draw (regalo) -- (caja4);
            \draw (caja4) -- (caja5);
            \draw (caja5) -- (caja6);
        \end{tikzpicture}
    \end{center}
\end{frame}

\begin{frame}{DFS}
    \begin{center}
        \begin{tikzpicture}
            \node[draw, circle, minimum size=1cm, line width=1pt] (regalo) at (0,0) {\faBox};
            \node[draw, circle, minimum size=1cm, line width=1pt] (caja1) at (-1,-2) {\faBox};
            \node[draw, circle, minimum size=1cm, line width=1pt, text=red] (caja2) at (-1.5,-4) {\faTimes};
            \node[draw, circle, minimum size=1cm, line width=1pt, text=red] (caja3) at (-0.5,-4) {\faTimes};
            \node[draw, circle, minimum size=1cm, line width=1pt] (caja4) at (1,-2) {\faBox};
            \node[draw, circle, minimum size=1cm, line width=1pt] (caja5) at (1,-4) {\faBox};
            \node[draw, circle, minimum size=1cm, line width=1pt, text=red] (caja6) at (1,-6) {\faTimes};

            \draw (regalo) -- (caja1);
            \draw (caja1) -- (caja2);
            \draw (caja1) -- (caja3);
            \draw (regalo) -- (caja4);
            \draw (caja4) -- (caja5);
            \draw (caja5) -- (caja6);
        \end{tikzpicture}
    \end{center}
\end{frame}

\begin{frame}{DFS}
    \begin{center}
        \begin{tikzpicture}
            \node[draw, circle, minimum size=1cm, line width=1pt] (regalo) at (0,0) {\faBox};
            \node[draw, circle, minimum size=1cm, line width=1pt] (caja1) at (-1,-2) {\faBox};
            \node[draw, circle, minimum size=1cm, line width=1pt, text=red] (caja2) at (-1.5,-4) {\faTimes};
            \node[draw, circle, minimum size=1cm, line width=1pt, text=red] (caja3) at (-0.5,-4) {\faTimes};
            \node[draw, circle, minimum size=1cm, line width=1pt] (caja4) at (1,-2) {\faBox};
            \node[draw, circle, minimum size=1cm, line width=1pt] (caja5) at (0.5,-4) {\faBox};
            \node[draw, circle, minimum size=1cm, line width=1pt, text=red] (caja6) at (0.5,-6) {\faTimes};
            \node[draw, circle, minimum size=1cm, line width=1pt] (caja7) at (1.5,-4) {\faBox};

            \draw (regalo) -- (caja1);
            \draw (caja1) -- (caja2);
            \draw (caja1) -- (caja3);
            \draw (regalo) -- (caja4);
            \draw (caja4) -- (caja5);
            \draw (caja5) -- (caja6);
            \draw (caja4) -- (caja7);
        \end{tikzpicture}
    \end{center}
\end{frame}

\begin{frame}{DFS}
    \begin{center}
        \begin{tikzpicture}
            \node[draw, circle, minimum size=1cm, line width=1pt] (regalo) at (0,0) {\faBox};
            \node[draw, circle, minimum size=1cm, line width=1pt] (caja1) at (-1,-2) {\faBox};
            \node[draw, circle, minimum size=1cm, line width=1pt, text=red] (caja2) at (-1.5,-4) {\faTimes};
            \node[draw, circle, minimum size=1cm, line width=1pt, text=red] (caja3) at (-0.5,-4) {\faTimes};
            \node[draw, circle, minimum size=1cm, line width=1pt] (caja4) at (1,-2) {\faBox};
            \node[draw, circle, minimum size=1cm, line width=1pt] (caja5) at (0.5,-4) {\faBox};
            \node[draw, circle, minimum size=1cm, line width=1pt, text=red] (caja6) at (0.5,-6) {\faTimes};
            \node[draw, circle, minimum size=1cm, line width=1pt, text=green] (caja7) at (1.5,-4) {\faGift};

            \draw (regalo) -- (caja1);
            \draw (caja1) -- (caja2);
            \draw (caja1) -- (caja3);
            \draw (regalo) -- (caja4);
            \draw (caja4) -- (caja5);
            \draw (caja5) -- (caja6);
            \draw (caja4) -- (caja7);
        \end{tikzpicture}
    \end{center}
\end{frame}

\begin{frame}{Comparación BFS vs DFS}
    \begin{center}
        \begin{tabular}{cc}
            \begin{minipage}{4cm}
                \centering
                \textbf{BFS}\\[0.1cm]
                {\scriptsize Nivel por nivel}
            \end{minipage} &
            \begin{minipage}{4cm}
                \centering
                \textbf{DFS}\\[0.1cm]
                {\scriptsize Rama completa}
            \end{minipage}
        \end{tabular}
        
        \vspace{0.8cm}
        
        \textbf{¿Cuándo usar cada uno?}
        \begin{itemize}
            \item \textbf{BFS}: Cuando la solución está cerca del inicio
            \item \textbf{DFS}: Cuando la solución está profunda en el árbol
            \item \textbf{Problema}: Nunca sabemos donde está el premio
        \end{itemize}
    \end{center}
\end{frame}

\begin{frame}{Algoritmo de Dijkstra}
    \textbf{Propósito:}
    Encontrar el camino más corto desde un nodo origen a \textbf{todos} los demás nodos en un grafo ponderado con pesos no negativos.
    
    \vspace{0.5cm}
    
    \textbf{Características:}
    \begin{itemize}
        \item Garantiza camino más corto
        \item Funciona con pesos positivos
        \item Complejidad: $O(V^2)$ o $O(E + V \log V)$
        \item Base para algoritmos más avanzados
    \end{itemize}
\end{frame}

\begin{frame}{Búsquedas Informadas}
    \begin{columns}
        \begin{column}{0.6\textwidth}
            \textbf{Características:}
            \begin{itemize}
                \item Tienen información adicional
                \item Usan función de evaluación
                \item Exploran menos nodos
                \item Pueden no ser óptimas
            \end{itemize}
        \end{column}
        \begin{column}{0.4\textwidth}
            \centering
            {\fontsize{60}{60}\selectfont \faBook}
        \end{column}
    \end{columns}
\end{frame}

\begin{frame}{A* (A-Star)}
    \textbf{Función de evaluación:}
    \begin{center}
        $f(n) = g(n) + h(n)$
    \end{center}
    
    \vspace{0.3cm}
    
    \begin{itemize}
        \item $g(n)$: Costo desde inicio hasta $n$
        \item $h(n)$: Heurística de $n$ hasta meta
        \item $f(n)$: Costo total estimado
    \end{itemize}
    
    \vspace{0.5cm}
    
    \textbf{Ventajas:}
    \begin{itemize}
        \item Más eficiente que Dijkstra
        \item Garantiza optimalidad si $h(n)$ es admisible
    \end{itemize}
\end{frame}

\begin{frame}{Greedy Best-First Search}
    \textbf{Función de evaluación:}
    \begin{center}
        $f(n) = h(n)$
    \end{center}
    
    \vspace{0.3cm}
    
    \textbf{Características:}
    \begin{itemize}
        \item Solo considera heurística
        \item No garantiza camino óptimo
        \item Puede ser más rápido
        \item Puede quedar atrapado
    \end{itemize}
    
    \vspace{0.5cm}
    
    \textbf{Comparación con A*:}
    \begin{itemize}
        \item A* considera costo acumulado + heurística
        \item GBFS solo considera heurística
    \end{itemize}
\end{frame}

\begin{frame}{Búsquedas en Listas}
    \begin{center}
        \begin{tabular}{cc}
            \begin{minipage}{4cm}
                \centering
                {\Large \faArrowRight}\\[0.2cm]
                \textbf{Secuencial}\\[0.1cm]
                {\scriptsize $O(n)$ - No ordenada}
            \end{minipage} &
            \begin{minipage}{4cm}
                \centering
                {\Large \faArrowRight}\\[0.2cm]
                \textbf{Binaria}\\[0.1cm]
                {\scriptsize $O(\log n)$ - Ordenada}
            \end{minipage}
        \end{tabular}
    \end{center}
\end{frame}

\begin{frame}{Búsqueda Secuencial}
    \textbf{Características:}
    \begin{itemize}
        \item No requiere ordenamiento
        \item Revisa elemento por elemento
        \item Complejidad: $O(n)$
        \item Peor caso: último elemento
    \end{itemize}
    
    \vspace{0.5cm}
    
    \textbf{Ejemplo:}
    Buscar una figurita de Messi en una bolsa de figuritas del mundial.
\end{frame}

\begin{frame}{Búsqueda Binaria}
    \textbf{Características:}
    \begin{itemize}
        \item Requiere lista ordenada
        \item Divide y conquista
        \item Complejidad: $O(\log n)$
        \item Muy eficiente
    \end{itemize}
    
    \vspace{0.5cm}
    
    \textbf{Ejemplo:}
    Buscar un número en una guía telefónica.
\end{frame}

\begin{frame}{Aplicaciones Prácticas - BFS y DFS}
    \textbf{Breadth-First Search y Depth-First Search:}
    
    \vspace{0.3cm}
    
    \textbf{Ejemplo:} Estamos en París y queremos comer una baguette. Salimos del hotel y nos encontramos con una calle que se llama Rue de Falopini. Buscamos en esa cuadra y no hay nada, giramos a la derecha y buscamos y seguimos girando y buscando...
    
    \vspace{0.3cm}
    
    \begin{itemize}
        \item \textbf{DFS}: Caminar la Rue de Falopini hasta el fondo
        \item \textbf{BFS}: Girar y buscar en la de al lado
    \end{itemize}
    
    \vspace{0.3cm}
    
    \textbf{Aplicación de negocios:} Segmentación de marketing
    \begin{itemize}
        \item \textbf{DFS}: Fitness $\rightarrow$ fitness vegano $\rightarrow$ fitness vegano de embarazadas
        \item \textbf{BFS}: Fitness $\rightarrow$ ropa $\rightarrow$ computadoras
    \end{itemize}
\end{frame}

\begin{frame}{Aplicaciones Prácticas - Dijkstra, A* y GBFS}
    \textbf{Dijkstra, A* y Greedy Best-First Search:}
    
    \vspace{0.3cm}
    
    \textbf{Aplicaciones principales:}
    \begin{itemize}
        \item Rutas de entrega o logística para optimizar costos de transporte
        \item Calcular rutas de transporte de datos entre torres
        \item Optimización de costos en telecomunicaciones
        \item Cualquier problema de optimización de rutas
    \end{itemize}
    
    \vspace{0.3cm}
    
    \textbf{¿Cuándo usar cada uno?}
    \begin{itemize}
        \item \textbf{Dijkstra}: Si no tenemos idea donde está la meta
        \item \textbf{A*}: Si tenemos forma de estimar la meta
        \item \textbf{GBFS}: Para sugerir algo rápido sin importar que sea óptimo
    \end{itemize}
\end{frame}

\begin{frame}{Aplicaciones Prácticas - Búsqueda en Listas}
    \textbf{Búsqueda Secuencial:}
    \begin{itemize}
        \item La usamos cuando no tenemos las cosas ordenadas
        \item Como buscar un elemento en una bolsa
    \end{itemize}
    
    \vspace{0.3cm}
    
    \textbf{Búsqueda Binaria:}
    \begin{itemize}
        \item Para encontrar un momento en el tiempo en que se produjo un evento
        \item Si tenemos un video de seguridad y queremos encontrar cuando se robaron una bici
    \end{itemize}
\end{frame}

\begin{frame}{Ejemplo: Búsqueda en video de seguridad}
    \textbf{Problema:} Encontrar el momento exacto del robo de mi bicicleta y el video es desde el comienzo de los tiempos, el Big Bang.
    
    \vspace{0.5cm}
    
    \textbf{Solución con búsqueda binaria:}
    \begin{center}
        $\text{pasos} = \left[ \log_2 \left( \frac{\text{rango de tiempo}}{\text{precision}} \right) \right]$
    \end{center}
    
    \vspace{0.3cm}
    
    \begin{itemize}
        \item Para encontrar al segundo: 59 pasos
        \item Para encontrar al milisegundo: 69 pasos
        \item Tiempo total: $\approx$ 3 minutos
    \end{itemize}
    
    \vspace{0.5cm}
    
    \begin{center}
        \textbf{¡God bless binary search!}
    \end{center}
\end{frame}

\begin{frame}{Terminamos}
    \begin{center}
        \Large{\textbf{¿Dudas?\\¿Consultas?}}
    \end{center}
    \begin{tikzpicture}[remember picture,overlay]
        \node[anchor=south,inner sep=30pt] at (current page.south) {
            \includegraphics[height=1cm]{../misc/UdeSA.png}
        };
    \end{tikzpicture}
\end{frame}

\end{document}
